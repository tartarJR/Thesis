% Institute of Computer Science thesis template
% authors: Sven Laur, Liina Kamm
% last change Tõnu Tamme 03.05.2019
%--
% Compilation instructions:
% 1. Choose main language on line 55-56 (English or Estonian)
% 2. Compile 1-3 times to get refences right
% pdflatex bachelors-thesis-template
% bibtex bachelors-thesis-template
%--
% Please use references like this:
% <text> <non-breaking-space> <cite/ref-command> <punctuation>
% This is an example~\cite{example}.

\documentclass[12pt]{article}

% A package for setting layout and margins for your thesis 
\usepackage[a4paper]{geometry}

%%=== A4 page setup ===
%\setlength{\paperwidth}{21.0cm} 
%\setlength{\paperheight}{29.7cm}
%\setlength{\textwidth}{16cm}
%\setlength{\textheight}{25cm}


% When you write in Estonian then you want to use text with right character set
% By default LaTeX does not know what to do with õäöu letters. You have to specify
% a correct input and font encoding. For that you have to Google the Web     
%
% For TexShop under MacOS X. The right lines are 
%\usepackage[applemac]{inputenc}
%\usepackage[T1]{fontenc} %Absolutely critical for *hyphenation* of words with non-ASCII letters.
%
% For Windows and Linux the right magic lines are   
% \usepackage[latin1]{inputenc}
% \usepackage[latin5]{inputenc}
%
\usepackage[utf8]{inputenc} %standard encoding since 2018 (can be commented out?)
\usepackage[T1]{fontenc} %Absolutely critical for *hyphenation* of words with non-ASCII letters.

% Typeset text in Times Roman instead of Computer Modern (EC)
\usepackage{times}

% Suggested packages:
\usepackage{microtype}  %towards typographic perfection...
\usepackage{inconsolata} %nicer font for code listings. (Use \ttfamily for lstinline bastype)


% Use package babel for English or Estonian 
% If you use Estonian make sure that Estonian hyphenation is installed 
% - hypen-estonian or eehyp packages
%
%===Choose the main language in thesis
\usepackage[estonian, english]{babel} %the thesis is in English 
%\usepackage[english, estonian]{babel} %the thesis is in Estonian


% Change Babel document elements 
\addto\captionsestonian{%
  \renewcommand{\refname}{Viidatud kirjandus}%
  \renewcommand{\appendixname}{Lisad}%
}


% If you have problems with Estonian keywords in the bibliography
\usepackage[backend=biber,sorting=none]{biblatex}
%\usepackage[style=alphabetic]{biblatex}
%% plain --> \usepackage[style=numeric]{biblatex}
%% abbrv --> \usepackage[style=numeric,firstinits=true]{biblatex}
%% unsrt --> \usepackage[style=numeric,sorting=none]{biblatex}
%% alpha --> \usepackage[style=alphabetic]{biblatex}
%\DefineBibliographyStrings{estonian}{and={ja}}
\addbibresource{references.bib}


% General packages for math in general, theorems and symbols 
% Read ftp://ftp.ams.org/ams/doc/amsmath/short-math-guide.pdf for further information
\usepackage{amsmath} 
\usepackage{amsthm}
\usepackage{amssymb}

% Optional calligraphic fonts    
% \usepackage[mathscr]{eucal}

% Print a dot instead of colon in table or figure captions
\usepackage[labelsep=period]{caption}

% Packages for building tables and tabulars 
\usepackage{array}
\usepackage{tabu}   % Wide lines in tables
\usepackage{xspace} % Non-eatable spaces in macros

% Including graphical images and setting the figure directory
\usepackage{graphicx}
\graphicspath{{figures/}}
\usepackage[section]{placeins}

% Packages for getting clickable links in PDF file
%\usepackage{hyperref}
\usepackage[hidelinks]{hyperref} %hide red (blue,green) boxes around links
\usepackage[all]{hypcap}


% Packages for defining colourful text together with some colours
\usepackage{color}
\usepackage{xcolor} 
%\definecolor{dkgreen}{rgb}{0,0.6,0}
%\definecolor{gray}{rgb}{0.5,0.5,0.5}
\definecolor{mauve}{rgb}{0.58,0,0.82}


% Standard package for drawing algorithms
% Since the thesis in article format we must define \chapter for
% the package algorithm2e (otherwise obscure errors occur) 
\let\chapter\section
\usepackage[ruled, vlined, linesnumbered]{algorithm2e}

% Fix a  set of keywords which you use inside algorithms
\SetKw{True}{true}
\SetKw{False}{false}
\SetKwData{typeInt}{Int}
\SetKwData{typeRat}{Rat}
\SetKwData{Defined}{Defined}
\SetKwFunction{parseStatement}{parseStatement}


% Nice todo notes
\usepackage{todonotes}

% comments and verbatim text (code)
\usepackage{verbatim}


% Proper way to create coloured code listings
\usepackage{listings}
\lstset{ 
  %language=python,                % the language of the code
  language=C++,
  basicstyle=\footnotesize,        % the size of the fonts that are used for the code
  %numbers=left,                   % where to put the line-numbers
  %numberstyle=\footnotesize,      % the size of the fonts that are used for the line-numbers
  numberstyle=\tiny\color{gray}, 
  stepnumber=1,                    % the step between two line-numbers. If it's 1, each line 
                                   % will be numbered
  numbersep=5pt,                   % how far the line-numbers are from the code
  backgroundcolor=\color{white},   % choose the background color. You must add \usepackage{color}
  showspaces=false,                % show spaces adding particular underscores
  showstringspaces=false,          % underline spaces within strings
  showtabs=false,                  % show tabs within strings adding particular underscores
  frame = lines,
  %frame=single,                   % adds a frame around the code
  rulecolor=\color{black},		   % if not set, the frame-color may be changed on line-breaks within 
                                   % not-black text (e.g. commens (green here))
  tabsize=2,                       % sets default tabsize to 2 spaces
  captionpos=b,                    % sets the caption-position to bottom
  breaklines=true,                 % sets automatic line breaking
  breakatwhitespace=false,         % sets if automatic breaks should only happen at whitespace
  %title=\lstname,                 % show the filename of files included with \lstinputlisting;
                                   % also try caption instead of title
  keywordstyle=\color{blue},       % keyword style
  commentstyle=\color{dkgreen},    % comment style
  stringstyle=\color{mauve},       % string literal style
  escapeinside={\%*}{*)},          % if you want to add a comment within your code
  morekeywords={*,game, fun}       % if you want to add more keywords to the set
}


% Obscure packages to write logic formulae and program semantics
% Unless you do a bachelor thesis on program semantics or static code analysis you do not need that
% http://logicmatters.net/resources/ndexamples/proofsty3.html <= writing type rules => use semantic::inference
% ftp://tug.ctan.org/tex-archive/macros/latex/contrib/semantic/semantic.pdf
\usepackage{proof}
\usepackage{semantic} 
\setlength{\inferLineSkip}{4pt}
\def\predicatebegin #1\predicateend{$\Gamma \vdash #1$}

% If you really want to draw figures in LaTeX use packages tikz or pstricks
% However, getting a corresponding illustrations is really painful  


% Define your favorite macros that you use inside the thesis 
% Name followed by non-removable space
\newcommand{\proveit}{ProveIt\xspace}

% Macros that make sure that the math mode is set
\newcommand{\typeF}[1] {\ensuremath{\mathsf{type_{#1}}}\xspace}
\newcommand{\opDiv}{\ensuremath{\backslash \mathsf{div}}\xspace} 

% Nice Todo box
\newcommand{\TODO}{\todo[inline]}

% A way to define theorems and lemmata
\newtheorem{theorem}{Theorem}

\setlength{\parskip}{2mm}

%%% BEGIN DOCUMENT
\begin{document}

%===BEGIN TITLE PAGE
\thispagestyle{empty}
\begin{center}

\iflanguage{english}{%
\large
UNIVERSITY OF TARTU\\%[2mm]
Institute of Computer Science\\
Software Engineering Curriculum\\%[2mm]
}{%
TARTU ÜLIKOOL\\
Arvutiteaduse instituut\\
Informaatika õppekava\\%[2mm]
}%\iflanguage

%\vspace*{\stretch{5}}
\vspace{25mm}

\Large Mustafa Ogün Öztürk

\vspace{4mm}

\huge Evaluating Maintainability of Android Applications: Mooncascade Case Study 

%\vspace*{\stretch{7}}
\vspace{20mm}

\iflanguage{english}{%
\Large Master's Thesis (30 ECTS)
}{%
\Large Magistritöö (30 EAP)
}%\iflanguage

\end{center}

\vspace{2mm}

\begin{flushright}
 {
 \setlength{\extrarowheight}{5pt}
 \begin{tabular}{r l} 
  \sffamily \iflanguage{english}{Supervisor}{Juhendaja}: & \sffamily Jakob Mass, MSc \\
 \end{tabular}
 }
\end{flushright}

%\vspace*{\stretch{3}}
%\vspace{10mm}

\vfill
\centerline{Tartu 2021}

%===END TITLE PAGE

% If the thesis is printed on both sides of the page then 
% the second page must be must be empty. Comment this out
% if you print only to one side of the page comment this out
%\newpage
%\thispagestyle{empty}    
%\phantom{Text to fill the page}
% END OF EXTRA PAGE WITHOUT NUMBER


%===COMPULSORY INFO PAGE
\newpage
%=== Info in English
\newcommand\EngInfo{{%
\selectlanguage{english}
\noindent\textbf{\large Evaluating Maintainability of Android Applications: Mooncascade Case Study}

\vspace*{3ex}

\noindent\textbf{Abstract:}

\noindent
%\textsc{Whitespace}
Android became one of the most comprehensive mobile platforms in the last decade.  This comprehensiveness also brought more challenges to the Android application development. Android’s nature, demanding business needs, the frequent update rate of Android applications, and lastly, changing development teams are the four major challenges for Android applications. Maintainability is defined as how easy it is to update, modify, and maintain software. At this point, maintainability emerges as a key concept because developing maintainable Android applications facilitate the above-mentioned difficulties. The primary goal of this study is to evaluate the impact of the technologies and the methods used to develop Android applications by Mooncascede, a software product development company, on maintainability. These methods and technologies include principles (e.g. Clean Code, SOLID), architectural/design patterns (Clean Architecture, MVVM), and third-party libraries (RxJava, Dagger 2 and so on). The evaluation was conducted using the triangulation strategy, which is a mixed-method approach. Qualitative evaluation was conducted via interviews with the case company's Android team (7 participants) and an Android developer survey filled by anonymous developers (over 150 participants). Also, quantitative evaluation was made via object-oriented software metrics. Study results reveal the positive impact of the evaluated methods and technologies on the maintainability of Android applications while pointing to the need for improvements. Results also indicate the need for a new maintainability model specific to the Android applications.

\vspace*{1ex}

\noindent\textbf{Keywords:} Android, Maintainability, Object-Oriented Metrics, Software Engineering

\vspace*{1ex}

\noindent\textbf{CERCS:} P170 Computer science, numerical analysis, systems, control

\vspace*{1ex}
}}%\newcommand\EngInfo


%=== Info in Estonian

\newcommand\EstInfo{{%
\selectlanguage{estonian}
\newpage
\noindent\textbf{\large Androidi rakenduste hooldatavuse hindamine: Mooncascade juhtumianalüüs}
\vspace*{1ex}

\noindent\textbf{Lühikokkuvõte:} 

\noindent
Android on saanud viimase kümnendi üheks kõikehõlmavaimaks mobiiliplatvormiks.  Samas on see omadus toonud kaasa ka katsumusi Androidi rakenduste arendamisel. Androidi rakenduste puhul on neli peamist proovikivi Androidi olemus, nõudlikud ärivajadused, rakenduste sagedane uuendamine ja muutuvad arendusmeeskonnad. Hooldatavus tähistab seda, kui lihtne on tarkvara uuendada, muuta ja hooldada. Praegu on sellest kujunemas peamine parameeter, sest hooldatavate Androidi rakenduste arendamine leevendab eelmainitud probleeme. Uurimuse põhieesmärk on hinnata Androidi rakenduste arendamisel tarvaarendusettevõtte Mooncascade kasutatud tehnoloogia ja meetodite mõju hooldatavusele. Need meetodid ja tehnoloogia hõlmavad põhimõtteid (nt Clean Code ehk puhas kood, SOLID), arhitektuurilisi ja disainimustreid (Clean Architecture ehk puhas arhitektuur, MVVM) ning kolmandate poolte teeke (RxJava, Dagger 2 jne). Hindamisel kasutati mitut meetodit hõlmavat triangulatsioonistrateegiat. Kvalitatiivse hindamise käigus tehti intervjuud uuritava ettevõtte Androidi meeskonnaga (7 osalejat) ja korraldati anonüümne küsimustik Androidi arendajatele (üle 150 osaleja). Lisaks tehti objektorienteeritud tarkvara näitajate kvantitatiivne hindamine. Uurimuse tulemustest nähtub hinnatud meetodite ja tehnoloogia positiivne mõju Androidi rakenduste hooldatavusele, aga ka täiustusvajadus. Samuti selgus tulemustest vajadus uue hooldatavusmudeli järele, mis on spetsiifiline Androidi rakendustele.

\vspace*{1ex}

\noindent\textbf{Võtmesõnad:} Android, hooldatavus, objektile suunatud mõõdikud, tarkvaratehnika

\vspace*{1ex}

\noindent\textbf{CERCS:} P170 Arvutiteadus, arvanalüüs, süsteemid, kontroll

\vspace*{1ex}
}}%\newcommand\EstInfo

%=== Determine the order of languages on Info page
\iflanguage{english}{\EngInfo}{\EstInfo}
\iflanguage{estonian}{\EngInfo}{\EstInfo}

\newpage
\tableofcontents

% Remember to remove this from the final thesis version
\newpage
\listoftodos[Unsolved issues]
% END OF TODO PAGE 

\newpage
\section{Introduction}
\label{section:1}
In the last decade, the impact of smartphones on our lives has significantly increased, and smartphones and mobile applications became people's primary way of interacting with technology. This situation has made the applications that work on these smartphones a vital part of daily and business life, from ordinary people to large companies. Today, mobile applications have become one of the most critical parts of digitalisation. Notably, as a successful open-source mobile operating system, Android has been a core element of this change, and the demand for Android applications has increased. Android application development has become one of the most necessary parts of the business area with a significant market share. Today, there are more than 2.5 billion active Android devices in the world \cite{1} . Therefore, it is difficult to ignore the importance of Android application development, considering the Android operating system's market share and many people's interactions with mobile applications running on Android devices. Consequently, Android application development has become an essential topic in the IT industry and academy. However, the increasing importance of the mobile era also brought more challenges to mobile application development and, of course, to Android application development.

Today, the difficulties that arise during Android application development can be examined under four major topics. These are Android's nature and its platform-specific components, demanding business requirements and sophisticated business needs, the frequent update rate of Android applications, and lastly, growing codebases and fast-changing development teams. When developing Android applications, it is essential to facilitate these difficulties. To overcome these difficulties, "maintainability" emerges as one of the most important non-functional requirements when developing Android applications. While developing Android applications, it is wise to use the technologies and techniques to increase the Android application's maintainability. Developing high-quality applications that can survive in the competitive Android market and achieving these goals in a time and cost-effective manner is only possible in this way.

This study's primary purpose is to identify and explain the methodologies and technologies used by Mooncascade, a leading software product development company with global reach, to develop real-world enterprise Android applications and evaluate these methodologies and technologies in terms of maintainability. To that aim, both qualitative analyses (in the forms of interviews, questionnaires for measuring the impact from the software complexity point of view) and quantitative analysis (measuring maintainability via object oriented software quality metrics) will be conducted.

Considering the scarcity of academic studies in this field and the inadequacy of these studies' content, and actuality problems, it can be said that this study will make a significant contribution to the academic studies in the Android field. The inadequacy and not being up-to-date with academic studies in the Android field are among the main motivations for creating this study. Another important source of motivation for this study is the negative effects of the problems arising from the lack of maintainability in Android application development on the developer experiences and the elimination of these negative effects and thus, to improve  Android developer experience.

The target audience of this study includes Android developers and researchers who are already experienced with Android application development basics and willing to learn advanced techniques and tools for Android application development. The study facilitates developers and researchers to follow the most up-to-date software engineering and Android development practices to develop state-of-the-art Android applications with high maintainability. The study only focuses on native Android application development.

\subsection{Problem Statement}
Mooncascade is a leading product development company with global reach, based in Estonia. Mooncascade aims to provide quality at every stage of the software product development process, helping clients build innovative solutions that inspire, disrupt, and challenge markets worldwide. The company provides various software development services, including web development, data science, quality assurance, and mobile application development. 

The Android development team is one of the most active units of Mooncascade. The team has been working with clients from different industry domains and developing countless Android applications for clients. From the client's point of view, Mooncascade's Android team's biggest goal is developing Android apps with high quality. High quality means, following the latest industry trends and technologies, improving and providing a pleasant mobile application experience for customers and end-users. From the software engineering perspective, essential goals are enhancing developer and team efficiency and increasing code quality and maintainability. These are the goals that will constitute this study's main subject since they are directly related to the four significant challenges encountered when developing Android applications, which are mentioned in the introduction section. To have a better understanding of those challenges, it is wise to explain them in a bit more detail. 

\noindent\textbf{Android platform-specific complexity:} Android applications are distinguished from traditional web and desktop software with their sophisticated features and specific structure. They need to fulfill some platform-specific requirements. Apps should work per Android OS, and they have to use components offered by the Android Software Development Kit, such as Activity, Fragment, Service. A typical Android app consists of components such as views, activities, fragments, services, broadcast receivers, and content providers. These components are unique to the Android platform and provided by the Android SDK. The Android system mostly controls the behaviors of these components. Android developers must obey the contract supplied by the system when using these components, and this situation sometimes limits developers to use some specific programming techniques when developing Android applications. These components are directly related to the Android OS, and they should be separated from the other possible layers to decrease the level of complexity and increase maintainability. Nevertheless, that is not always the case, and sometimes this separation is not as easy as it sounds. Besides, there are other components used in an Android app that are not a part of the Android SDK for managing network operations, database transactions, and business rules. Whether it is a part of the Android SDK or not, each of these components represents a different "concern" of an Android application. The necessity of working together in harmony for all these components, the challenges that arise from the Android operating system and Android SDK's nature, and the limited resources of mobile devices make Android applications complicated software systems that are hard to develop. 

\noindent\textbf{Business-specific complexity:} Android applications get more and more sophisticated to fulfill increasing user needs and business requirements. Mobile applications become more functional as user and business needs increase. Consequently, the complexity of Android apps from the software development point of view increases, and apps become more business-critical \cite{2}. When this business-specific complexity comes together with the platform-specific complexity mentioned above, it makes developers’ jobs even harder as its influence on development is strong. In this regard, it makes sense to separate the business-specific requirements and logic from the rest of the application to improve the maintainability of the application and ease complexity related to business-specific needs.

\noindent\textbf{High update rate:} Android applications have a high update rate because of bug fixes and the frequent addition of new features based on the changing business requirement and user needs. That makes the software development life cycles of the Android applications quite active \cite{3}. For that reason, Android applications should be developed so that the addition of new features and fixing bugs can be done smoothly. Thus, problems can be solved in a time and cost-efficient manner. In that way, decreasing maintenance costs, shortening release times, and improving developer efficiency while developing Android applications could be possible.

\noindent\textbf{Growing Codebases and Fast-changing Development Teams:} Android applications' gets harder to maintain as the codebase and the development team grows or changes. The codebase of an Android application has to be in an orderly fashion that enables developers to read and understand the app's purpose quickly. Also, any time a new developer joins the team, the time required to onboard a new developer to the codebase is directly related to the way that the Android applications were developed.

Over the last decade, a couple of different ideas have been in place to resolve these issues in the context of Android application development, and similar views with the same purpose are continually evolving. However, the primary purpose of all these ideas is the same. All these ideas and methodologies, whether in the form of software architecture, design patterns, or coding conventions, aim to improve the "maintainability" of the Android applications.

In the context of Mooncascade's Android team, from the maintainability point of view, the challenges explained above become even more critical since the company provides services in the forms of software development and consultancy, and the reason for this is how Mooncascade works. In other means, the team is divided into sub-teams, and these sub-teams work on different projects for different clients from different domains. Over time members of the sub-teams can change, a project can be extended, or maintenance might be needed. In this situation, the extension or maintenance of a project might need to be done with different developers. That is the point where maintainability emerges as a critical non-functional requirement. High maintainability means better code readability and understandability, less onboarding time for a new developer, easily extendable and changeable code. If projects are developed with high maintainability, such a process can be managed more time and cost-efficiently.

The team had internalized some set of tools, techniques, and technologies over time to that aim. Although these tools, techniques, and technologies used by Mooncascade's Android team are used by the Android community worldwide, the impact and benefits are empirically unknown from the software engineering perspective. Knowing the benefits and effects of these tools, techniques, and technologies used by the team in software maintainability is essential to understand how useful and practical they are. Thus, to identify what these tools, techniques, and technologies used by Mooncascade's Android team are and what benefits and impacts they bring, this study will answer the following research questions. 

\begin{itemize}
\item \noindent\textbf{RQ1:} What are the metrics and/or methods for measuring maintainability in the context of Android application development?
\item \noindent\textbf{RQ2:} What are the methods, techniques, tools, and technologies used by Mooncascade's Android team to develop quality Android applications?
\item \noindent\textbf{RQ3:} How efficient and impactful are the methods, techniques, tools, and technologies used by Mooncascade's Android when developing Android applications, in terms of increasing software maintainability?
\end{itemize}

\subsection{Scope and Goal}
This study addresses the practices used by Mooncascade’s Android team to resolve the challenges mentioned in the previous section. To this end, the study aims to present comprehensive and up-to-date resources used by Mooncascade's Android team, including the tools, libraries, and techniques and how those are used to achieve the goal. The study will identify, understand, and share the practices and technologies used by Mooncascade's Android team. Moreover, it will present the determined practices in the forms of code samples and instructions. In this way, the study aims to facilitate resolving the challenges faced when developing state-of-the-art Android applications, which are also mentioned in the introduction section, by providing advanced techniques for developers and researchers.  

Moreover,  as a part of this study, interviews will be conducted amongst the Android developers of Mooncascade's Android team to evaluate the impact of the researched practices and technologies. Researched methods will also be compared to the data collected through an Android developer survey conducted amongst the Android community to support the validity and up-to-dateness. Lastly, the identified and studied practices will be evaluated from the maintainability point of view by using software quality metrics.

\subsection{Contributions}
The rest of this document is organized as follows. In Chapter 2, brief information about the Android platform and Android SDK is given. The chapter continues with key concepts for the maintainability of software systems. Lastly, the methods and technologies used by Mooncascade's Android team to tackle the problems mentioned in the first chapter are presented. In Chapter 3, why and how the literature review was carried out within the scope of this study and the results of this literature review is shared. Also, a brief analysis of the results is presented. In Chapter 4, the methods used to achieve the primary goal of this study are presented. The contents of the survey and interviews are discussed in detail. Also, metrics used while evaluating methods and technologies used by Mooncascade's Android team in terms of maintainability are explained. Thus, the first research question will be answered in this chapter. In Chapter 5, the results for the impact of the practices that are used by Mooncascade's Android team on the maintainability of Android applications are shared. The results obtained from the Android developer survey, interviews with the case company's Android developers and evaluation with the object-oriented metrics are presented as well. Chapter 6 presents the discussion and interpretation of answers to the research questions and evaluation results obtained in the previous chapters. Outcomes of the evaluations will be shared. Also, the limitations of the study will be mentioned. The last chapter concludes this thesis, and future research opportunities are discussed.



\subsection{Thesis Outline}
The rest of this study is structured as follows:

\noindent\textbf{2. Background:}
This chapter starts with brief information about the Android platform and Android SDK. Later, the chapter continues with key concepts for the maintainability of software systems. Lastly, the methods and technologies used by Mooncascade's Android team to tackle the problems mentioned in the first chapter are presented.

\noindent\textbf{3. Research Methodology}

In this section, the methods used to achieve the primary goal of this study are presented. The contents of the survey and interviews are discussed in detail. Also, metrics used while evaluating methods and technologies used by Mooncascade's Android team in terms of maintainability are explained. Thus, the first research question will be answered in this chapter.

\noindent\textbf{4. Evaluation}

The results for the impact of the practices that are used by Mooncascade's Android team on the maintainability of Android applications are shared in this chapter. The results obtained from the Android developer survey, interviews with the case company's Android developers and evaluation with the object-oriented metrics are presented, respectively.

\noindent\textbf{5. Discussion}

This chapter will present the discussion and interpretation of answers to the research questions and evaluation results obtained in the previous chapters. Outcomes of the evaluations will be shared. Also, the limitations of the study will be mentioned.

\noindent\textbf{6. Conclusion}

The study's epitome and results, along with the final thoughts and comments, will be presented in this section. Also, future research opportunities will be discussed.


\newpage
\section{Background}
\label{section:2}
This chapter covers the key software engineering and Android concepts. Also, the way of Android development at Mooncascade is presented. The chapter aims to facilitate the understanding of maintainability, which is the focal point of this study.

\subsection{Android Platform}
\subsubsection{Android OS}
Android is an open-source operating system for mobile devices. The Android project was initially created by the Open Handset Alliance which includes organizations from various industries such as Google, Vodafone, T-Mobile, LG, Huawei, Asus, Acer, and eBay to give some examples \footnote{\url{http://www.openhandsetalliance.com/oha_members.html}}. The main goal of the Android project is to provide an open software platform accessible for a variety of stakeholders such as developers, engineers, carriers, and device manufacturers to turn their innovative and imaginative ideas into successful real-world products that improve the mobile experience for the end-users. Today, numerous organizations from Open Handset Alliance and also other organizations are supporting and investing in Android and the project is led by Google. Android is designed in a distributed way to avoid the issue of the central point of failure. In another means, different industry players confine or control the advancements of another. As a result, a production-quality consumer product comes along with open source code that is ready for customization.

\subsubsection{Fundementals of Android Applications}
The Android operating system's APIs written in Java programming language are provided through this layer. These APIs are the core building blocks for the developers in order to develop Android applications. A couple of noticeable key core building blocks include the following:
\begin{itemize}
\item View System: It provides an extensible set of views for creating user interfaces and user experience.
\item Resource Manager: Providing tools for accessing and managing non-code embedded application resources such as strings, colors, values, layouts, images, etc.
\item Activity Manager: It provides the tools for managing the application lifecycle and navigation back stack.
\item Content Providers: It provides tools for enabling data sharing between different applications.
\item Notification Manager: It provides tools for developers to add the ability of notifications and alerts for Android applications.
\item Location Manager: Provides tools for developers to manage location-related data and location updates.
\end{itemize}
The practices that will be covered in this study mainly occur on this layer because the code is developed and structured in the Java API Framework layer.

\subsection{Key Software Engineering Concepts}
This study focuses on solving the maintainability problems of Android applications. However, it is essential to know the fundamental components that create an Android application to really understand the problem.

Since the launch of the first mobile device that works with Android,  the Android operating system has improved, and the way the Android applications are developed changed a lot, but some fundamental components for developing Android applications have more or less stayed the same. Each of these fundamental components was developed for a specific purpose by the creators of the Android Software Development Kit (Android SDK). Knowing these components and their responsibilities are necessary to understand the problem that this study is trying to solve. %Because as it was already stated in the introduction section, one of the reasons for the complications that arise when developing Android applications is the nature of the fundamental Android components.

%Ever since the Android operating system has started running on mobile devices, Android applications are being developed. As of the second quarter of 2021, there are more than two and a half million applications in the Google Play Store \footnote{\url{https://www.appbrain.com/stats/number-of-android-apps}}.Since the launch of the first mobile device that works with Android,  the Android operating system has improved, and the way the Android applications are developed changed a lot, but some fundamental components for developing Android applications have more or less stayed the same. Each of these fundamental components was developed for a specific purpose by the creators of the Android Software Development Kit (Android SDK). Knowing these components and their responsibilities are necessary to understand the problem that this study is trying to solve. Because as it was already stated in the introduction section, one of the reasons for the complications that arise when developing Android applications is the nature of the fundamental Android components. So, in this section, some brief information will be given about these fundamental Android components. For more information and technical details regarding these components, it is recommended to read official Android documentation.

Java, Kotlin, and C++ programming languages can be used for developing native Android applications. There are also other ways of developing Android applications. But as it was already mentioned in the introduction section, this study only focuses on native Android application development. Native Android development means the creation of Android applications that run on Android-powered devices by using the Android Software Development Kit. When developing Android applications in a native way, in addition to the programming side, Android applications are supported by different types of resources such as XML layout files, XML resources, images, data files, etc. The detailed examination of these resources is not within the scope of this study. However, knowing that Android applications do not only consist of code might be useful to see the bigger picture of an Android application. Though, for the purpose of understanding the problem that this study tries to resolve, it is essential to have a basic understanding of the fundamental Android components. Consequently, knowing the nature of an Android application and its components is the first step for solving the maintainability issues of the Android applications and then there come the best practices and latest technologies of Android application development and how to apply them into the Android application development processes. These fundamental components can be listed as:
\begin{itemize}
    \item Activities
    \item Services
    \item Broadcast receivers
    \item Content providers
\end{itemize}

In addition to these main components, the Fragment component also has an important place among Android components. Knowing the details of all these components (e.g. implementation details, life-cycle) plays an important role in solving the complications caused by Android components while developing Android applications. Details regarding key Android components and their responsibilities are beyond the scope of this study. However, as stated before, knowing the nature, responsibilities and working methods of these key components will facilitate understanding of the issues addressed in this study. For this reason, the summary information in this section has been shared to address these components, albeit briefly, and to raise awareness about their effects on the maintainability of Android applications. Therefore, examining the official Android documentation in this area can help to understand the study better \footnote{\url{https://developer.android.com/guide/components/fundamentals}}.

%The remaining of this section will introduce fundamental Android components and the brief information regarding these components to help to understand the problem that is stated in the study. Since the impact of activities is bigger when it comes to the maintainability of Android applications, the information to be provided regarding this component is slightly more detailed when compared to the information regarding the rest of the components.

%Since the impact of activities and fragments is bigger when it comes to Android applications' maintainability, the information was shared regarding these components in the upcoming sections. For the rest of the components, Android's official documentation can be examined \footnote{\url{https://developer.android.com/guide/components/fundamentals}}. 

\subsection{Android Development at Mooncascade}
\label{section:2.3}
Mooncasade's Android team has made important strides in standardizing the methods and technologies they recently use. Therefore, up-to-date methods and technologies that can be used effectively by all team members have been determined, and different developers have applied these methods and technologies in different projects. The biggest reason for this change is undoubtedly the time and cost-effective development of Android applications and the increase in the development teams' effectiveness. In fact, looking at the general situation, it can be said that the aim is to develop maintainable, high-quality Android applications. In this section, brief information about the methods and technologies used by Mooncascade's Android team while developing Android applications are presented. Sharing information about these methods and technologies at the baseline level is important for this study because qualitative and quantitative evaluations of their impact on maintainability and obtaining results on these effects are the primary purpose of this study.

First of all, the team uses Kotlin programming language for Android development. It is believed that the use of Kotlin can improve code quality, readability, and productivity \cite{44}. It is also the supported programming language by Google \cite{43}. For the same reasons the team prefers Kotlin over Java. Also, the team tries to adapt the Clean Code and SOLID principles to their daily programming tasks. Clean Code principles are believed to be to improve the readability and understandability of the code \cite{46}. Also, the application of SOLID principles facilitates the separation of concerns and improves code quality \cite{26}. The team applies these principles for these reasons and tries to maximize their application by code reviews.

When it comes to the architecture of Android application, the team prefers Clean Architecture\footnote{\url{https://blog.cleancoder.com/uncle-bob/2012/08/13/the-clean-architecture.html}}. Clean Architecture provides a high level of separation of concerns \cite{56} and facilitates changing third-party dependencies and implementation details of the software systems that it is used on. Also, it helps to modify the different layers of the application without affecting the business logic. Thus, it makes Android applications easy to understand, modify and test \cite{47}. The team also uses Model-View-View Model (MVVM) design pattern to present data and applies this design pattern with the help of tools provided by the Android Architecture Components framework published by Google's Android team\footnote{\url{https://developer.android.com/topic/libraries/architecture}\label{ft:arch-components}}. With the increase of the scale and the complexity of Android applications, a proper design pattern for the presentation of the data became essential for Android applications to achieve high cohesion and low coupling. Such design patterns enable the different degrees of separation for data, logic and view concerns. Also, an important characteristic for an efficient design pattern is eliminating the bidirectional dependency of views and view models. Thus decoupling of data and view becomes possible, and an important software design objective of high cohesion and low coupling is achieved. The MVVM design pattern provides these requirements for Android applications \cite{48}.

The team also uses some third-party Android and Java/Kotlin libraries to develop Android applications. Although such libraries are not a must when developing Android applications, the use of some of them saves Android developers a lot of time and effort. However, the team tends to reduce the use of third-party libraries as much as possible and prefers to use raw solutions whenever possible. Community support, up-to-dateness, reliability and long-term maintainability criteria are considered when selecting the third-party libraries used. RxJava\footnote{\url{https://github.com/ReactiveX/RxJava}}, Dagger 2\footnote{\url{https://dagger.dev/}}, Retrofit\footnote{\url{https://github.com/square/retrofit}}, Apollo\footnote{\url{https://www.apollographql.com/docs/android/}} and Android Architecture Components$^{\ref{ft:arch-components}}$ are the most prominent libraries/frameworks and have the most impact on concepts such as Android application architecture and maintainability. RxJava is a library for designing asynchronous and event-based software applications by using observable sequences. Dagger is a static, compile-time framework used for dependency injection and it can be used by Java, Kotlin, and Android based applications. Retrofit is used to integrate REST-based back-end systems to Android applications, while Apollo is used for the integration of GraphQL based back-end systems. Lastly, Android architecture components are a set of libraries that facilitate designing robust, testable, and maintainable apps.





\newpage
\addtocontents{toc}{\protect\newpage}
\section{Research Methodology}
\label{section:3}
A literature review was carried out within the scope of this study to examine previous studies conducted on the maintainability of Android applications and obtain more comprehensive information on maintainability measurement. Considering the tight relationship between the topic and industry, Systematic Literature Review (SLR) was not found sufficent for finding relevant resources of data for the study. Hence, a Multivocal Literature Review (MLR) was conducted. As a type of SLR, MLR is collecting grey literature as well alongside formal literature \cite{40}. MLR considers resources like blogs, books, articles, academic literature and allows gathering information from academics, developers, practitioners, and independent researchers \cite{41}.

\subsection{Literature Search Query}
When other academic studies dealing with the measurement of maintainability in Android or other software systems are considered, it can be seen that many studies use scientific and quantitative methods. The work of Verdecchia et al. in the maintainability and architecture of Android applications can be shown as a successful example of this situation \cite{14}. Although it cannot be claimed that such quantitative measurements are wrong, it would not be wrong to say that these measures are inadequate at times. It is essential to make qualitative evaluations, and quantitative evaluations in areas where technologies are rapidly developing and trends change quickly, especially in Android application development. In this way, it may be possible to measure developers' experiences that differ in the face of rapid change and development and the effects of these experiences on the maintainability issue we are working on more efficiently. For these reasons, it was deemed appropriate to add a qualitative evaluation technique to this study’s scope. An Android developer survey and some interviews were conducted within this study's scope as a part of qualitative evaluations. The contents and purposes of these surveys and conferences are discussed in detail in their respective sub-sections below.

\subsubsection{Android Developer Survey}
The first step of qualitative evaluations in this study is the Android developer survey. The consistency and stability of principles and third-party libraries used in Android applications indirectly affect the maintainability of applications. Therefore, although it does not directly contribute to measuring maintainability, this survey was conducted to identify current Android trends and provide support for this study's evaluation. While determining this survey's questions, priority was given to principles and technologies that directly and indirectly affect maintainability. Although the questions are generally prepared to cover the methods used by the Mooncascade Android's team, it can be said that the questions reflect the Android technology stack in general. The content of the survey can be accessed publicly\footnote{\url{https://forms.gle/MoiTGMV874yzJwuv8}}. The questions are organised with the help of the Forms application provided by Google. The Android developer survey has been delivered to Android developers from different companies in different countries through accounts or groups of Android developer communities on social media platforms such as LinkedIn, Discord, and Twitter. The author of the study has also shared the survey with many of his colleagues/ex-colleagues working in different companies/countries.

\subsubsection{Interviews with Team Members}
The second step of the qualitative evaluations carried out within this study’s scope is the interviews conducted with Mooncascade's Android team members. Unlike the Android developer survey that was explained in the previous section, these interviews are designed and conducted to qualitatively evaluate the techniques and technologies used by Mooncascade's Android team in terms of maintainability. Also, the interviews aim to determine the importance of maintainability from the case company's point of view.  Thus, proving or disproving the study's claim that maintainability is a key concept to overcome the issues mentioned in the problem statement section would be possible. Eight questions were determined for this purpose. The interview questions can be accessed publicly\footnote{\url{https://forms.gle/paMSj5e8Up5ZW6raA}}. Three main criteria were taken into consideration while preparing these questions. First of all, questions were chosen to get to know about the participants' background and experience. Later, some questions were designed to learn participants' understanding of maintainability in software engineering. Lastly, questions were drafted to learn about participants' thoughts about the impact of technologies and principles used by Mooncascade on maintainability. The interview was conducted privately with each team member.  It is aimed that the data gathered through these interviews will support the accuracy and validity of evaluations.


\subsection{Results}
\label{section:3.2}
Quantitative evaluation constitutes the most critical part of this study in terms of measuring maintainability. In this study, many other studies on the measurement of maintainability in Android and software engineering were examined. The purpose of these reviews is to find the most appropriate maintainability measurement metrics. Various metrics can be used to evaluate object-oriented software systems in terms of quality and maintainability. The most popular of these metrics are detailed in Barak et al., (2012) \cite{33}. As a result of this examination, it was seen that the concepts of complexity, cohesion and coupling were emphasised in many studies, and it was concluded that the measurements made based on these concepts would be more efficient when measuring maintainability. The effects of these concepts on maintainability have been mentioned in the second chapter, and detailed studies in this field have been referred to. Studies have shown that results retrieved from evaluating these concepts proved to define the level of maintainability \cite{33}.

The definition of complexity in software engineering is the difficulty to understand the interactions between the parts of a software system. Higher levels of complexity in software increase the risk of accidentally preventing interactions, increasing the chance of introducing bugs when making changes, thus decreasing maintainability. High coupling between classes also causes complexity when maintaining the software. Changes done in a class reflect the dependent class due to the dependency relationship of the classes. Thus, the software system becomes challenging to maintain. In software engineering, cohesion is how well the methods of a class are related to each other. While the classes' relationship is desired to be loosely coupled, their methods and data fields are desired to be related. Lack of cohesion threatens modularity and software maintenance. 
\begin{figure}[ht!]
    \centering
    \includegraphics[scale=0.5]{figures/maintainability_factors.png}
    \caption{Relationship between cohesion, coupling, complexity and maintainability \protect\cite{33}}
    \label{fig:maintainability_factors}
\end{figure}

As can be seen, complexity, cohesion and coupling are in a tight relationship among themselves, and they all directly affect software maintainability. The figure below shows the relationship between maintainability, complexity, cohesion and coupling. Therefore, based on this result, research was done on measuring the concepts of complexity, cohesion and coupling effectively, and the most suitable metrics for measurement were determined. As a result of this research, five metrics that will enable an object-oriented software system to be evaluated effectively in terms of complexity, cohesion and coupling have been determined. While selecting these metrics, priority has been given to metrics that can handle a software system as a whole in the areas of complexity, cohesion and coupling. This enables the system to be evaluated in terms of maintainability in the most effective way. To emphasise again, although these metrics are used in measuring complexity, cohesion and coupling, they allow maintainability to be measured directly due to the tight relationship between these concepts and software maintainability. These metrics and their intended use are listed below. 
\begin{itemize}
    \item \textbf{Weighted Method Count (WMC):} This metric is used to measure object-oriented software systems’ complexity. WMC represents a class's cyclomatic complexity, also known as McCabe complexity \cite{35}. It, therefore, portrays the complexity of a class as a whole, and this measure can be used to indicate the maintainability level of the class. The number of methods and complexity can be used to divine maintaining effort. If the number of methods is high, that class is described as domain-specific and is less reusable. Also, such classes tend to be prone to change and defects.
    \item \textbf{Depth of Inheritance Tree (DIT):} This is another metric to measure software complexity. Inheritance increases software reusability; however, one side can create complexity by possibly violating encapsulation since the subclass needs to access the superclass. Furthermore, changes made during maintenance might increase the inheritance tree's depths by adding more children. Therefore, by assessing the inheritance tree available in the product, it is easy to predict how much effort needed to make it stable \cite{33}. It is harder to predict its behaviour if the tree depth is high, and this causes maintenance issues.
    \item \textbf{Number of Children (NOC):} NOC measures the number of descendants of a class, and it is used to measure the coupling level for the corresponding class. NCO also indicates the reusability level of a software system. It is assumed that the number of child classes and the maintainer's responsibility to maintain the children's behaviour are directly proportional. If the NOC level is high, it is harder to maintain and modify the class \cite{36}.
    \item \textbf{Coupling Between Object Classes (CBO):} This metric calculates the number of connections to other classes from a particular class, and it is used to measure coupling. A class is considered coupled if it depends on another class to get its work done \cite{34}. CBO metric is related to the reusability of the class. High coupling makes the code more difficult to maintain because changes in other classes can also affect that class. Therefore these classes are less reusable and less maintainable.
    \item \textbf{Coupling Between Object Classes (CBO):} This metric is used to determine how class methods are related to each other, and it is applied to evaluate cohesion. Cohesion promotes the maintainability of the software systems. High cohesion for a class meant the class is understandable, maintainable and easy to modify \cite{33}.
\end{itemize}

Apart from the above metrics, many other metrics can measure quality and maintainability in object-oriented software systems. The most well-known of these metrics are the Line of Code (LOC) and Halstead Effort metrics. Studies conducted using these metrics in Android, and other software development fields were examined within this study's scope. The methods related to size were not preferred because they were too classical, and they do not give very effective results on maintainability. Halstead complexity metrics were not preferred because they concentrated more on the complexity of classes and methods. Regarding the use of these metrics, Prabowo's work on Android apps' maintainability can be examined \cite{19}. Instead, metrics that facilitate evaluating the quality and maintainability of software systems as a whole were preferred.

After determining the metrics to be used, research has been conducted on how these metrics will be applied. As a result of the research, it has been concluded that metrics can be applied manually or with a tool's help. Since manual implementation will take more time and is prone to error, it has been decided to apply metrics with a reliable tool. Later, a search was done for a static code analysis tool to work with Android Studio and Kotlin programming languages and support the selected metrics. During the research, the CodeMR static code analysis tool drew attention. CodeMR is a powerful software quality tool that is integrated with IDEs and supports multiple programming languages. Java, Scala, Kotlin and C++ \cite{37}. The tool provides an understanding of software quality through its metrics to measure coupling, complexity, cohesion and size. These metrics are often affected by various code characteristics, making them promising for evaluating software maintainability.  Besides, the tool provides a visualisation centric approach and generates detailed reports supported by different visualisation options. In this way, it facilitates the application of metrics and makes the results more understandable with detailed reports and advanced visualisation techniques. The tool can also be installed as a plugin in Android Studio and is very easy to use. Apart from that, the tool also makes it possible to measure with many other metrics. The complete list of metrics and other features that the tool offers can be accessed via the tool's documentation. Finally, 2 academic studies conducted using this tool were examined before the tool was started to be used, and information about the tool operation was obtained\cite{38}\cite{39}. Finally, the CodeMR static code analysis tool has been chosen to be used in this study due to the support of Kotlin, its ability to be installed as an add-on to Android Studio, its support for selected metrics, and its advanced visualisation and reporting mechanisms. After this decision, communication was established with the CodeMR team, and a free license was obtained to be used in academic studies. 

The use of metrics (CodeMR static code analysis tool) to measure the impact of technologies, methods and principles used by the Mooncascade Android team on maintainability was realised as described following. First of all, a pilot project where metrics can be applied has been determined. While determining the pilot project, the priority was to find a project with an old version developed using outdated technologies, methods, and a weakly structured architecture. However, this project should also have had a new version developed with current technologies and methods mentioned in this study. Thus, using the metrics mentioned above, the impact of the team's current methods on maintainability could be measured. Because of Mooncascade's wide range of realised and ongoing projects, finding a project that met these conditions was possible. Although the full content cannot be disclosed due to the privacy principles, it was possible to apply quantitative evaluation by applying selected metrics to the old and new codebases of an Android application developed very recently and actively in use. In this way, it was aimed to evaluate the effects of the methods, principles and technologies used by the Mooncascade Android team on maintainability at the maximum level effectively. Detailed information about the results and evaluation of the results will be shared in section \ref{section:5}.

\subsection{Summary}
Results of the literature review showed that most studies approach the maintainability of Android applications from a software architecture/design pattern perspective only. Other techniques and technologies can also be used to increase the maintainability of Android applications besides architectural patterns. Still, there are very few studies that focused on these techniques and technologies. The fact that the studies do not focus on these techniques and technologies can be considered a limitation of the academic literature regarding this topic. Apart from this, it is seen that similar metrics and methods are used in many studies in this field. Considering the differences of Android applications compared to other software, it can be mentioned that different methods should be used. In addition, it was not possible to come across studies involving relatively new Android technologies. In some studies, the difficulties of evaluating Android applications developed with a relatively new programming language such as Kotlin in terms of maintainability are also addressed. In general, results indicate the need for a new maintainability model for Android applications.

\newpage
\section{Mooncascade Case Study}
\label{section:4}
\subsection{Case Company}

\subsection{Principles}
\subsubsection{SOLID Principles}
\label{section:4.2.1}
\subsubsection{Clean code}
\label{section:4.2.2}

\subsection{Programming Language}
\label{section:4.3}

\subsection{Architecture}
\subsubsection{MVVM}
\label{section:4.4.1}
\subsubsection{Clean Architecture}
\label{section:4.4.2}

\subsection{Libraries}


\subsubsection{Dependency Injection}
\label{section:4.5.1}
\subsubsection{Networking}
\label{section:4.5.2}
\subsubsection{Asynchronous Events}
\label{section:4.5.3}
\subsubsection{Android Architecture Components}
\label{section:4.5.4}
\subsubsection{Dependency Management}
\label{section:4.5.5}

\subsection{Summary}

\newpage
\section{Evaluation}
\label{section:5}
In this section, the results obtained from applying quantitative and qualitative assessment methods, whose details were shared in chapter \ref{section:4}, will be presented. As explained in chapter \ref{section:4} before, the Android developer survey findings and the results obtained from the interviews made with the members of Mooncascade's Android team, which were applied within the qualitative evaluation scope, will be shared in this section. Results obtained from the evaluation with object-oriented metrics, which were detailed in the third section, will also be presented in this section. Thus, through the results obtained from the qualitative and quantitative evaluations, the second research question will also be answered in this section.

\subsection{Android Developer Survey}
\label{section:5.1}
First of all, the information obtained while answering the first research question guided the whole of this study. Studies conducted to answer this research question have shown that qualitative measurements are necessary as well as quantitative measurements. It has been seen that qualitative measurements can make important contributions to the results of the evaluation with the information obtained directly from the developers and support the quantitative measurement results. Therefore, qualitative assessment methods were determined and decided to be applied in the scope of this study. The fact that experienced Android developers make evaluations about the methods and technologies they use every day from the maintainability point of view and the results obtained from these evaluations can be added to this study increases the study's accuracy. From this point of view, it would not be wrong to say that the addition of qualitative methods as well as quantitative methods to studies focusing on the measurement of software development concepts such as maintainability would increase the qualification of the study.

The research conducted to answer the first research question has also shown that the maintainability of object-oriented software systems can be evaluated quantitatively by using many different metrics. In this study, an evaluation method based on the concepts of complexity, coupling and cohesion was chosen in order to measure the maintainability of Android applications and 5 metrics that can measure these concepts were determined. However, it cannot be said that these methods and metrics used in this study to measure the maintainability of software systems are the best solutions that can be used for this purpose. It was explained in the previous sections why these methods and metrics are chosen. Although these methods and metrics are sufficient for this study, it would not be wrong to say that there may be more effective quantitative measurement methods. Several different studies cover different maintainability evaluation methods for Android applications \cite{43,34}. However. the best quantitative measurement method for maintainability evaluation is controversial, and it would be appropriate to conduct comparative studies using different metrics and methods to determine the most effective solution. Nevertheless, the used methods are considered sufficient for this study, and the existence of more effective methods is beyond the scope of this research.

\subsection{Interviews with Team Members}
%Considering the date this application started to be developed, the use of old-fashioned technologies is normal, but the lack of organization throughout the application is unusual. This unusual situation and the other features listed above seriously affect the understandability and maintainability of the application. On the other hand, this situation offers a great opportunity for this study to measure maintainability.

%It would not be wrong to say that this situation negatively affects the efficiency of quantitative evaluation and comparison because the features used when making the evaluation are less complicated than the other application features. Although using relatively less complicated features while making the assessment slightly reduces the results' effectiveness, it does not mean that these results are false or unrealistic. However, this situation is the study's biggest limitation and gives an idea of future study subjects. As a result, the metrics, whose details are shared in section 3.2, have been applied to these four different features of two code bases. 

%and these developers can be considered as mid or senior level Android developers and 40\% of the participants have five years or more experience.
%When the 164 participants of the survey are examined, it can be easily said that the survey level is sufficient in terms of participant diversity. This fact proves that the respondents are Android developers who have sufficient experience and are proficient at Android development. The responses to the rest of the survey questions can be interpreted based on that fact, which is believed to provide more accurate results.

%Considering that Kotlin is a programming language suggested by Google and Android and is more "programmer-friendly" than Java, it is not difficult to understand that the above table is not surprising. As of 2020, Google declared that more than 60\% of Android applications were developed with Kotlin. It can be said that the survey results largely overlap with this statement\footnote{\url{https://developer.android.com/kotlin}}. On the other hand, he fact that some users still use Java can be explained by the existence of Android applications developed with Java before Kotlin was declared as an official programming language for Android. As stated in section \ref{section:4.3}, Mooncascade's Android team develops Android applications using Kotlin programming language, unless otherwise requested by its customers. When the survey results presented in detail above and the company's choice are compared, it is seen that this choice coincides with the Android community’s current trends.

%Android Architecture Components framework provides some out of box solutions for MVVM. We see that Android developers highly adopt it as of the first quarter of 2021. In other words,  it can be said that as the developer experience increases, the tendency of the developers to choose more than one design pattern also increases. In this case, it can be said that experienced Android developers make the presentational design pattern selection by considering which design pattern will fit the project size and content, rather than what is more popular. As another proof of this situation, it can be shown that developers with 0-3 years of experience have answered this question by selecting the MVVM option. In other words, it is possible to talk about the tendency of Android developers at the beginning of their career to choose popular or "hype" technologies. proving that the knowledge of architecture and design pattern in software development correlates with experience. Lastly, concerning this question, it will be helpful to mention the participants’ tendency to choose design patterns such as MVC, MVP and MVI. Comparing the survey results with Figure 7, which is presented in section \ref{section:2.7} and cited from a study conducted a few years ago in Android architectures, we are faced with similar results despite minor differences. When we look at the comparison results, it is seen that MVVM and MVP were popular among the Android community a few years ago, but MVVM is more preferred today. As mentioned before, it can be said that since the MVVM design pattern started to be provided as an out of box solution by the Google Android team three years ago, this situation increased usage of the MVVM design pattern. In the survey, we also see that 18 of the participants declared that they used the MVC design pattern. Although the MVC design pattern is considered an outdated design pattern in the Android community, the existence of projects developed using this pattern, and considering the suitability of this pattern for small projects; it is understandable why the pattern is still in use. This fact is not surprising, given the MVI design pattern’s growing popularity during 2020 and 2021. It can be said that this population will increase even more in the upcoming period. As stated in section \ref{section:4.4.1}, Mooncascade's Android team prefers the MVVM presentational design pattern when developing Android applications. When the survey results (presented in detail above) and the company's choice are compared, it is seen that this choice coincides with the Android community’s current trends.

%Clean Architecture's details, pros and cons were previously shared, but it is widely used among developers, as seen from the survey results. As can be seen in Fig. \ref{fig:arch_patterns}, which is cited from a study on Android architecture carried out a few years ago. Considering the advantages of Clean Architecture, especially when developing large and complex Android applications, and the growing and complexity of Android applications, developers' choice of Clean Architecture makes much sense. Finally, it is possible to say that the Clean Architecture choice of the Mooncascade Android team coincides with the Android developer trends.

%The importance of the SOLID principles and their use requirements were discussed in detail in section \ref{section:SOLID}. Considering how important it is to comply with SOLID principles in software development processes, it can be said that this rate is below expected. It is not easy to understand why people who develop software professionally in the Android field or any other field do not want to follow SOLID principles or are not aware of these principles, especially if these people are experienced developers. As stated in chapter 4 before, Mooncascade's Android team actively applies SOLID principles in Android application development processes. This selection is compatible with general Android developer behaviour, as can be seen in the results above.


%The purpose, advantages and disadvantages of these principles are given in section 4 in detail. Although there are many advantages of Clean Code principles, discussions are still going on in Android and other software development communities about Uncle Bob and his principles. From this point of view, it can be understood that although most of them actively use these principles, some developers do not. This situation can be interpreted as applying advanced concepts such as Clean Code or SOLID while developing the software directly proportional to the experience. Mooncascade's Android team mainly applies Clean Code principles in Android application development processes. This selection is compatible with general Android developer behaviour when compared to the results above. Further information about how Mooncascade's Android team applies Clean Code principles can be found in section \ref{section:4.4.2}.

%Although the network library's use does not directly affect maintainability, this question was included in the questionnaire. It was also among the aims of this study to identify developer tendencies. Also, the use of some advanced networking libraries indirectly affects maintainability due to the out of box solutions they offer. For this reason, it was deemed appropriate to add this question to the survey. It would not be wrong to say that this library is mainly preferred due to its ease when integrating back-end systems running on REST architecture into Android applications. More detailed information and comments about these libraries can be found in \ref{section:4.5.2}. Mooncascade's Android team prefers Retrofit or Apollo libraries depending on the back-end system’s type to be used in the project. This preference is in line with the survey results. APOLLO ALTERNATIFI YOK

%This question was included in the survey, considering that many Android applications are based on asynchronous events and the impact of the tools used in managing these events on the application architecture and thus on maintainability.The use of more than one solution can be explained by applications that need to be maintained or preferring a solution based on the project needs. This situation can be explained by maintaining some previously coded applications using the AsyncTask and are still in use. The use of this solution is no longer recommended \footnote{\url{https://developer.android.com/reference/android/os/AsyncTask}}. Usage of Kotlin coroutines is increasing among Android developers, as it is easier to learn and use than the RxJava library and because it requires no external dependency. Although RxJava has a steep learning curve and faces the growing popularity of the Kotlin coroutines, it is still preferred by many Android developers for the advanced features it offers. However, there has been a severe increase of applications that have recently migrated their RxJava solutions to Kotlin coroutines\cite{42}. Although the Mooncascade Android prefers RxJava for now, it has been continuing its efforts to switch to Kotlin Coroutines solution. Details on how RxJava is by used are shared in section \ref{section:4.5.3}

%, which significantly impact software maintainability and software architecture when developing Android applications.Dagger 2 and Hilt are DI frameworks recommended by the Android team. However, it is predicted that Hilt's use will surpass Dagger 2 soon, primarily due to the ease of learning it brings and the decrease in boilerplate code  \footnote{\url{https://developer.android.com/training/dependency-injection/hilt-android}}. It can be said that the Koin is preferred among Android developers because of its ease of learning and its ability to get integrated into Android applications with much less boilerplate code when compared to Dagger 2. Also, it is essential to mention that Koin was developed by using Kotlin programming language. This situation is not surprising given that all of the participants, who were not aware of the concept of DI, had less than a year of experience. Because DI is an advanced software development concept, and its practical implementation is a technique that requires solid experience. It is not mandatory to use any DI framework when developing Android applications. Therefore, it can be mentioned that 17.5\% of the participants stated that they do not use any framework and apply their custom solutions. Mooncascade's Android team applies DI principles in their projects and makes these applications through the Dagger 2 framework. Details on how this framework is used are shared in section \ref{section:4.5.1}. The Team is also considering migrating to Hilt soon.

%The high rate of usage is understandable, considering the out of box solutions it offers in solving some of the difficulties encountered while developing Android applications (which were mentioned in the first section, e.g. activity/fragment life-cycle) and the other facilities it provides for Android developers. In addition to this situation, there are groups in the Android community that are distant from this framework because it causes some other difficulties while solving the previously mentioned problems. This claim is controversial, and its details are beyond the focus of this study. However, this may be the reason why some participants do not prefer using this framework.
%Mooncascade's Android team prefers to use the Android Architecture Components framework. Details on how this framework is used are shared in section \ref{section:4.5.4}.

%In this section, the interpretation of the metric values obtained from the analysis of cb-2 is presented.

%When the cb-1 values of the WMC metric used in complexity measurement are examined, it is seen that almost all entities of this codebase have low WMC levels. These results are at the expected level, as it is known that this codebase applies SOLID principles and some of the Clean Code principles well.
%The fact that most of the classes belonging to this codebase have low complexity character increases the maintainability of the codebase.

%When the DIT results are examined, it is seen that there are similar results to the cb-1 results. The base class structure in cb-1 is also found in cb-2. Many children classes inherit these base classes. Therefore, DIT values for some classes are middle-high level. This also indicates that inheritance is widely used throughout the project. Besides, when the content of the traits in inheritance was examined, it was observed that these features were functions with low complexity that frequently repeat between classes.From this point of view, it would not be wrong to say that there is no excessive use of inheritance for this codebase. The current inheritance practice is in a way that will increase reusability and enable maintainability.

%The NOC metric values of cb-2 are also similar to cb-1, and the reasons are similar to the reasons explained in the previous paragraph. It is seen that these values are complying with the DIT. In other words, the application of inheritance is not excessive and does not increase complexity. On the contrary, the use of this way of inheritance effectively utilizes reusability and thus increases maintainability. Besides, since the application is evaluated on only 4 features, the NOC values remain relatively low. These values can be higher, especially for the base classes, if a fully functional application is evaluated because more entities will inherit the base classes. However, this situation will not pose a problem in terms of complexity unless the DIT values are high; that is, the depth of inheritance is not high. The same is true for cb-1 as well.

%When the COB metric values of the classes are examined, it is seen that the application is in a good situation in terms of coupling. Since abstraction, dependency inversion and dependency injection principles are applied very tightly for this codebase, it is expected that the application will be smooth or with minimal problems in terms of coupling. Since low coupling provides great advantages in terms of reusability and ease of modification, it makes a great contribution to the maintainability of this codebase.

%The results for the LCOM metric of cb-2 display a rather interesting insight.
%As explained earlier, LCOM measures the relationship between methods of a class. The low relationship between the methods of the classes indicates that a class has multiple responsibilities, which reduces the understandability and ease of modification of the class. Therefore, low cohesion means low maintainability. Considering that the principles of SOLID and SoC are strictly applied while developing the cb-2, it should be considered normal that the classes are concise and have a single responsibility. Therefore the cohesion values are also high. 

%When the class-based metric values collected from assessing CB-2 are examined, it is shown that there are visible improvements in complexity, coupling and cohesion areas even for simple features such as splash, login, register. When Fig. \ref{fig:cb-2-donuts} and Fig. \ref{fig:cb-2-package} are examined, the second point that draws attention is that the dimensions are significantly reduced despite the increase in the number of classes and packages. Concise classes and the increase in the number of packages provides a more organized code base and a better-fragmented understanding of responsibility. This situation is also noticeable when the codebase is examined with the help of an IDE. The organization and understandability of cb-2 are at a high level. Besides, the levels of the classes in complexity, coupling are quite low, and cohesion is high. This situation can be shown as proof that the SOLID and SoC principles are applied correctly, and it can be said that there is a very positive effect on maintainability. On the other hand, when the project's bigger picture is examined (See Fig. 26), a few classes with moderate complexity and cohesion issues stand out. When these classes were investigated, it was seen that they were the classes called "View Model" in the MVVM design pattern. These classes are responsible for how and when the data will be displayed (in other words, display/view logic) in the MVVM design pattern. According to the principles of the MVVM design pattern, each view should have only one view model. In this case, the view models belonging to the views with more than one responsibility also have the logic of belonging to more than one responsibility. Therefore, these classes become more complex, and the cohesion of the classes decreases due to the methods and dependencies they have for different responsibilities. As long as the principles of the MVVM design pattern are followed, it would not be appropriate to divide these responsibilities between different classes. Nevertheless, the effects of the technology and principles used in the development of cb-2 on maintainability are obvious. It has been observed as a result of evaluating the cb-2 that these principles and technologies make a big difference even in the development of relatively simple features. While the results are not perfect, they are important as they offer a starting point for improving maintainability.

%Considering the general situation, it is seen that the participants stated that the current methods and technologies used have positive effects on maintainability, but they also touched on some negative points. 

%In this section, the interpretation of the metric values obtained from the analysis of cb-1 is given. When the class-based metric values obtained from evaluating cb-1 are examined, there are a few points that attract attention. 

%When the cb-1 values of the WMC metric used in complexity measurement are examined, it is seen that there are 2 problematic classes and some other classes have minor issues in terms of complexity. The classes A and B, which are mentioned in section \ref{section:4.3.4}, and have WMC values in the medium-high range, can be seen as problematic in terms of complexity. When evaluating the WMC metric values of the classes, it should be taken into account that classes with high complexity will have low maintainability characteristics.

%If the cb-1 results of the DIT metric related to complexity and inheritance are examined, generally minor problems are observed. The DIT values of some classes seem to be at the middle-high level. There are a few inherited base classes (e.g. BaseActivity, BaseFragment), and these classes contain frequently used simple functionality. These classes are the ones that usually have higher DIT values. It can be said that this situation does not pose a serious problem when the application is considered. Fig. \ref{fig:cb-1-donuts} shows how the children classes inheriting these base classes reflect on the DIT results. Even though inheritance has a positive effect on the reusability level of the software systems, excessive and deep use of inheritance is considered a threat to maintainability. However, in this case, the usage of inheritance does not seem deep and excessive.

%When looking at the NOC results, the classes seem to have low values. From this point of view, it can be said that the application of inheritance in the codebase is low. Although this situation can be interpreted as the coupling between the classes is low, when the evaluation results of the coupling metrics are examined, it is seen that this is not the case. However, low values of this metric also indicate low reusability of a software system. On the other hand, as stated in the previous paragraph, it is seen that some base classes and their inheritance in the codebase are reflected in NOC results as well. This situation can be seen when examining Fig. \ref{fig:cb-1-donuts}. Classes with low-medium NOC values correspond to these base classes.

%When the COB metric values of the classes are examined, it is seen that the application has some problems in terms of coupling. It can be said that classes with high CBO values tend to have low reusability and low maintainability character. When these two classes with high and very high COB values are examined, it is seen that these classes are A and B classes mentioned in the paragraph where the WMC metric results were interpreted. It is worth noting that these two classes are problematic classes for cb-1. Since the SOLID principles and dependency injection applications were not handled properly while developing the cb-1, it would not be wrong to say that these results are not surprising for the CBO values.

%When the results regarding the LCOM metric used in the measurement of cohesion level are examined, it is seen that the codebase has serious problems with the cohesion. This situation can be easily noticed when Fig. \ref{fig:cb-1-donuts} is examined. When Fig. \ref{fig:cb-1-package} is examined, the size of the classes and packages draw attention. This situation can be explained by the separation of responsibilities and the correct application of the "Single Responsibility" principle, the first of the SOLID principles. Classes that do not implement this principle correctly have different functionalities, independent of each other, which reduces cohesion.Classes with low cohesion are characterized as being closed to modification, and it is known that such classes tend to carry more than one responsibility in general. Classes of this character have low understandability, and this negatively affects maintainability. From this point of view, it would not be wrong to say that there are maintainability problems in a serious part of the application.

%Considering the metric results in general, it is seen that although the evaluation is made over features with relatively low complexity (splash, login, register, etc.), problems from CB-1 stand out. Apart from the comments based on the above metric values, when Fig. \ref{fig:cb-1-package} is examined, the first thing that catches the eye is a complex and unorganized packaging structure. Layer and feature-based packaging methods are internal in the project, which causes serious maintainability, understandability and organization problems. It is also noteworthy that some classes are large in size, which can be interpreted as lack of proper separation of concerns. The also codebase appears to have complexity, coupling and cohesion problems even with these relatively low complication features. In a few classes, these problems are at a very high level, and maintainability and organization problems at different levels in the project draw attention. Especially the coupling problem stands out for this code base. Considering that there are no healthy abstraction and dependency injection applications in the project, this result is not very surprising. Apart from this, the problematic classes that draw attention in the paragraphs where metrics are interpreted appear in figure 23 as well with their abnormal sizes and their issues in complexity, coupling and cohesion. Large circles of red, orange and yellow represent these problematic classes. When all these analysis results are taken into account, it is clear that the cb-1 code base has complexity, coupling and cohesion problems even for simple features, and therefore shows a low maintainability character.

%In this section, the interpretation of the results obtained from the feature-based metric comparison over the login feature of cb-1 and cb-2 are shared.

%First of all, it is seen that the cb-1 codebase uses 6 classes and interfaces for the visual part of the login feature, while cb-2 uses only 2 classes for this feature. From this point of view, the cb-2 code is better in terms of organization and understandability. On the other hand, when the metric values given in Fig. \ref{fig:login-metric-table} are examined, it is seen that the results of cb-2 are better than the results of cb-1.

%When the metric values shared in Fig. \ref{fig:login-metric-table-2} are examined, the difference between the view responsibilities of the two projects on complexity draws attention. When the values of WMC, DIT and NOC metrics used in complexity measurement are compared, it is seen that the complexity level of cb-2 for the view responsibility is lower. On the other hand, when the results of the complexity metric values of view logic related responsibilities are examined, it is observed that there is not much difference between the two code bases. As explained in previous sections, these results should be considered normal given the complexity of functionality of the classes involved in this responsibility.

%When the CBO metric related to measuring the coupling level is examined, it is seen that the cb-2 codebase gives better results. Since the SOLID and DI principles are used much more effectively in the cb-2 codebase, these results are not different from what is expected. As mentioned before, cb-1 uses the MVP design pattern. Coupling is increasing due to the bi-directional dependency between view and presentation layers in the MVP design pattern. It would not be wrong to say that this situation increases the coupling level of the cb-1. However, this situation is not the case for cb-2, which uses the MVVM design pattern. In the MVVM design pattern, only the view layer has a dependency on the presentation layer.

%When the results of the LCOM metric related to cohesion are examined, it is seen that the results are similar to the results of the complexity metrics. While the results of cb-2 are much better than cb-1 for the responsibility of view, there is not much difference between the results for the responsibility of presentation. As explained in the previous section, there is only one view model per view principle in the MVVM design pattern. Therefore, one view model might have different responsibilities, especially those that belong to the complex views. Thus this circumstance can be shown as a reason for this situation. The same is true for the MVP design pattern as well. There can only be one presenter for each view. Naturally, cb-1, which uses the MVP design pattern, seems to have low cohesion values for presentation responsibility.

%As a result, when looking at the general situation of the feature-based metric comparison, there are big improvements in the complexity and cohesion areas for the classes related to the view responsibility of cb-2 compared to the cb-1. At the same time, there is not much difference in terms of view logic/presentation responsibility. On the other hand, in the coupling, it has been determined that the cb-2 codebase is better than the cb-1, and the reasons for this situation have been mentioned in the previous sections.

\subsection{Evaluation with Object-Oriented Metrics}
In this part of the study, how the quantitative measurement methods are applied and these measurements' results are discussed. Besides, the findings obtained from the comparison of the results are also shared in this section.

\subsubsection{Sample Codebases}
\label{section:5.3.1}
In this section, detailed information about the codebases which the metrics mention in section \ref{section:4.2} are shared. As mentioned in section \ref{section:4.2}, two codebases belonging to the same project have been selected to apply the metrics. To facilitate the explanation, these codebases will be mentioned in the form of \textbf{\textit{CB-1}} and \textbf{\textit{CB-2}} in the remainder of the study.

When CB-1 is examined in detail, the following situation is encountered. First of all, it is the Android application's actively used version. The project was started to be developed using the Java programming language. Later, some features were developed using Kotlin programming language. There is no consistent choice of software architecture throughout the application. Although the application consists of 5 different modules, the modules' boundaries are not determined according to a certain standard. Some modules are feature-based, while some are layer-based. A similar situation is observed in packaging. The packaging organization of the application is inferior. While some features of the application have been developed with the MVP design pattern, some features have been developed with MVVM. It is controversial to what extent these design patterns are applied correctly. Static classes and singleton solutions have been used dangerously in practice. Besides, the SOLID principles have been ignored and no dependency injection is applied. It is also worth noting the use of some outdated libraries. Also, when looking at the Git history of the application, the commits of 11 different developers are seen. This situation is critical in describing the developer circulation in the application and explaining its serious organisation problems. The relationship between the developer circulation and the maintainability of software systems was previously mentioned in section \ref{section:1.1}.

CB-2 was developed as part of this study to evaluate the impact of the methods and technologies used by the case company (see \ref{section:2.3}) by comparing it to CB-1, a codebase developed without these methods and technologies. CB-2 is developed using Kotlin programming language. Clean architecture and clean code, and SOLID principles are followed during the development. The application modules are separated based on the layers of different responsibilities (view, presentation, domain, data, local storage, networking), and the packaging is feature-based. While developing the application, maintainable and reliable Android libraries have been used. The organisation level of this codebase is very high and consistent, and it has been arranged to be a standard throughout the whole application. On the other hand,  development for this version of the application's is still ongoing, and there are only 4 main features that have already been developed. The features currently developed for CB-2 are splash, login, register and main screen features. In this respect, CB-2 falls short in terms of developed features compared to CB-1. Therefore, the evaluation was only be made over the features that have been developed in both CB-1 and CB-2. The other features of CB-1 were removed from the codebase before the evaluation. Removal of such features was relatively easy since the developed features are quite independent of the rest of the application.




\subsubsection{CodeMR}
It was mentioned that it is important to know the applications' situation and the technology and principles used during the development of the applications to understand the quantitative measurement results better. In addition to this, knowing which features of the applications will be compared is also important to interpret the results better and comment on the evaluation's effectiveness. As stated in the previous paragraph, the development of only four features of the code-base-2 has been completed. Therefore, while quantitative measurements were being made, the evaluation was only be made over the features that have been developed in the code-base-2. In other words, object-oriented quality and maintainability metrics were applied to splash, login, register and homepage features of both code-base-1 and code-base-2. To have a more efficient comparison of the results, applying the metrics to only the intersection of both code bases' fully functional features was essential. So, the other fully functional features of code-base-1 were ignored when making the quantitative evaluation and then the comparison. Only these previously mentioned four features were included in the evaluation of code-base-1, and the rest of the features were removed. These removed features had no impact on the results of evaluation of code-base-1. It would not be wrong to say that this situation negatively affects the efficiency of quantitative evaluation and comparison because the features mentioned above and used when making the evaluation are less complicated than the rest of the application. Although the use of relatively less complicated features while making the assessment slightly reduces the results' effectiveness, it does not mean that these results are false or unrealistic. However, this situation draws attention to the study's biggest limitation and gives an idea of what future study subjects may be. As a result, the metrics, whose details are shared in section 3.2, have been applied to these four different code bases' features. The quantitative evaluation and comparison in this study were made as was explained above.

\subsubsection{Way of Evaluation and Presentation of Results}
It was mentioned that it is important to know the applications' situation and the technology and principles used to understand the quantitative measurement results better. Knowing which features of the applications will be compared is also important to interpret the results better and comment on the evaluation's effectiveness. As stated in the previous section, the development of only four features of the cb-2 has been completed. Therefore, while quantitative measurements were being made, the evaluation was only be made over the features that have been developed in both cb-1 and cb-2, namely, splash, login, register and homepage features. So, the other features of cb-1 were ignored when making the quantitative evaluation and then the comparison. It would not be wrong to say that this situation negatively affects the efficiency of quantitative evaluation and comparison because the features used when making the evaluation are less complicated than the other application features. Although using relatively less complicated features while making the assessment slightly reduces the results' effectiveness, it does not mean that these results are false or unrealistic. However, this situation is the study's biggest limitation and gives an idea of future study subjects. As a result, the metrics, whose details are shared in section 3.2, have been applied to these four different features of two code bases. 

Presentation of the results is done as explained below. The charts and the numeric metrics values obtained from the evaluation have been shared for the project level. Although these techniques can be applied at the class and package level, it was preferred to present the project level results. It would not be very effective and useful to present these results for each class and package since there are many classes and packages for both codebases. On the other hand, at the class level, for making some evaluation and comparison, a few sample classes with high functionality were selected from both codebases, and they were evaluated and compared. Also, in accordance with the confidentiality requirements, the package and class names that will call the application name were hidden in the shared analysis results and figures. Then the results were presented via visualisation methods and numeric values. In the following sections, the results of this evaluation and the findings for comparisons are shared.

\subsubsection{CB-1 Results}
When the numerical analysis results produced on the CB-2 via the CodeMR tool are reviewed, it is seen that the tool analyzes 1160 lines of code belonging to this codebase. These lines of code belong to 139 classes in 59 different packages of 4 different features which were mentioned in section \ref{section:4.3.1}. In Fig. \ref{fig:CB-2-metric-table.png}, a CodeMR table that reflects the general situation of CB-2 is shared.
\begin{figure}[ht!]
    \centering
    \includegraphics[scale=0.45]{figures/CB-2-metric-table.png}
    \caption{CodeMR Metrics Overview of CB-2}
    \label{fig:CB-2-metric-table.png}
\end{figure}
\FloatBarrier

Between the analyzed classes, it is seen that two classes corresponding to 9\% of the total code size have low-medium WMC values, and the rest of the classes appear to have a low level of WMC values. It appears that the majority of the classes belong to CB-2 have low complexity levels. When the values of the DIT metric of the classes are examined, it is seen that 10 classes, corresponding to 28.9\% of the total code size, have middle-upper level DIT values and the other 27 classes corresponding to 14.1\% of the total code size have low-medium level DIT values. The rest of the classes have a low level of DIT values.  When DIT values of these classes are examined, it is seen that some classes have a DIT value of 2 and classes with more than a DIT value of 2 values are quite low (DIT values between 1-3 considered as low-medium by CodeMR). Classes with middle-upper-level DIT values are all view models or Activity/Fragment classes with a base class containing the common basic functionality. When the NOC metric values of the classes are examined, it is seen that 7 classes corresponding to only 6.6\% of the whole codebase have low-medium NOC values, and 1 very concise class has medium-high NOC value. The rest of the classes have low NOC values. According to the results, it is seen that ten classes are corresponding to 26.6\% of the application's code volume with a low-medium COB value and the rest of the classes have a low COB value. Results also showed that, all classes within the CB-2 have low LCOM values. Fig. \ref{fig:CB-2-donuts} presents the results of the class-based metric values visualized by the "Metric Distribution" method offered by the CodeMR.

\begin{figure}[ht!]
    \centering
    \includegraphics[scale=1]{figures/CB-2-donuts.png}
    \caption{CodeMR Metric Distribution for CB-2}
    \label{fig:CB-2-donuts}
\end{figure}
\FloatBarrier

\newpage
In Fig. \ref{fig:CB-2-package}, a general view of CB-2 in terms of complexity, coupling and cohesion is shown. As explained in the previous section, the largest circle represents the project, inner circles represent the packages and small circles inside the inner circles represent the classes. Sizes of the circles proportional to the size of the represented entity.
\begin{figure}[ht!]
    \centering
    \includegraphics[scale=1.1]{figures/CB-2-package.png}
    \caption{CodeMR Metric Distribution for CB-2}
    \label{fig:CB-2-package}
\end{figure}
\FloatBarrier







\subsubsection{CB-2 Results}
The login feature has been selected among the available features to compare the maintainability differences between the two codebases. The reason for choosing it is that it has a more complex functionality/logic (e.g. validation, different error types, etc.) than other existing features mentioned in the previous sections. Classes containing view and view logic layers for the login feature from both codebases have been selected for comparison. The reason for selecting such layers is that these layers were considered the most involved parts of the compared feature of the two codebases in terms of software complexity. While making the evaluation, other lower-level dependencies were excluded. When the CB-1 (uses the MVP design pattern) is examined, it is seen that 6 classes and interfaces are used related to the presentation and logic of data related to login. These classes and interfaces are: LoginActivityView (interface), LoginActivity (class), LoginActivityPresenter (class), LoginFragmentView (interface), LoginFragment (class), LoginFragmentPresenter (class). When the CB-2 (uses the MVVM design pattern) is examined, it is seen that only 2 classes are used and they are LoginActivity and LoginViewModel. The difference in the number of classes and interfaces used for the login feature in the codebases is due to the differences in the design patterns used and the fact that the CB-1 also uses the Fragment class of Android. Table. \ref{fig:login-metric-table} presents the metric values obtained for each class from the quantitative evaluation performed using the Android Studio CodeMR plugin.
\begin{table}[htb]
    \centering
    \includegraphics[scale=0.5]{figures/login-metric-table.png}
    \caption{CodeMR Metric Values for Login Feature} 
    \label{fig:login-metric-table}
\end{table}

In order to examine the results better, groupings can be made between these classes, and the metric values of these groups can be compared. This grouping can be made based on the responsibilities of the classes. The responsibilities to be taken as a basis while determining groups are the view and view presentation. More detailed information on these responsibilities was shared in previous sections. These classes and their interfaces can be grouped for each codebase as follows. For CB-1, LoginActivityView, LoginActivity, LoginFragmentView and LoginFragment are only responsible for the view responsibility.  LoginActivityPresenter and LoginFragmentPresenter are the classes responsible for how and when the data is presented. For CB-2, LoginActivity is responsible for the view responsibility, and LoginViewModel is responsible for how and when data is presented. This grouping will be useful for comparing the metric values shared in Table \ref{fig:login-metric-table}. Table \ref{fig:login-metric-table-2} presents the metric values for each group of both codebases that are obtained from the analysis.

\begin{table}[htb]
    \centering
    \includegraphics[scale=0.5]{figures/login-metric-table-2.png}
    \caption{CodeMR Metric Values for Login Feature}
    \label{fig:login-metric-table-2}
\end{table}

Results showed that the CB-1 codebase uses 6 entities for the selected layers of the login feature, while CB-2 uses only 2. From this point of view, the CB-2 has better organization and understandability. Moreover, when the values of WMC, DIT and NOC metrics used in complexity measurement are compared, it is seen that the complexity level of CB-2 for the view responsibility is lower. On the other hand, when the results of the complexity metric values of presentation related responsibilities are examined, it is observed that there is not much difference between the two code bases. These results should be considered normal given the complexity of functionality of the classes involved in this responsibility. When the CBO metric related to measuring the coupling level is examined, it is seen that the CB-2 codebase gives better results. Since the SOLID and DI principles are used much more effectively in the CB-2 codebase, these results are expected. Also, CB-1 uses the MVP design pattern. The coupling of the CB-1 is increasing due to the bi-directional dependency between view and presentation layers in the MVP design pattern. However, this situation is not the case for CB-2, which uses the MVVM design pattern. In the MVVM design pattern, only the view layer has a dependency on the presentation layer. When the results of the LCOM metric related to cohesion are examined, it is seen that the results are similar to the results of the complexity metrics. While the results of CB-2 are much better than CB-1 for the responsibility of view, there is not much difference between the results for the responsibility of presentation. There is only one view model per view principle in the MVVM design pattern. Therefore, one view model might have different responsibilities, especially those that belong to the complex views. The same is true for the MVP design pattern as well. There can be only one presenter for each view. Naturally, CB-1, which uses the MVP design pattern, seems to have low cohesion values for presentation responsibility. 

\subsubsection{Feature-Based Comparison Results}
In this section, results obtained from comparing the analysis of cb-1 and cb-2 codebases are shared. The analysis results of the two codebases have been compared to make a more efficient evaluation. Besides, to make a more effective evaluation, the numerical evaluation results obtained from a few sample classes with high functionality were collected, and the results were compared. To get more accurate results, while selecting the sample classes, classes with high complexity and more detailed business logic were studied. Evaluations of all these comparisons are shared in this section.

The login feature has been selected from among the available features to compare the maintainability differences between the two codebases. The reason for choosing it is that it has a more complex logic (e.g. validation, different error types, etc.) than other existing features mentioned in the previous sections. Classes containing view and view logic parts for the login feature from both codebases have been selected for comparison. While making the evaluation, other lower-level dependencies were excluded. When the cb-1 using the MVP design pattern is examined, it is seen that 6 classes and interfaces are used related to the presentation and logic of data related to login. These classes and interfaces are: LoginActivityView (interface), LoginActivity (class), LoginActivityPresenter (class), LoginFragmentView (interface), LoginFragment (class), LoginFragmentPresenter (class). When the cb-2, which uses the MVVM design pattern, is examined, it is seen that only 2 classes are used and they are LoginActivity and LoginViewModel. The difference in the number of classes and interfaces used for the login feature in the codebases is due to the differences in the design patterns used and the fact that the cb-1 also uses the Fragment class of Android. In Fig. \ref{fig:login-metric-table}, the metric values of each class listed above, obtained from the quantitative evaluation results, can be seen in the form of a table. The metric values shared in the tables were obtained as a result of the analysis performed using the Android Studio CodeMR plugin.
\begin{figure}[ht!]
    \centering
    \includegraphics[scale=0.65]{figures/login-metric-table.png}
    \caption{CodeMR Metric Values for Login Feature}
    \label{fig:login-metric-table}
\end{figure}
\FloatBarrier

In order to examine the results better, groupings can be made between these classes, and the metric values of these groups can be compared. This grouping can be made based on the responsibilities of the classes. The responsibilities to be taken as a basis while groups are determined as view and view logic. More detailed information on these responsibilities was shared in previous sections. These classes and their interfaces can be grouped for each codebase as follows. For cb-1, LoginActivityView, LoginActivity, LoginFragmentView and LoginFragment are only responsible for the presentation of the data.  LoginActivityPresenter and LoginFragmentPresenter are the classes responsible for how and when the data is presented. For cb-2, LoginActivity is responsible for presenting data, and LoginViewModel is responsible for how and when data is presented. This grouping will be useful for comparing the metric values shared in Fig. \ref{fig:code-mr-metric-val}. In the table below, the metric values for each group of both codebases that are obtained from the analysis made within the scope of this study are shared.

\begin{figure}[ht!]
    \centering
    \includegraphics[scale=0.65]{figures/login-metric-table-2.png}
    \caption{CodeMR Metric Values for Login Feature}
    \label{fig:login-metric-table-2}
\end{figure}
\FloatBarrier

Although the interpretation of the results is partially possible when the data in the above-shared tables are examined, the interpretation and detailed analysis of this given are made and shared in section \ref{section:6.3.5}.


\newpage
\section{Discussion}
\label{section:6}
\subsection{RQ1 Discussion}
First of all, the information obtained while answering the first research question guided the whole of this study. Research conducted to answer RQ1 has shown that the use of quantitative measurements together with qualitative measures can increase the effectiveness of the study. It was seen that qualitative measurements can make important contributions to the results of the evaluation with the information obtained directly from the developers and support the quantitative measurement results. The fact that experienced Android developers make evaluations about the methods and technologies they use every day from the maintainability point of view and the results obtained from these evaluations can be added to this study increased the study's accuracy. From this point of view, it would not be wrong to say that the addition of qualitative methods as well as quantitative methods to studies focusing on the measurement of software development concepts such as maintainability would increase the qualification of the study. The research conducted to answer the first research question showed that the maintainability of object-oriented software systems can be evaluated quantitatively by using many different metrics. In this study, a different quantitative maintainability model based on the concepts of complexity, coupling and cohesion was formed to measure the maintainability of Android applications. This model was created by inspiring the problems encountered while developing Android applications, mentioned in section \ref{section:1.1}. Subsequently, 5 metrics that are suitable for this evaluation model and can measure these concepts were determined. However, it cannot be said that these methods and metrics used in this study is the best that can be used for this purpose. Although these methods and metrics were sufficient for this study, it would not be wrong to say that there may be more effective quantitative measurement methods. It would be appropriate to conduct comparative studies to determine the most effective solution. 

\subsection{RQ2 Discussion}
\input{chapters/6-discussion/6.2-rq2-discussion/6.2-text}

\subsection{RQ3 Discussion}
\subsubsection{Interpretation of Android Survey Results}
\label{section:6.3.1}
and these developers can be considered as mid or senior level Android developers and 40\% of the participants have five years or more experience.
When the 164 participants of the survey are examined, it can be easily said that the survey level is sufficient in terms of participant diversity. This fact proves that the respondents are Android developers who have sufficient experience and are proficient at Android development. The responses to the rest of the survey questions can be interpreted based on that fact, which is believed to provide more accurate results.

Considering that Kotlin is a programming language suggested by Google and Android and is more "programmer-friendly" than Java, it is not difficult to understand that the above table is not surprising. As of 2020, Google declared that more than 60\% of Android applications were developed with Kotlin. It can be said that the survey results largely overlap with this statement\footnote{\url{https://developer.android.com/kotlin}}. On the other hand, he fact that some users still use Java can be explained by the existence of Android applications developed with Java before Kotlin was declared as an official programming language for Android. As stated in section \ref{section:4.3}, Mooncascade's Android team develops Android applications using Kotlin programming language, unless otherwise requested by its customers. When the survey results presented in detail above and the company's choice are compared, it is seen that this choice coincides with the Android community’s current trends.

Android Architecture Components framework provides some out of box solutions for MVVM. We see that Android developers highly adopt it as of the first quarter of 2021. 
In other words,  it can be said that as the developer experience increases, the tendency of the developers to choose more than one design pattern also increases. In this case, it can be said that experienced Android developers make the presentational design pattern selection by considering which design pattern will fit the project size and content, rather than what is more popular. As another proof of this situation, it can be shown that developers with 0-3 years of experience have answered this question by selecting the MVVM option. In other words, it is possible to talk about the tendency of Android developers at the beginning of their career to choose popular or "hype" technologies. proving that the knowledge of architecture and design pattern in software development correlates with experience. Lastly, concerning this question, it will be helpful to mention the participants’ tendency to choose design patterns such as MVC, MVP and MVI. Comparing the survey results with Figure 7, which is presented in section \ref{section:2.7} and cited from a study conducted a few years ago in Android architectures, we are faced with similar results despite minor differences. When we look at the comparison results, it is seen that MVVM and MVP were popular among the Android community a few years ago, but MVVM is more preferred today. As mentioned before, it can be said that since the MVVM design pattern started to be provided as an out of box solution by the Google Android team three years ago, this situation increased usage of the MVVM design pattern. In the survey, we also see that 18 of the participants declared that they used the MVC design pattern. Although the MVC design pattern is considered an outdated design pattern in the Android community, the existence of projects developed using this pattern, and considering the suitability of this pattern for small projects; it is understandable why the pattern is still in use. This fact is not surprising, given the MVI design pattern’s growing popularity during 2020 and 2021. It can be said that this population will increase even more in the upcoming period. As stated in section \ref{section:4.4.1}, Mooncascade's Android team prefers the MVVM presentational design pattern when developing Android applications. When the survey results (presented in detail above) and the company's choice are compared, it is seen that this choice coincides with the Android community’s current trends.

Clean Architecture's details, pros and cons were previously shared, but it is widely used among developers, as seen from the survey results. As can be seen in Fig. \ref{fig:arch_patterns}, which is cited from a study on Android architecture carried out a few years ago. Considering the advantages of Clean Architecture, especially when developing large and complex Android applications, and the growing and complexity of Android applications, developers' choice of Clean Architecture makes much sense. Finally, it is possible to say that the Clean Architecture choice of the Mooncascade Android team coincides with the Android developer trends.

The importance of the SOLID principles and their use requirements were discussed in detail in section \ref{section:SOLID}. Considering how important it is to comply with SOLID principles in software development processes, it can be said that this rate is below expected. It is not easy to understand why people who develop software professionally in the Android field or any other field do not want to follow SOLID principles or are not aware of these principles, especially if these people are experienced developers. As stated in chapter 4 before, Mooncascade's Android team actively applies SOLID principles in Android application development processes. This selection is compatible with general Android developer behaviour, as can be seen in the results above.


The purpose, advantages and disadvantages of these principles are given in section 4 in detail. 
Although there are many advantages of Clean Code principles, discussions are still going on in Android and other software development communities about Uncle Bob and his principles. From this point of view, it can be understood that although most of them actively use these principles, some developers do not. This situation can be interpreted as applying advanced concepts such as Clean Code or SOLID while developing the software directly proportional to the experience. Mooncascade's Android team mainly applies Clean Code principles in Android application development processes. This selection is compatible with general Android developer behaviour when compared to the results above. Further information about how Mooncascade's Android team applies Clean Code principles can be found in section \ref{section:4.4.2}.

Although the network library's use does not directly affect maintainability, this question was included in the questionnaire. It was also among the aims of this study to identify developer tendencies. Also, the use of some advanced networking libraries indirectly affects maintainability due to the out of box solutions they offer. For this reason, it was deemed appropriate to add this question to the survey. 
It would not be wrong to say that this library is mainly preferred due to its ease when integrating back-end systems running on REST architecture into Android applications. More detailed information and comments about these libraries can be found in \ref{section:4.5.2}. Mooncascade's Android team prefers Retrofit or Apollo libraries depending on the back-end system’s type to be used in the project. This preference is in line with the survey results. APOLLO ALTERNATIFI YOK

This question was included in the survey, considering that many Android applications are based on asynchronous events and the impact of the tools used in managing these events on the application architecture and thus on maintainability.
The use of more than one solution can be explained by applications that need to be maintained or preferring a solution based on the project needs. This situation can be explained by maintaining some previously coded applications using the AsyncTask and are still in use. The use of this solution is no longer recommended \footnote{\url{https://developer.android.com/reference/android/os/AsyncTask}}. Usage of Kotlin coroutines is increasing among Android developers, as it is easier to learn and use than the RxJava library and because it requires no external dependency. Although RxJava has a steep learning curve and faces the growing popularity of the Kotlin coroutines, it is still preferred by many Android developers for the advanced features it offers. However, there has been a severe increase of applications that have recently migrated their RxJava solutions to Kotlin coroutines\cite{42}. Although the Mooncascade Android prefers RxJava for now, it has been continuing its efforts to switch to Kotlin Coroutines solution. Details on how RxJava is by used are shared in section \ref{section:4.5.3}

, which significantly impact software maintainability and software architecture when developing Android applications.
Dagger 2 and Hilt are DI frameworks recommended by the Android team. However, it is predicted that Hilt's use will surpass Dagger 2 soon, primarily due to the ease of learning it brings and the decrease in boilerplate code  \footnote{\url{https://developer.android.com/training/dependency-injection/hilt-android}}. It can be said that the Koin is preferred among Android developers because of its ease of learning and its ability to get integrated into Android applications with much less boilerplate code when compared to Dagger 2. Also, it is essential to mention that Koin was developed by using Kotlin programming language. This situation is not surprising given that all of the participants, who were not aware of the concept of DI, had less than a year of experience. Because DI is an advanced software development concept, and its practical implementation is a technique that requires solid experience. It is not mandatory to use any DI framework when developing Android applications. Therefore, it can be mentioned that 17.5\% of the participants stated that they do not use any framework and apply their custom solutions. Mooncascade's Android team applies DI principles in their projects and makes these applications through the Dagger 2 framework. Details on how this framework is used are shared in section \ref{section:4.5.1}. The Team is also considering migrating to Hilt soon.

The high rate of usage is understandable, considering the out of box solutions it offers in solving some of the difficulties encountered while developing Android applications (which were mentioned in the first section, e.g. activity/fragment life-cycle) and the other facilities it provides for Android developers. In addition to this situation, there are groups in the Android community that are distant from this framework because it causes some other difficulties while solving the previously mentioned problems. This claim is controversial, and its details are beyond the focus of this study. However, this may be the reason why some participants do not prefer using this framework.
 Mooncascade's Android team prefers to use the Android Architecture Components framework. Details on how this framework is used are shared in section \ref{section:4.5.4}.

\subsubsection{Interpretation of Interview Results}
\label{section:6.3.2}
The fact that most of the classes belonging to this codebase have low complexity character increases the maintainability of the codebase.

This indicates that inheritance is widely used throughout the project.
Besides, when the content of the traits in inheritance was examined, it was observed that these features were functions with low complexity that frequently repeat between classes.From this point of view, it would not be wrong to say that there is no excessive use of inheritance for this codebase. The current inheritance practice is in a way that will increase reusability and enable maintainability.

It is seen that these values are complying with the DIT values shared in the previous paragraph. In other words, the application of inheritance is not excessive and does not increase complexity. On the contrary, the use of this way of inheritance effectively utilizes reusability and thus increases maintainability.

When the COB metric values of the classes are examined, it is seen that the application is in a good situation in terms of coupling. 

Since abstraction, dependency inversion and dependency injection principles are applied very tightly for this codebase, it is expected that the application will be smooth or with minimal problems in terms of coupling. Since low coupling provides great advantages in terms of reusability and ease of modification, it makes a great contribution to the maintainability of this codebase.

The results for the LCOM metric of cb-2 display a rather interesting insight.
As explained earlier, LCOM measures the relationship between methods of a class. The low relationship between the methods of the classes indicates that a class has multiple responsibilities, which reduces the understandability and ease of modification of the class. Therefore, low cohesion means low maintainability. Considering that the principles of SOLID and SoC are strictly applied while developing the cb-2, it should be considered normal that the classes are concise and have a single responsibility. Therefore the cohesion values are also high. 

When the class-based metric values collected from assessing CB-2 are examined, there are a few topics to point. First of all, it is determined that there are visible improvements in complexity, coupling and cohesion areas even for simple features such as splash, login, register. 

When Fig. \ref{fig:cb-2-donuts} and Fig. \ref{fig:cb-2-package} are examined, the second point that draws attention is that the dimensions are significantly reduced despite the increase in the number of classes and packages. Concise classes and the increase in the number of packages provides a more organized code base and a better-fragmented understanding of responsibility. This situation is also noticeable when the codebase is examined with the help of an IDE. The organization and understandability of cb-2 are at a high level. Besides, the levels of the classes in complexity, coupling are quite low, and cohesion is high. This situation can be shown as proof that the SOLID and SoC principles are applied correctly, and it can be said that there is a very positive effect on maintainability. On the other hand, when the project's bigger picture is examined (See Fig. 26), a few classes with moderate complexity and cohesion issues stand out. When these classes were investigated, it was seen that they were the classes called "View Model" in the MVVM design pattern. These classes are responsible for how and when the data will be displayed (in other words, display/view logic) in the MVVM design pattern. According to the principles of the MVVM design pattern, each view should have only one view model. In this case, the view models belonging to the views with more than one responsibility also have the logic of belonging to more than one responsibility. Therefore, these classes become more complex, and the cohesion of the classes decreases due to the methods and dependencies they have for different responsibilities. As long as the principles of the MVVM design pattern are followed, it would not be appropriate to divide these responsibilities between different classes. Nevertheless, the effects of the technology and principles used in the development of cb-2 on maintainability are obvious. It has been observed as a result of evaluating the cb-2 that these principles and technologies make a big difference even in the development of relatively simple features. While the results are not perfect, they are important as they offer a starting point for improving maintainability.

\subsubsection{Interpretation of CB-1 Results}
\label{section:6.3.3}
In this section, the interpretation of the metric values obtained from the analysis of cb-1 is given. When the class-based metric values obtained from evaluating cb-1 are examined, there are a few points that attract attention. 

When the cb-1 values of the WMC metric used in complexity measurement are examined, it is seen that there are 2 problematic classes and some other classes have minor issues in terms of complexity. The classes A and B, which are mentioned in section \ref{section:5.3.4}, and have WMC values in the medium-high range, can be seen as problematic in terms of complexity. When evaluating the WMC metric values of the classes, it should be taken into account that classes with high complexity will have low maintainability characteristics.

If the cb-1 results of the DIT metric related to complexity and inheritance are examined, generally minor problems are observed. The DIT values of some classes seem to be at the middle-high level. There are a few inherited base classes (e.g. BaseActivity, BaseFragment), and these classes contain frequently used simple functionality. These classes are the ones that usually have higher DIT values. It can be said that this situation does not pose a serious problem when the application is considered. Fig. \ref{fig:cb-1-donuts} shows how the children classes inheriting these base classes reflect on the DIT results. Even though inheritance has a positive effect on the reusability level of the software systems, excessive and deep use of inheritance is considered a threat to maintainability. However, in this case, the usage of inheritance does not seem deep and excessive.

When looking at the NOC results, the classes seem to have low values. From this point of view, it can be said that the application of inheritance in the codebase is low. Although this situation can be interpreted as the coupling between the classes is low, when the evaluation results of the coupling metrics are examined, it is seen that this is not the case. However, low values of this metric also indicate low reusability of a software system. On the other hand, as stated in the previous paragraph, it is seen that some base classes and their inheritance in the codebase are reflected in NOC results as well. This situation can be seen when examining Fig. \ref{fig:cb-1-donuts}. Classes with low-medium NOC values correspond to these base classes.

When the COB metric values of the classes are examined, it is seen that the application has some problems in terms of coupling. It can be said that classes with high CBO values tend to have low reusability and low maintainability character. When these two classes with high and very high COB values are examined, it is seen that these classes are A and B classes mentioned in the paragraph where the WMC metric results were interpreted. It is worth noting that these two classes are problematic classes for cb-1. Since the SOLID principles and dependency injection applications were not handled properly while developing the cb-1, it would not be wrong to say that these results are not surprising for the CBO values.

When the results regarding the LCOM metric used in the measurement of cohesion level are examined, it is seen that the codebase has serious problems with the cohesion. This situation can be easily noticed when Fig. \ref{fig:cb-1-donuts} is examined. When Fig. \ref{fig:cb-1-package} is examined, the size of the classes and packages draw attention. This situation can be explained by the separation of responsibilities and the correct application of the "Single Responsibility" principle, the first of the SOLID principles. Classes that do not implement this principle correctly have different functionalities, independent of each other, which reduces cohesion.Classes with low cohesion are characterized as being closed to modification, and it is known that such classes tend to carry more than one responsibility in general. Classes of this character have low understandability, and this negatively affects maintainability. From this point of view, it would not be wrong to say that there are maintainability problems in a serious part of the application.

Considering the metric results in general, it is seen that although the evaluation is made over features with relatively low complexity (splash, login, register, etc.), problems from CB-1 stand out. Apart from the comments based on the above metric values, when Fig. \ref{fig:cb-1-package} is examined, the first thing that catches the eye is a complex and unorganized packaging structure. Layer and feature-based packaging methods are internal in the project, which causes serious maintainability, understandability and organization problems. It is also noteworthy that some classes are large in size, which can be interpreted as lack of proper separation of concerns. The also codebase appears to have complexity, coupling and cohesion problems even with these relatively low complication features. In a few classes, these problems are at a very high level, and maintainability and organization problems at different levels in the project draw attention. Especially the coupling problem stands out for this code base. Considering that there are no healthy abstraction and dependency injection applications in the project, this result is not very surprising. Apart from this, the problematic classes that draw attention in the paragraphs where metrics are interpreted appear in figure 23 as well with their abnormal sizes and their issues in complexity, coupling and cohesion. Large circles of red, orange and yellow represent these problematic classes. When all these analysis results are taken into account, it is clear that the cb-1 code base has complexity, coupling and cohesion problems even for simple features, and therefore shows a low maintainability character.

\subsubsection{Interpretation of CB-2 Results}
\label{section:6.3.4}
In this section, the interpretation of the metric values obtained from the analysis of cb-2 is presented.

When the cb-1 values of the WMC metric used in complexity measurement are examined, it is seen that almost all entities of this codebase have low WMC levels. These results are at the expected level, as it is known that this codebase applies SOLID principles and some of the Clean Code principles well.
The fact that most of the classes belonging to this codebase have low complexity character increases the maintainability of the codebase.

When the DIT results are examined, it is seen that there are similar results to the cb-1 results. The base class structure in cb-1 is also found in cb-2. Many children classes inherit these base classes. Therefore, DIT values for some classes are middle-high level. This also indicates that inheritance is widely used throughout the project. Besides, when the content of the traits in inheritance was examined, it was observed that these features were functions with low complexity that frequently repeat between classes.From this point of view, it would not be wrong to say that there is no excessive use of inheritance for this codebase. The current inheritance practice is in a way that will increase reusability and enable maintainability.

The NOC metric values of cb-2 are also similar to cb-1, and the reasons are similar to the reasons explained in the previous paragraph. It is seen that these values are complying with the DIT. In other words, the application of inheritance is not excessive and does not increase complexity. On the contrary, the use of this way of inheritance effectively utilizes reusability and thus increases maintainability. Besides, since the application is evaluated on only 4 features, the NOC values remain relatively low. These values can be higher, especially for the base classes, if a fully functional application is evaluated because more entities will inherit the base classes. However, this situation will not pose a problem in terms of complexity unless the DIT values are high; that is, the depth of inheritance is not high. The same is true for cb-1 as well.

When the COB metric values of the classes are examined, it is seen that the application is in a good situation in terms of coupling. Since abstraction, dependency inversion and dependency injection principles are applied very tightly for this codebase, it is expected that the application will be smooth or with minimal problems in terms of coupling. Since low coupling provides great advantages in terms of reusability and ease of modification, it makes a great contribution to the maintainability of this codebase.

The results for the LCOM metric of cb-2 display a rather interesting insight.
As explained earlier, LCOM measures the relationship between methods of a class. The low relationship between the methods of the classes indicates that a class has multiple responsibilities, which reduces the understandability and ease of modification of the class. Therefore, low cohesion means low maintainability. Considering that the principles of SOLID and SoC are strictly applied while developing the cb-2, it should be considered normal that the classes are concise and have a single responsibility. Therefore the cohesion values are also high. 

When the class-based metric values collected from assessing CB-2 are examined, it is shown that there are visible improvements in complexity, coupling and cohesion areas even for simple features such as splash, login, register. When Fig. \ref{fig:cb-2-donuts} and Fig. \ref{fig:cb-2-package} are examined, the second point that draws attention is that the dimensions are significantly reduced despite the increase in the number of classes and packages. Concise classes and the increase in the number of packages provides a more organized code base and a better-fragmented understanding of responsibility. This situation is also noticeable when the codebase is examined with the help of an IDE. The organization and understandability of cb-2 are at a high level. Besides, the levels of the classes in complexity, coupling are quite low, and cohesion is high. This situation can be shown as proof that the SOLID and SoC principles are applied correctly, and it can be said that there is a very positive effect on maintainability. On the other hand, when the project's bigger picture is examined (See Fig. 26), a few classes with moderate complexity and cohesion issues stand out. When these classes were investigated, it was seen that they were the classes called "View Model" in the MVVM design pattern. These classes are responsible for how and when the data will be displayed (in other words, display/view logic) in the MVVM design pattern. According to the principles of the MVVM design pattern, each view should have only one view model. In this case, the view models belonging to the views with more than one responsibility also have the logic of belonging to more than one responsibility. Therefore, these classes become more complex, and the cohesion of the classes decreases due to the methods and dependencies they have for different responsibilities. As long as the principles of the MVVM design pattern are followed, it would not be appropriate to divide these responsibilities between different classes. Nevertheless, the effects of the technology and principles used in the development of cb-2 on maintainability are obvious. It has been observed as a result of evaluating the cb-2 that these principles and technologies make a big difference even in the development of relatively simple features. While the results are not perfect, they are important as they offer a starting point for improving maintainability.

\subsubsection{Interpretation of Feature-Based Comparison}
\label{section:6.3.5}
In this section, the interpretation of the results obtained from the feature-based metric comparison over the login feature of cb-1 and cb-2 are shared.

First of all, it is seen that the cb-1 codebase uses 6 classes and interfaces for the visual part of the login feature, while cb-2 uses only 2 classes for this feature. From this point of view, the cb-2 code is better in terms of organization and understandability. On the other hand, when the metric values given in Fig. \ref{fig:login-metric-table} are examined, it is seen that the results of cb-2 are better than the results of cb-1.

When the metric values shared in Fig. \ref{fig:login-metric-table-2} are examined, the difference between the view responsibilities of the two projects on complexity draws attention. When the values of WMC, DIT and NOC metrics used in complexity measurement are compared, it is seen that the complexity level of cb-2 for the view responsibility is lower. On the other hand, when the results of the complexity metric values of view logic related responsibilities are examined, it is observed that there is not much difference between the two code bases. As explained in previous sections, these results should be considered normal given the complexity of functionality of the classes involved in this responsibility.

When the CBO metric related to measuring the coupling level is examined, it is seen that the cb-2 codebase gives better results. Since the SOLID and DI principles are used much more effectively in the cb-2 codebase, these results are not different from what is expected. As mentioned before, cb-1 uses the MVP design pattern. Coupling is increasing due to the bi-directional dependency between view and presentation layers in the MVP design pattern. It would not be wrong to say that this situation increases the coupling level of the cb-1. However, this situation is not the case for cb-2, which uses the MVVM design pattern. In the MVVM design pattern, only the view layer has a dependency on the presentation layer.

When the results of the LCOM metric related to cohesion are examined, it is seen that the results are similar to the results of the complexity metrics. While the results of cb-2 are much better than cb-1 for the responsibility of view, there is not much difference between the results for the responsibility of presentation. As explained in the previous section, there is only one view model per view principle in the MVVM design pattern. Therefore, one view model might have different responsibilities, especially those that belong to the complex views. Thus this circumstance can be shown as a reason for this situation. The same is true for the MVP design pattern as well. There can only be one presenter for each view. Naturally, cb-1, which uses the MVP design pattern, seems to have low cohesion values for presentation responsibility.

As a result, when looking at the general situation of the feature-based metric comparison, there are big improvements in the complexity and cohesion areas for the classes related to the view responsibility of cb-2 compared to the cb-1. At the same time, there is not much difference in terms of view logic/presentation responsibility. On the other hand, in the coupling, it has been determined that the cb-2 codebase is better than the cb-1, and the reasons for this situation have been mentioned in the previous sections.

\subsection{Limitations}
\input{chapters/6-discussion/6.4-limitations/6.4-text}

\clearpage
\section{Conclusion}
\label{section:7}
This study covered the subject of evaluating the impact of the methods and technologies used by the software product development company Mooncascade while developing Android applications on the maintainability of these applications.
There are four major challenges encountered when developing Android applications. These challenges are Android’s nature, demanding business needs, the frequent update rate of Android applications, and changing development teams. A maintainability model was formed by focusing on these major challenges. This model was formed based on the correlation between the major Android development challenges and the well-known software engineering concepts such as complexity, coupling and cohesion, whose relationships with maintainability were proven. While applying this maintainability model, quantitative and qualitative evaluation methods have been used. Quantitative and qualitative evaluation methods were determined to measure maintainability and the evaluations were carried out using these methods. To make a qualitative evaluation, a public Android survey was carried out with random Android developers and interviews were conducted with each member of the Android team of the company. To make the quantitative evaluation, a set of object-oriented software metrics were used. These metrics were applied to common features of the different codebases belonging to the same project through the CodeMR static code analysis tool.

First of all, this study has demonstrated the importance of maintainability for Android applications and addressed the important matters in terms of the maintainability of Android applications. These matters stand out as usage of principles and conventions to increase software understandability, implementing human-readable code, proper software architecture and design pattern selection and use of stable third-party libraries. Quantitative and qualitative evaluations proved that the maintainability of Android applications developed by paying attention to these matters will increase. Moreover, results indicated the positive impact of the methods and technologies used by the case company on the maintainability of Android applications. The comparison between the two codebases, one developed using the case company's methods and technologies, the other developed without a specific order and standard, showed that even for the relatively simple application features, maintainability had been increased. Outcomes also revealed the shortcomings and areas open to improvement for the methods and technologies used by the case company. The areas open to improvement are re-evaluating the use of libraries that are in danger of becoming outdated, such as RxJava, architectural scaling and selection according to the project, and making the coding conventions more standardized. It is predicted that re-evaluating these issues will further increase the positive effect on the maintainability of Android applications. 

Lastly, the findings obtained as a result of answering the first research question showed that a new model is needed to measure the maintainability of Android applications. The main reason for this need is the differences of Android applications from traditional software systems and their update rates. Especially Android's unique ecosystem points out the need for new methods and metrics to measure the maintainability of applications running on this ecosystem. While creating these new metrics, it is anticipated that besides the specific dynamics of Android applications, it may also be beneficial to use metrics that can include effort and time measurement that can be associated with high updating rates of the Android applications.

\subsection{Future Work}
Considering the lack of methods that can effectively measure the maintainability of Android applications, it would not be wrong to say that future research will focus on this issue as a continuation of this study. In addition, in the case of determining these methods, it is also among the targets to try the methods on more complex application features and get more effective results, thus eliminating the limitations of this study.

\newpage
\printbibliography
\addcontentsline{toc}{section}{References}

\newpage
\section*{Appendix}
\addcontentsline{toc}{section}{Appendix}
%\section*{I. Glossary}
%\addcontentsline{toc}{subsection}{I. Glossary}
%\TODO{What to do here?}
%\newpage

%=== Licence in English
\newcommand{\licencehint}[2]{\\\hspace*{#1}\textsl(#2)\par}
\newcommand\EngLicence{{%
\selectlanguage{english}
\section*{I. Licence}

\addcontentsline{toc}{subsection}{II. Licence}

\subsection*{Non-exclusive licence to reproduce thesis and make thesis public}
I, \textbf{Mustafa Ogün Öztürk}, %author's name

\begin{enumerate}
\item
herewith grant the University of Tartu a free permit (non-exclusive licence) to
\par
reproduce, for the purpose of preservation, including for adding to the DSpace digital archives until the expiry of the term of copyright,
\par
\textbf{Evaluating Maintainability of Android Applications: Mooncascade Case Study}, %

\par
supervised by Jakob Mass. %supervisor's name
  
\item
I grant the University of Tartu a permit to make the work specified in p. 1 available to the public via the web environment of the University of Tartu, including via the DSpace digital archives, under the Creative Commons licence CC BY NC ND 3.0, which allows, by giving appropriate credit to the author, to reproduce, distribute the work and communicate it to the public, and prohibits the creation of derivative works and any commercial use of the work until the expiry of the term of copyright.
\item
I am aware of the fact that the author retains the rights specified in p. 1 and 2.
\item
I certify that granting the non-exclusive licence does not infringe other persons' intellectual property rights or rights arising from the personal data protection legislation. 
\end{enumerate}

\noindent
Mustafa Ogün Öztürk\\ %author's name
\textbf{\textsl{14/05/2021}}
}}%\newcommand\EngLicence

%=== Licence in Estonian
\newcommand\EstLicence{{%
\selectlanguage{estonian}
\section*{II. Litsents}

\addcontentsline{toc}{subsection}{II. Litsents}

\subsection*{Lihtlitsents lõputöö reprodutseerimiseks ja üldsusele kättesaadavaks tegemiseks}

Mina, \textbf{Alice Cooper}, %author's name
  \licencehint{10mm}{autori nimi}

\begin{enumerate}
\item
annan Tartu Ülikoolile tasuta loa (lihtlitsentsi) minu loodud teose
\par
\textbf{Tüübituletus neljandat järku loogikavalemitele}, %title of thesis
    \licencehint{10mm}{lõputöö pealkiri}
\par
mille juhendaja(d) on Axel Rose ja May Flower, %supervisor's name(s)
  \licencehint{10mm}{juhendaja nimi}
\par
reprodutseerimiseks eesmärgiga seda säilitada, sealhulgas lisada digitaalarhiivi DSpace kuni autoriõiguse kehtivuse lõppemiseni.
\par
\item
Annan Tartu Ülikoolile loa teha punktis 1 nimetatud teos üldsusele kättesaadavaks Tartu Ülikooli veebikeskkonna, sealhulgas digitaalarhiivi DSpace kaudu Creative Commonsi litsentsiga CC BY NC ND 3.0, mis lubab autorile viidates teost reprodutseerida, levitada ja üldsusele suunata ning keelab luua tuletatud teost ja kasutada teost ärieesmärgil, kuni autoriõiguse kehtivuse lõppemiseni.
\item
Olen teadlik, et punktides 1 ja 2 nimetatud õigused jäävad alles ka autorile.
\item
Kinnitan, et lihtlitsentsi andmisega ei riku ma teiste isikute intellektuaalomandi ega isikuandmete kaitse õigusaktidest tulenevaid õigusi. 
\end{enumerate}

\noindent
Alice Cooper\\ %author's name
\textbf{\textsl{pp.kk.aaaa}}
}}%\newcommand\EstLicence


%===Choose the licence in active language
\iflanguage{english}{\EngLicence}{\EstLicence}

\end{document}