More than fifty studies were found and reviewed as a result of the execution of the search query. Grey literature results are not included in these numbers (e.g. blog posts, official documentation, books). However, the reviewed grey literature was also cited in this study. Moreover, twenty-six studies amongst the literature results were reviewed in detail since they are closely related to this study's field of work. These studies can be grouped according to their topics as follows.
\begin{itemize}
    \item 9 papers related to Android app maintainability \cite{34,50,52,2,53,43,44,18,19}.
    \item 4 papers related to Android app architecture \cite{56,14,48,47}.
    \item 10 papers related to software maintainability \cite{23,26,33,36,45,4,22,46,35,49}.
    \item 3 papers related to software architecture \cite{27,28,25}.
\end{itemize}

In addition to the numbers mentioned above, several other studies were reviewed to better understand the studies conducted regarding software maintainability, metrics that can be used to measure software maintainability, general software engineering principles such as separation of concerns, and SOLID. The literature review also showed that there are many different ways to evaluate software maintainability \cite{33,36,45,35,49}.  

Based on these numbers, it can be said that the importance of maintainability for software systems and Android applications is recognized by the researchers. There are several remarkable studies focusing on the evaluation of the maintainability of Android applications. It would be helpful to refer to these studies previously conducted in this field to understand the topic and be aware of possible limitations. Although the literature review showed that some studies approached the maintainability of Android applications differently, examining the maintainability is similar in most studies. 

For example, Ivanov et al. (2020) emphasize the importance of maintainability for Android applications, and they investigate the evolution of various maintainability issues along the lifetime of Android applications. Their results uncover the frequency and evolution of maintainability issues of Android applications and point to the frequent maintainability issues \cite{53}.

Hugo Källstrom (2016) conducted a study on the maintainability of Android applications. He implemented three different design patterns (MVC, MVP, MVVM) and evaluated the maintainability of these applications by comparing their performance, modifiability, and testability levels. Results of this study do not provide much insight into the impact of the used design patterns on maintainability \cite{18}. 

Also, Prabowo et al. (2018) researched the impact of the MVP and anti-pattern approaches on the maintainability of Android applications. They compared maintainability between two applications built with design pattern (MVP) and without design pattern using maintainability metrics such as LoC, Halstead Effort, and Cyclomatic Complexity. Their results showed that usage of MVP increases the maintainability level of the Android applications \cite{19}. 

Saifan and Al-Rabad (2017) present a different perspective on measuring the maintainability of Android applications by focusing on analyzing a set of Object-Oriented metrics and Android Metrics. Their research reveals that maintainability can be measured in different ways, and the metrics to be used depends on the source code.  They also mentioned that Android applications have special features that distinguished them, and to measure the maintainability of Android applications new method is needed \cite{34}. 

The study of Andrä et al. (2020) draws attention to the most recent research on this subject. They use several different maintainability metrics and match these metrics with different "Clean Code" conventions to evaluate the maintainability of the Android applications developed with Kotlin in class and method level. They tried different static code analysis tools for this purpose. Their results showed that there is currently no suitable tool or plugin for calculating maintainability metrics for Kotlin applications \cite{43}. 

Panca et al. (2016) evaluate the impact of design pattern selection on the maintainability of Android applications. For that purpose, they implemented different applications using different design patterns such as singleton, memento, state, iterator, factory, builder, and flyweight. They used maintainability metrics such as LoC, Halstead Effort, and Cyclomatic Complexity for maintainability evaluation and compared the maintainability values of different applications. Their results showed that the maintainability metrics value is increased compared to before using anti-pattern \cite{52}. 

In addition, the research of Verdecchia et al. (2019) on software architecture choices in Android applications provides insights into the relationship between software architecture, and maintainability \cite{14}.

Apart from these studies, some detailed studies on metrics and methods that can be used for the maintainability of software systems are important for understanding the maintainability evaluation. For example, Shyam R. Chidamber and Chris F. Kemerer (1994) proposed a set of software metrics for object-oriented design, respectively, Weighted Methods Per Class (WMC), Depth of Inheritance Tree (DIT), Number of children (NOC), Coupling between objects (CBO) Metric 5: Response For a Class (RFC), Lack of Cohesion in Methods (LCOM) \cite{36}. The research of Barak et al. (2012) also presents that many object-oriented software metrics can be used to measure the maintainability (including the ones mentioned above) of software. Still, a few metrics are not applicable in predicting the maintainability of this approach \cite{33}. I. Heitlager et al. (2007) identifies the problems of a common maintainability evaluation method, "maintainability Index" (MI) and propose a new maintainability model by defining several requirements for a proper maintainability evaluation method. They also identify foremost maintainability characteristics as analysability, changeability, stability, testability \cite{45}.