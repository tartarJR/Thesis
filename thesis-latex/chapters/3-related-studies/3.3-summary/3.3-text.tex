Results of the literature review showed that most studies approach the maintainability of Android applications from a software architecture/design pattern perspective only. Other techniques and technologies can also be used to increase the maintainability of Android applications besides architectural patterns. Still, there are very few studies that focused on these techniques and technologies. The fact that the studies do not focus on these techniques and technologies can be considered a limitation of the academic literature regarding this topic. Apart from this, it is seen that similar metrics and methods are used in many studies in this field. Considering the differences of Android applications compared to other software, it can be mentioned that different methods should be used. In addition, it was not possible to come across studies involving relatively new Android technologies. In some studies, the difficulties of evaluating Android applications developed with a relatively new programming language such as Kotlin in terms of maintainability are also addressed. In general, results indicate the need for a new maintainability model for Android applications.