%\section*{I. Glossary}
%\addcontentsline{toc}{subsection}{I. Glossary}
%\TODO{What to do here?}
%\newpage

%=== Licence in English
\newcommand{\licencehint}[2]{\\\hspace*{#1}\textsl(#2)\par}
\newcommand\EngLicence{{%
\selectlanguage{english}
\section*{I. Licence}

\addcontentsline{toc}{subsection}{I. Licence}

\subsection*{Non-exclusive licence to reproduce thesis and make thesis public}
I, \textbf{Mustafa Ogün Öztürk}, %author's name

\begin{enumerate}
\item
herewith grant the University of Tartu a free permit (non-exclusive licence) to
\par
reproduce, for the purpose of preservation, including for adding to the DSpace digital archives until the expiry of the term of copyright,
\par
\textbf{Evaluating Maintainability of Android Applications: Mooncascade Case Study}, %

\par
supervised by Jakob Mass. %supervisor's name
  
\item
I grant the University of Tartu a permit to make the work specified in p. 1 available to the public via the web environment of the University of Tartu, including via the DSpace digital archives, under the Creative Commons licence CC BY NC ND 3.0, which allows, by giving appropriate credit to the author, to reproduce, distribute the work and communicate it to the public, and prohibits the creation of derivative works and any commercial use of the work until the expiry of the term of copyright.
\item
I am aware of the fact that the author retains the rights specified in p. 1 and 2.
\item
I certify that granting the non-exclusive licence does not infringe other persons' intellectual property rights or rights arising from the personal data protection legislation. 
\end{enumerate}

\noindent
Mustafa Ogün Öztürk\\ %author's name
\textbf{\textsl{14/05/2021}}
}}%\newcommand\EngLicence

%=== Licence in Estonian
\newcommand\EstLicence{{%
\selectlanguage{estonian}
\section*{II. Litsents}

\addcontentsline{toc}{subsection}{II. Litsents}

\subsection*{Lihtlitsents lõputöö reprodutseerimiseks ja üldsusele kättesaadavaks tegemiseks}

Mina, \textbf{Alice Cooper}, %author's name
  \licencehint{10mm}{autori nimi}

\begin{enumerate}
\item
annan Tartu Ülikoolile tasuta loa (lihtlitsentsi) minu loodud teose
\par
\textbf{Tüübituletus neljandat järku loogikavalemitele}, %title of thesis
    \licencehint{10mm}{lõputöö pealkiri}
\par
mille juhendaja(d) on Axel Rose ja May Flower, %supervisor's name(s)
  \licencehint{10mm}{juhendaja nimi}
\par
reprodutseerimiseks eesmärgiga seda säilitada, sealhulgas lisada digitaalarhiivi DSpace kuni autoriõiguse kehtivuse lõppemiseni.
\par
\item
Annan Tartu Ülikoolile loa teha punktis 1 nimetatud teos üldsusele kättesaadavaks Tartu Ülikooli veebikeskkonna, sealhulgas digitaalarhiivi DSpace kaudu Creative Commonsi litsentsiga CC BY NC ND 3.0, mis lubab autorile viidates teost reprodutseerida, levitada ja üldsusele suunata ning keelab luua tuletatud teost ja kasutada teost ärieesmärgil, kuni autoriõiguse kehtivuse lõppemiseni.
\item
Olen teadlik, et punktides 1 ja 2 nimetatud õigused jäävad alles ka autorile.
\item
Kinnitan, et lihtlitsentsi andmisega ei riku ma teiste isikute intellektuaalomandi ega isikuandmete kaitse õigusaktidest tulenevaid õigusi. 
\end{enumerate}

\noindent
Alice Cooper\\ %author's name
\textbf{\textsl{pp.kk.aaaa}}
}}%\newcommand\EstLicence

%===Choose the licence in active language
\iflanguage{english}{\EngLicence}{\EstLicence}

\newpage
\section*{II. Attachments}
\addcontentsline{toc}{subsection}{II. Attachments}
The thesis is supplied with accompanying files.
Answer data set for the Android developer survey can be found on GitHub with the following URL: https://github.com/tartarJR/Thesis