%=== Info in English
\newcommand\EngInfo{{%
\selectlanguage{english}
\noindent\textbf{\large Evaluating Maintainability of Android Applications: Mooncascade Case Study}

\vspace*{3ex}

\noindent\textbf{Abstract:}

\noindent
%\textsc{Whitespace}
Android became one of the most extensive mobile platforms and active software development branches in the last decade. Android applications are known for their frequent updates to meet changing user requirements. At his point, maintainability emerges as a key non-functional requirement for Android applications because it determines how understandable, modifiable Android applications are and how easy it is to maintain them. Thus, developing maintainable Android applications is vital, especially for software product development companies with fast development and client circulation. In this case study, the technologies and methods used when developing Android applications by Mooncascede, a software product development company, will be evaluated from the maintainability point of view. The primary goal of this study is to identify and present the impact of these technologies and the methods on the maintainability of Android applications.

\vspace*{1ex}

\noindent\textbf{Keywords:} Android, Maintainability, Metrics, Software Engineering

\vspace*{1ex}

\noindent\textbf{CERCS:} P170 Computer science, numerical analysis, systems, control

\vspace*{1ex}
}}%\newcommand\EngInfo


%=== Info in Estonian
\newcommand\EstInfo{{%
\selectlanguage{estonian}
\noindent\textbf{\large Tüübituletus neljandat järku loogikavalemitele}
\vspace*{1ex}

\noindent\textbf{Lühikokkuvõte:} 

\noindent
Androidist sai viimase kümnendi jooksul üks ulatuslikumaid mobiilplatvorme ja aktiivseid tarkvaraarendusharusid. Androidi rakendused on tuntud oma sagedaste värskenduste poolest, et need vastaksid muutuvatele kasutajate nõuetele. Tema sõnul kerkib hooldatavus Androidi rakenduste peamise mittefunktsionaalse nõudena, sest see määrab, kui arusaadavad, muudetavad Androidi rakendused on ja kui lihtne on neid hooldada. Seega on ülihooldatavate Android-rakenduste väljatöötamine ülitähtis, eriti kiire arenduse ja klientide ringlusega tarkvaratoodete arendamise ettevõtete jaoks. Selles juhtumiuuringus hinnatakse tarkvaratoodete arendusettevõtte Mooncascede Androidi rakenduste väljatöötamisel kasutatud tehnoloogiaid ja meetodeid hooldatavuse seisukohast. Selle uuringu esmane eesmärk on tuvastada ja tutvustada nende tehnoloogiate ja meetodite mõju Androidi rakenduste hooldatavusele.

\vspace*{1ex}

\noindent\textbf{Võtmesõnad:} Android, hooldatavus, mõõdikud, tarkvaratehnika

\vspace*{1ex}

\noindent\textbf{CERCS:} P170 Arvutiteadus, arvanalüüs, süsteemid, kontroll

\vspace*{1ex}
}}%\newcommand\EstInfo