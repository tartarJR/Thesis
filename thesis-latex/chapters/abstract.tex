%=== Info in English
\newcommand\EngInfo{{%
\selectlanguage{english}
\noindent\textbf{\large Evaluating Maintainability of Android Applications: Mooncascade Case Study}

\vspace*{3ex}

\noindent\textbf{Abstract:}

\noindent
%\textsc{Whitespace}
Android became one of the most comprehensive mobile platforms in the last decade.  This comprehensiveness also brought more challenges to the Android application development. Android’s nature, demanding business needs, the frequent update rate of Android applications, and lastly, changing development teams are the four major challenges for Android applications. Maintainability is defined as how easy it is to update, modify, and maintain software. At this point, maintainability emerges as a key concept because developing maintainable Android applications facilitate the above-mentioned difficulties. The primary goal of this study is to evaluate the impact of the technologies and the methods used to develop Android applications by Mooncascede, a software product development company, on maintainability. These methods and technologies include principles (e.g. Clean Code, SOLID), architectural/design patterns (Clean Architecture, MVVM), and third-party libraries (RxJava, Dagger 2 and so on). The evaluation was conducted using the triangulation strategy, which is a mixed-method approach. Qualitative evaluation was conducted via interviews with the case company's Android team (7 participants) and an Android developer survey filled by anonymous developers (over 150 participants). Also, quantitative evaluation was made via object-oriented software metrics. Study results reveal the positive impact of the evaluated methods and technologies on the maintainability of Android applications while pointing to the need for improvements. Results also indicate the need for a new maintainability model specific to the Android applications.

\vspace*{1ex}

\noindent\textbf{Keywords:} Android, Maintainability, Object-Oriented Metrics, Software Engineering

\vspace*{1ex}

\noindent\textbf{CERCS:} P170 Computer science, numerical analysis, systems, control

\vspace*{1ex}
}}%\newcommand\EngInfo


%=== Info in Estonian

\newcommand\EstInfo{{%
\selectlanguage{estonian}
\newpage
\noindent\textbf{\large Androidi rakenduste hooldatavuse hindamine: Mooncascade juhtumianalüüs}
\vspace*{1ex}

\noindent\textbf{Lühikokkuvõte:} 

\noindent
Android on saanud viimase kümnendi üheks kõikehõlmavaimaks mobiiliplatvormiks.  Samas on see omadus toonud kaasa ka katsumusi Androidi rakenduste arendamisel. Androidi rakenduste puhul on neli peamist proovikivi Androidi olemus, nõudlikud ärivajadused, rakenduste sagedane uuendamine ja muutuvad arendusmeeskonnad. Hooldatavus tähistab seda, kui lihtne on tarkvara uuendada, muuta ja hooldada. Praegu on sellest kujunemas peamine parameeter, sest hooldatavate Androidi rakenduste arendamine leevendab eelmainitud probleeme. Uurimuse põhieesmärk on hinnata Androidi rakenduste arendamisel tarvaarendusettevõtte Mooncascade kasutatud tehnoloogia ja meetodite mõju hooldatavusele. Need meetodid ja tehnoloogia hõlmavad põhimõtteid (nt Clean Code ehk puhas kood, SOLID), arhitektuurilisi ja disainimustreid (Clean Architecture ehk puhas arhitektuur, MVVM) ning kolmandate poolte teeke (RxJava, Dagger 2 jne). Hindamisel kasutati mitut meetodit hõlmavat triangulatsioonistrateegiat. Kvalitatiivse hindamise käigus tehti intervjuud uuritava ettevõtte Androidi meeskonnaga (7 osalejat) ja korraldati anonüümne küsimustik Androidi arendajatele (üle 150 osaleja). Lisaks tehti objektorienteeritud tarkvara näitajate kvantitatiivne hindamine. Uurimuse tulemustest nähtub hinnatud meetodite ja tehnoloogia positiivne mõju Androidi rakenduste hooldatavusele, aga ka täiustusvajadus. Samuti selgus tulemustest vajadus uue hooldatavusmudeli järele, mis on spetsiifiline Androidi rakendustele.

\vspace*{1ex}

\noindent\textbf{Võtmesõnad:} Android, hooldatavus, objektile suunatud mõõdikud, tarkvaratehnika

\vspace*{1ex}

\noindent\textbf{CERCS:} P170 Arvutiteadus, arvanalüüs, süsteemid, kontroll

\vspace*{1ex}
}}%\newcommand\EstInfo