%=== Info in English
\newcommand\EngInfo{{%
\selectlanguage{english}
\noindent\textbf{\large Evaluating Maintainability of Android Applications: Mooncascade Case Study}

\vspace*{3ex}

\noindent\textbf{Abstract:}

\noindent
%\textsc{Whitespace}
Android became one of the most comprehensive mobile platforms in the last decade. Nowadays, Android applications are known for their frequent updates to meet changing user requirements. At this point, maintainability emerges as a key concept because it determines how easy it is to update, modify, and maintain software. The primary goal of this study is to evaluate the impact of the technologies and the methods used to develop Android applications by Mooncascede, a software product development company, on maintainability. The evaluation was conducted using the triangulation strategy, which is a mixed-method approach. Qualitative evaluation was conducted via interviews with the case company's Android team and an Android developer survey filled by anonymous developers. Also, quantitative evaluation was made via object-oriented software metrics. Study results reveal the positive impact of the evaluated methods and technologies on the maintainability of Android applications while pointing to the need for improvements.

\vspace*{1ex}

\noindent\textbf{Keywords:} Android, Maintainability, Metrics, Software Engineering

\vspace*{1ex}

\noindent\textbf{CERCS:} P170 Computer science, numerical analysis, systems, control

\vspace*{1ex}
}}%\newcommand\EngInfo


%=== Info in Estonian
\newcommand\EstInfo{{%
\selectlanguage{estonian}
\noindent\textbf{\large Tüübituletus neljandat järku loogikavalemitele}
\vspace*{1ex}

\noindent\textbf{Lühikokkuvõte:} 

\noindent
Android on saanud viimase kümnendi üheks kõikehõlmavaimaks mobiiliplatvormiks. Tänapäeval on Androidi rakendused tuntud sagedaste uuenduste poolest, millega püütakse tulla vastu kasutajate muutuvatele nõudmistele. Praegu on hooldatavusest kujunemas peamine parameeter, mis määrab, kui lihtne on tarkvara uuendada, muuta ja hooldada. Uurimuse põhieesmärk on hinnata Androidi rakenduste arendamisel tarvaarendusettevõtte Mooncascade kasutatud tehnoloogia ja meetodite mõju hooldatavusele. Hindamisel kasutati mitut meetodit hõlmavat triangulatsioonistrateegiat. Kvalitatiivse hindamise käigus tehti intervjuud uuritava ettevõtte Androidi meeskonnaga ja korraldati anonüümne küsimustik Androidi arendajatele. Lisaks tehti objektorienteeritud tarkvara näitajate kvantitatiivne hindamine. Uurimuse tulemustest nähtub hinnatud meetodite ja tehnoloogia positiivne mõju Androidi rakenduste hooldatavusele, aga ka täiustusvajadus.

\vspace*{1ex}

\noindent\textbf{Võtmesõnad:} Android, hooldatavus, mõõdikud, tarkvaratehnika

\vspace*{1ex}

\noindent\textbf{CERCS:} P170 Arvutiteadus, arvanalüüs, süsteemid, kontroll

\vspace*{1ex}
}}%\newcommand\EstInfo