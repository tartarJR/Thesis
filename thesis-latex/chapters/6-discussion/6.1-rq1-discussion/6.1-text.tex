First of all, the information obtained while answering the first research question guided the whole of this study. Research conducted to answer RQ1 has shown that the use of quantitative measurements together with qualitative measures can increase the effectiveness of the study. It was seen that qualitative measurements can make important contributions to the results of the evaluation with the information obtained directly from the developers and support the quantitative measurement results. The fact that experienced Android developers make evaluations about the methods and technologies they use every day from the maintainability point of view and the results obtained from these evaluations can be added to this study increased the study's accuracy. From this point of view, it would not be wrong to say that the addition of qualitative methods as well as quantitative methods to studies focusing on the measurement of software development concepts such as maintainability would increase the qualification of the study. The research conducted to answer the first research question showed that the maintainability of object-oriented software systems can be evaluated quantitatively by using many different metrics. In this study, a different quantitative maintainability model based on the concepts of complexity, coupling and cohesion was formed to measure the maintainability of Android applications. This model was created by inspiring the problems encountered while developing Android applications, mentioned in section \ref{section:1.1}. Subsequently, 5 metrics that are suitable for this evaluation model and can measure these concepts were determined. However, it cannot be said that these methods and metrics used in this study is the best that can be used for this purpose. Although these methods and metrics were sufficient for this study, it would not be wrong to say that there may be more effective quantitative measurement methods. It would be appropriate to conduct comparative studies to determine the most effective solution. 