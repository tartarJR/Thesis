First of all, it should be stated that the features used in quantitative evaluations were relatively less complex features of the Android application. Therefore the efficiency of evaluation and comparison was affected by this situation. Using more complex features could have provided better insight into the impact of the methods and technologies used by the case company. Unfortunately, this was not possible within the scope of this study due to not having the time required to develop more complex features. Still, it does not mean that these results are false or unrealistic. Moreover, considering that there may be insufficiencies in the methods used in the evaluations, the necessity to carry out more detailed studies to find a proper method for evaluating the maintainability of Android applications is undeniable. For example, there are different methods to evaluate the software system's maintainability, and different metrics can be used. Therefore choosing the most efficient metrics to measure the maintainability of software systems and Android applications is controversial, especially when the differences of Android applications are taken into account. Further research should be conducted to find the most efficient quantitative evaluation method.