When evaluating the WMC metric values of the classes, it should be taken into account that classes with high complexity will have low maintainability characteristics.

These two classes, which have WMC values in the medium-high range, can be seen as problematic in terms of complexity.

Although inheritance has a positive effect on the reusability level of the software systems, excessive and deep use of inheritance is considered a threat to maintainability. However, in this case usage of inheritance does not seem deep and excessive.

From this point of view, it can be said that the inheritance among the classes is low. Although this situation can be interpreted as the coupling between the classes is low, when the evaluation results of the next metric are examined, it is seen that this is not the case. However, low values of this metric also indicate low reusability of a software system.

When the COB metric values of the classes are examined, it is seen that the application has some problems in terms of coupling.

It can be said that classes with high CBO values tend to have low reusability and low maintainability character. When these two classes with high and very high COB values are examined, it is seen that these classes are A and B classes mentioned in the paragraph where the WMC metric results were interpreted. It is worth noting that these two classes are problematic classes for cb-1.

Classes with low cohesion are characterized as being closed to modification, and it is known that such classes tend to carry more than one responsibility in general. Classes of this character have low understandability, and this negatively affects maintainability. From this point of view, it would not be wrong to say that there are maintainability problems in a serious part of the application.

When the class-based metric values obtained from evaluating CB-1 are examined, there are a few points that attract attention. Although the evaluation is made over features with relatively low complexity (splash, login, register, etc.), when the above shared metric values are examined based on classes, a few problematic classes of CB-1 stand out.

 When the project is examined in the light of this information, the first thing that catches the eye is a complex and unorganized packaging structure. Layer and feature-based packaging methods are internal in the project, which causes serious maintainability, understandability and organization problems. It is also noteworthy that some classes are large in size, which can be interpreted as lack of proper separation of concerns. The also codebase appears to have complexity, coupling and cohesion problems even with these relatively low complication features. In a few classes, these problems are at a very high level, and maintainability and organization problems at different levels in the project draw attention. Especially the coupling problem stands out for this code base. Considering that there are no healthy abstraction and dependency injection applications in the project, this result is not very surprising. Apart from this, the problematic classes that draw attention in the paragraphs where metrics are interpreted appear in figure 23 as well with their abnormal sizes and their issues in complexity, coupling and cohesion. Large circles of red, orange and yellow represent these problematic classes. When all these analysis results are taken into account, it is clear that the cb-1 code base has complexity, coupling and cohesion problems even for simple features, and therefore shows a low maintainability character.