The fact that most of the classes belonging to this codebase have low complexity character increases the maintainability of the codebase.

This indicates that inheritance is widely used throughout the project.
Besides, when the content of the traits in inheritance was examined, it was observed that these features were functions with low complexity that frequently repeat between classes.From this point of view, it would not be wrong to say that there is no excessive use of inheritance for this codebase. The current inheritance practice is in a way that will increase reusability and enable maintainability.

It is seen that these values are complying with the DIT values shared in the previous paragraph. In other words, the application of inheritance is not excessive and does not increase complexity. On the contrary, the use of this way of inheritance effectively utilizes reusability and thus increases maintainability.

When the COB metric values of the classes are examined, it is seen that the application is in a good situation in terms of coupling. 

Since abstraction, dependency inversion and dependency injection principles are applied very tightly for this codebase, it is expected that the application will be smooth or with minimal problems in terms of coupling. Since low coupling provides great advantages in terms of reusability and ease of modification, it makes a great contribution to the maintainability of this codebase.

The results for the LCOM metric of cb-2 display a rather interesting insight.
As explained earlier, LCOM measures the relationship between methods of a class. The low relationship between the methods of the classes indicates that a class has multiple responsibilities, which reduces the understandability and ease of modification of the class. Therefore, low cohesion means low maintainability. Considering that the principles of SOLID and SoC are strictly applied while developing the cb-2, it should be considered normal that the classes are concise and have a single responsibility. Therefore the cohesion values are also high. 

When the class-based metric values collected from assessing CB-2 are examined, there are a few topics to point. First of all, it is determined that there are visible improvements in complexity, coupling and cohesion areas even for simple features such as splash, login, register. 

When Fig. \ref{fig:cb-2-donuts} and Fig. \ref{fig:cb-2-package} are examined, the second point that draws attention is that the dimensions are significantly reduced despite the increase in the number of classes and packages. Concise classes and the increase in the number of packages provides a more organized code base and a better-fragmented understanding of responsibility. This situation is also noticeable when the codebase is examined with the help of an IDE. The organization and understandability of cb-2 are at a high level. Besides, the levels of the classes in complexity, coupling are quite low, and cohesion is high. This situation can be shown as proof that the SOLID and SoC principles are applied correctly, and it can be said that there is a very positive effect on maintainability. On the other hand, when the project's bigger picture is examined (See Fig. 26), a few classes with moderate complexity and cohesion issues stand out. When these classes were investigated, it was seen that they were the classes called "View Model" in the MVVM design pattern. These classes are responsible for how and when the data will be displayed (in other words, display/view logic) in the MVVM design pattern. According to the principles of the MVVM design pattern, each view should have only one view model. In this case, the view models belonging to the views with more than one responsibility also have the logic of belonging to more than one responsibility. Therefore, these classes become more complex, and the cohesion of the classes decreases due to the methods and dependencies they have for different responsibilities. As long as the principles of the MVVM design pattern are followed, it would not be appropriate to divide these responsibilities between different classes. Nevertheless, the effects of the technology and principles used in the development of cb-2 on maintainability are obvious. It has been observed as a result of evaluating the cb-2 that these principles and technologies make a big difference even in the development of relatively simple features. While the results are not perfect, they are important as they offer a starting point for improving maintainability.