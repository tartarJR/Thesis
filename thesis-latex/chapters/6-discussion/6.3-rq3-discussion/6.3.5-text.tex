In this section, the interpretation of the results obtained from the feature-based metric comparison over the login feature of cb-1 and cb-2 are shared.

First of all, it is seen that the cb-1 codebase uses 6 classes and interfaces for the visual part of the login feature, while cb-2 uses only 2 classes for this feature. From this point of view, the cb-2 code is better in terms of organization and understandability. On the other hand, when the metric values given in Fig. \ref{fig:login-metric-table} are examined, it is seen that the results of cb-2 are better than the results of cb-1.

When the metric values shared in Fig. \ref{fig:login-metric-table-2} are examined, the difference between the view responsibilities of the two projects on complexity draws attention. When the values of WMC, DIT and NOC metrics used in complexity measurement are compared, it is seen that the complexity level of cb-2 for the view responsibility is lower. On the other hand, when the results of the complexity metric values of view logic related responsibilities are examined, it is observed that there is not much difference between the two code bases. As explained in previous sections, these results should be considered normal given the complexity of functionality of the classes involved in this responsibility.

When the CBO metric related to measuring the coupling level is examined, it is seen that the cb-2 codebase gives better results. Since the SOLID and DI principles are used much more effectively in the cb-2 codebase, these results are not different from what is expected. As mentioned before, cb-1 uses the MVP design pattern. Coupling is increasing due to the bi-directional dependency between view and presentation layers in the MVP design pattern. It would not be wrong to say that this situation increases the coupling level of the cb-1. However, this situation is not the case for cb-2, which uses the MVVM design pattern. In the MVVM design pattern, only the view layer has a dependency on the presentation layer.

When the results of the LCOM metric related to cohesion are examined, it is seen that the results are similar to the results of the complexity metrics. While the results of cb-2 are much better than cb-1 for the responsibility of view, there is not much difference between the results for the responsibility of presentation. As explained in the previous section, there is only one view model per view principle in the MVVM design pattern. Therefore, one view model might have different responsibilities, especially those that belong to the complex views. Thus this circumstance can be shown as a reason for this situation. The same is true for the MVP design pattern as well. There can only be one presenter for each view. Naturally, cb-1, which uses the MVP design pattern, seems to have low cohesion values for presentation responsibility.

As a result, when looking at the general situation of the feature-based metric comparison, there are big improvements in the complexity and cohesion areas for the classes related to the view responsibility of cb-2 compared to the cb-1. At the same time, there is not much difference in terms of view logic/presentation responsibility. On the other hand, in the coupling, it has been determined that the cb-2 codebase is better than the cb-1, and the reasons for this situation have been mentioned in the previous sections.