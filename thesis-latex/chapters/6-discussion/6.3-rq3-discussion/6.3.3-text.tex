In this section, the interpretation of the metric values obtained from the analysis of cb-1 is given. When the class-based metric values obtained from evaluating cb-1 are examined, there are a few points that attract attention. 

When the cb-1 values of the WMC metric used in complexity measurement are examined, it is seen that there are 2 problematic classes and some other classes have minor issues in terms of complexity. The classes A and B, which are mentioned in section \ref{section:5.3.4}, and have WMC values in the medium-high range, can be seen as problematic in terms of complexity. When evaluating the WMC metric values of the classes, it should be taken into account that classes with high complexity will have low maintainability characteristics.

If the cb-1 results of the DIT metric related to complexity and inheritance are examined, generally minor problems are observed. The DIT values of some classes seem to be at the middle-high level. There are a few inherited base classes (e.g. BaseActivity, BaseFragment), and these classes contain frequently used simple functionality. These classes are the ones that usually have higher DIT values. It can be said that this situation does not pose a serious problem when the application is considered. Fig. \ref{fig:cb-1-donuts} shows how the children classes inheriting these base classes reflect on the DIT results. Even though inheritance has a positive effect on the reusability level of the software systems, excessive and deep use of inheritance is considered a threat to maintainability. However, in this case, the usage of inheritance does not seem deep and excessive.

When looking at the NOC results, the classes seem to have low values. From this point of view, it can be said that the application of inheritance in the codebase is low. Although this situation can be interpreted as the coupling between the classes is low, when the evaluation results of the coupling metrics are examined, it is seen that this is not the case. However, low values of this metric also indicate low reusability of a software system. On the other hand, as stated in the previous paragraph, it is seen that some base classes and their inheritance in the codebase are reflected in NOC results as well. This situation can be seen when examining Fig. \ref{fig:cb-1-donuts}. Classes with low-medium NOC values correspond to these base classes.

When the COB metric values of the classes are examined, it is seen that the application has some problems in terms of coupling. It can be said that classes with high CBO values tend to have low reusability and low maintainability character. When these two classes with high and very high COB values are examined, it is seen that these classes are A and B classes mentioned in the paragraph where the WMC metric results were interpreted. It is worth noting that these two classes are problematic classes for cb-1. Since the SOLID principles and dependency injection applications were not handled properly while developing the cb-1, it would not be wrong to say that these results are not surprising for the CBO values.

When the results regarding the LCOM metric used in the measurement of cohesion level are examined, it is seen that the codebase has serious problems with the cohesion. This situation can be easily noticed when Fig. \ref{fig:cb-1-donuts} is examined. When Fig. \ref{fig:cb-1-package} is examined, the size of the classes and packages draw attention. This situation can be explained by the separation of responsibilities and the correct application of the "Single Responsibility" principle, the first of the SOLID principles. Classes that do not implement this principle correctly have different functionalities, independent of each other, which reduces cohesion.Classes with low cohesion are characterized as being closed to modification, and it is known that such classes tend to carry more than one responsibility in general. Classes of this character have low understandability, and this negatively affects maintainability. From this point of view, it would not be wrong to say that there are maintainability problems in a serious part of the application.

Considering the metric results in general, it is seen that although the evaluation is made over features with relatively low complexity (splash, login, register, etc.), problems from CB-1 stand out. Apart from the comments based on the above metric values, when Fig. \ref{fig:cb-1-package} is examined, the first thing that catches the eye is a complex and unorganized packaging structure. Layer and feature-based packaging methods are internal in the project, which causes serious maintainability, understandability and organization problems. It is also noteworthy that some classes are large in size, which can be interpreted as lack of proper separation of concerns. The also codebase appears to have complexity, coupling and cohesion problems even with these relatively low complication features. In a few classes, these problems are at a very high level, and maintainability and organization problems at different levels in the project draw attention. Especially the coupling problem stands out for this code base. Considering that there are no healthy abstraction and dependency injection applications in the project, this result is not very surprising. Apart from this, the problematic classes that draw attention in the paragraphs where metrics are interpreted appear in figure 23 as well with their abnormal sizes and their issues in complexity, coupling and cohesion. Large circles of red, orange and yellow represent these problematic classes. When all these analysis results are taken into account, it is clear that the cb-1 code base has complexity, coupling and cohesion problems even for simple features, and therefore shows a low maintainability character.