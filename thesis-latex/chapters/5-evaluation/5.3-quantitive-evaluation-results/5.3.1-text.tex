Before moving on to the results and comparisons of quantitative measurement methods, it will be useful to understand how and under what conditions these methods are applied to understand the results better. Previously quantitative measurement methods, metrics to be used and information on why they were chosen shared in section \ref{section:3.2}. More detailed information about the codebases to which these metrics were applied and why these codebases were selected will be shared in this subsection.

As mentioned in section \ref{section:3.2}, two codebases belonging to the same project have been selected to apply the metrics. The first of these codebases, defined as \textbf{\textit{code-base-1}} in this study, represents the unorganized codebase that has been poorly structured. The second of these codebases is a well-structured code base, which has been defined as \textbf{\textit{code-base-2}} and started to be developed using the technologies and methods mentioned in detail in section \ref{section:4}. To facilitate the explanation, these codebases will be mentioned in the form of \textbf{\textit{code-base-1}} and \textbf{\textit{code-base-2}} in the remainder of the study.

When code-base-1 is examined in detail, a situation like the following is encountered. First of all, the Android application's active version implemented with this project is working on this codebase, and it is actively used. The project was started to be developed using the Java programming language. Later, some features were developed using Kotlin programming language. There is no consistent choice of software architecture throughout the application. Although the application consists of 5 different modules, the modules' boundaries are not determined according to a certain standard. Some modules are feature-based, while some are layer-based. A similar situation is observed in packaging. The packaging organization of the application is inferior. While some features of the application have been developed with the MVP design pattern, some features have been developed with MVVM. It is controversial to what extent these design patterns are applied correctly. Static classes and singleton solutions have been used dangerously in practice. Besides, the SOLID principles have been partially ignored, while the dependency injection principles have been completely ignored. It is also worth noting the use of several outdated libraries. Finally, some low-maintainability manual solutions have been used for networking and database spaces. Also, when looking at the Git history of the application, the commits of 11 different developers are seen. This situation is critical in describing the developer circulation in the application and explaining its serious organisation problems. The relationship between the developer circulation and the maintainability of software systems was previously mentioned in section \ref{section:1.1}. Considering the date this application started to be developed, the use of old-fashioned technologies is normal, but the lack of organization throughout the application is unusual. This unusual situation and the other features listed above seriously affect the understandability and maintainability of the application. On the other hand, this situation offers a great opportunity for this study to measure maintainability.

When looking at code-base-2, it is seen that the opposite of code-base-1 is the case. It was decided to develop this version of the application because the problems in code-base-1 became more visible, and the sustainability problems increased. While this version of the application is being developed, methods, technologies and architecture, detailed in section \ref{section:4}, are used. It has started to be developed using Kotlin programming language, and the entire application is in Kotlin programming language.  Clean architecture and clean code, and SOLID principles are followed during the development. The application modules are separated based on the layers, and the packaging is feature-based. While developing the application, maintainable and reliable Android libraries have been used. The organisation level of the application's this version is very high, and it has been arranged to be a standard throughout the whole application. On the other hand,  development for this version of the application's is still ongoing, and there are only 4 main features that have already been developed. The features currently developed for code-base-2 are splash, login, register and main screen features. In this respect, code-base-2 falls short in terms of fully developed features compared to code-base-1.

The following section explains how both of these codebases were used while conducting the quantitative evaluation.


