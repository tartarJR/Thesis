In this section, detailed information about the codebases which the metrics mention in section \ref{section:4.2} are shared. As mentioned in section \ref{section:4.2}, two codebases belonging to the same project have been selected to apply the metrics. To facilitate the explanation, these codebases will be mentioned in the form of \textbf{\textit{CB-1}} and \textbf{\textit{CB-2}} in the remainder of the study.

When CB-1 is examined in detail, the following situation is encountered. First of all, it is the Android application's actively used version. The project was started to be developed using the Java programming language. Later, some features were developed using Kotlin programming language. There is no consistent choice of software architecture throughout the application. Although the application consists of 5 different modules, the modules' boundaries are not determined according to a certain standard. Some modules are feature-based, while some are layer-based. A similar situation is observed in packaging. The packaging organization of the application is inferior. While some features of the application have been developed with the MVP design pattern, some features have been developed with MVVM. It is controversial to what extent these design patterns are applied correctly. Static classes and singleton solutions have been used dangerously in practice. Besides, the SOLID principles have been ignored and no dependency injection is applied. It is also worth noting the use of some outdated libraries. Also, when looking at the Git history of the application, the commits of 11 different developers are seen. This situation is critical in describing the developer circulation in the application and explaining its serious organisation problems. The relationship between the developer circulation and the maintainability of software systems was previously mentioned in section \ref{section:1.1}.

CB-2 was developed as part of this study to evaluate the impact of the methods and technologies used by the case company (see \ref{section:2.3}) by comparing it to CB-1, a codebase developed without these methods and technologies. CB-2 is developed using Kotlin programming language. Clean architecture and clean code, and SOLID principles are followed during the development. The application modules are separated based on the layers of different responsibilities (view, presentation, domain, data, local storage, networking), and the packaging is feature-based. While developing the application, maintainable and reliable Android libraries have been used. The organisation level of this codebase is very high and consistent, and it has been arranged to be a standard throughout the whole application. On the other hand,  development for this version of the application's is still ongoing, and there are only 4 main features that have already been developed. The features currently developed for CB-2 are splash, login, register and main screen features. In this respect, CB-2 falls short in terms of developed features compared to CB-1. Therefore, the evaluation was only be made over the features that have been developed in both CB-1 and CB-2. The other features of CB-1 were removed from the codebase before the evaluation. Removal of such features was relatively easy since the developed features are quite independent of the rest of the application.


