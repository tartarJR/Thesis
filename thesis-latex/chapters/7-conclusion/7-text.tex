This study covered the subject of evaluating the impact of the methods and technologies used by the software product development company Mooncascade while developing Android applications on the maintainability of these applications.
There are four major challenges encountered when developing Android applications. These challenges are Android’s nature, demanding business needs, the frequent update rate of Android applications, and changing development teams. A maintainability model was formed by focusing on these major challenges. This model was formed based on the correlation between the major Android development challenges and the well-known software engineering concepts such as complexity, coupling and cohesion, whose relationships with maintainability were proven. While applying this maintainability model, quantitative and qualitative evaluation methods have been used. Quantitative and qualitative evaluation methods were determined to measure maintainability and the evaluations were carried out using these methods. To make a qualitative evaluation, a public Android survey was carried out with random Android developers and interviews were conducted with each member of the Android team of the company. To make the quantitative evaluation, a set of object-oriented software metrics were used. These metrics were applied to common features of the different codebases belonging to the same project through the CodeMR static code analysis tool.

First of all, this study has demonstrated the importance of maintainability for Android applications and addressed the important matters in terms of the maintainability of Android applications. These matters stand out as usage of principles and conventions to increase software understandability, implementing human-readable code, proper software architecture and design pattern selection and use of stable third-party libraries. Quantitative and qualitative evaluations proved that the maintainability of Android applications developed by paying attention to these matters will increase. Moreover, results indicated the positive impact of the methods and technologies used by the case company on the maintainability of Android applications. The comparison between the two codebases, one developed using the case company's methods and technologies, the other developed without a specific order and standard, showed that even for the relatively simple application features, maintainability had been increased. Outcomes also revealed the shortcomings and areas open to improvement for the methods and technologies used by the case company. The areas open to improvement are re-evaluating the use of libraries that are in danger of becoming outdated, such as RxJava, architectural scaling and selection according to the project, and making the coding conventions more standardized. It is predicted that re-evaluating these issues will further increase the positive effect on the maintainability of Android applications. 

Lastly, the findings obtained as a result of answering the first research question showed that a new model is needed to measure the maintainability of Android applications. The main reason for this need is the differences of Android applications from traditional software systems and their update rates. Especially Android's unique ecosystem points out the need for new methods and metrics to measure the maintainability of applications running on this ecosystem. While creating these new metrics, it is anticipated that besides the specific dynamics of Android applications, it may also be beneficial to use metrics that can include effort and time measurement that can be associated with high updating rates of the Android applications.

\subsection{Future Work}
Considering the lack of methods that can effectively measure the maintainability of Android applications, it would not be wrong to say that future research will focus on this issue as a continuation of this study. In addition, in the case of determining these methods, it is also among the targets to try the methods on more complex application features and get more effective results, thus eliminating the limitations of this study.