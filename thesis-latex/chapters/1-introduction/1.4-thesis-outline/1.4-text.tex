The rest of this study is structured as follows:

\noindent\textbf{2. Background:}

This section starts with some basic information about the Android environment and the nature of the Android applications. With this information, it aims to facilitate understanding of how Android's nature affects the maintainability and complexity of Android apps when developing Android applications.

Later, the section continues with some programming and software engineering fundamentals. First, it describes "maintainability" from the software engineering point of view. Later, it explains why these terms are essential for software engineering and then mainly in Android application development. Besides, the information regarding the issues caused by the lack of maintainability will be discussed. After that, the section explains what software architecture is and what relation it has with software maintainability. Besides, in general, software architecture in Android will be covered, and an overview of some popular solutions from the industry will be shared. Lastly, an overview of the white and gray literature review will be conducted as a part of this study.

\noindent\textbf{3. Research Methodology}

In this section, the metrics used while evaluating the tools, techniques, and technologies used by Mooncascade's Android team in terms of maintainability will be explained. Besides, the contents of the survey and interview to be made for the same purpose will be discussed in detail. Thus, the first research question will be answered in this section.

\noindent\textbf{4. Mooncascade Case Study}

The identified practices used by Mooncascade's Android team to tackle the problems mentioned in the introduction will be shared in this section. Also, extensive information on technologies and third-party libraries used by Mooncascade's Android team to achieve the goal of developing state-of-the-art Android applications with high maintainability will be provided. The practices, technologies, and techniques used by Mooncascade's Android team to accomplish the goal of developing state-of-the-art Android applications will be shared in the form of instructions and coding examples. The second research question will be answered in this section.

\newpage
\noindent\textbf{5. Evaluation}

The impact of the practices that are used by Mooncascade's Android team on the maintainability of Android applications will be evaluated in this section. The evaluation will be both in qualitative and quantitative form. To fulfill the evaluation in the qualitative form, surveys will be conducted amongst Mooncascade's Android team members. The quantitative evaluation will be fulfilled by using software maintainability metrics. Lastly, in order to identify the accuracy of the practices used by Mooncascade's Android team, a developer survey will be conducted amongst the Android community, and the results will be shared. The third research question will be answered in this section.

\noindent\textbf{6. Discussion}

This section will present the discussion and interpretation of answers to the research questions and evaluation results gathered in the previous section. Outcomes of the evaluations will be shared. The pros and cons of the researched practices in this study and their impact on Android application development will be interpreted from the maintainability point of view. Lastly, the limitations and restrictions of the study will be mentioned.

\noindent\textbf{7. Conclusion}

The study's epitome, along with the final thoughts and comments, will be presented in this section. Also, future research opportunities will be discussed.
