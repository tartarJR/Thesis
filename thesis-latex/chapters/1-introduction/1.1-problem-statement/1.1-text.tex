Mooncascade is a product development company based in Estonia. The company provides software development services, including Android application development. Demand for mobile applications in the industry has been increasing in the last decade. The company has been having an increasing number of requests for Android applications and facing several different challenges during the development and maintenance phases of these Android applications. These challenges are detailed as follows:

\noindent\textbf{Android platform-specific complexity:} A typical Android app consists of components such as activities, fragments, services, broadcast receivers, and content providers provided by the Android SDK and controlled by the Android OS. Besides, there are other components used in an Android app that are not a part of the Android SDK, e.g. managing network operations and business rules. The necessity of working together in harmony for all these components make Android applications complex software systems.

\noindent\textbf{Business-specific complexity:} Android applications get more and more sophisticated to fulfil increasing user needs, and apps become more business-critical. Consequently, the complexity of Android apps increases in terms of software complexity \cite{2}. This situation increases the complexity and affects the maintainability of the Android applications.

\noindent\textbf{High update rate:} Android applications have a high update rate because of bug fixes and the frequent addition of new features based on the changing business requirement and user needs. That makes the software development life cycles of the Android applications quite active \cite{3}. For that reason, Android applications should be developed in a way that modifications for new features and bug fixing can be done smoothly.

\noindent\textbf{Growing Codebases and Fast-changing Development Teams:} It gets harder to maintain Android applications as the codebase, and the development team grows or changes. The codebase of an Android application must be in an orderly fashion that enables developers to read and understand the app's purpose quickly.

The team aims to develop maintainable Android applications to solve these problems and have internalized some methods and technologies for this purpose. Although these methods and technologies are widely known by the Android community and are believed to be useful for maintainability, their impact on maintainability is empirically unknown. Thus, to evaluate the impact of these tools, techniques, and technologies used by Mooncascade's Android in terms of maintainability, this study will answer the following research questions. 

\begin{itemize}
\item \noindent\textbf{RQ1:} How can the maintainability of Android applications be measured?
\item \noindent\textbf{RQ2:} How efficient are the methods and technologies used by Mooncascade when developing Android applications in terms of software maintainability?
\end{itemize}