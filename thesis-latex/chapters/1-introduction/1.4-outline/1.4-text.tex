The rest of this study is structured as follows:

\noindent\textbf{2. Background:}
This chapter starts with brief information about the Android platform and Android SDK. Later, the chapter continues with key concepts for the maintainability of software systems. Lastly, the methods and technologies used by Mooncascade's Android team to tackle the problems mentioned in the first chapter are presented.

\noindent\textbf{3. Research Methodology}

In this section, the methods used to achieve the primary goal of this study are presented. The contents of the survey and interviews are discussed in detail. Also, metrics used while evaluating methods and technologies used by Mooncascade's Android team in terms of maintainability are explained. Thus, the first research question will be answered in this chapter.

\noindent\textbf{4. Evaluation}

The results for the impact of the practices that are used by Mooncascade's Android team on the maintainability of Android applications are shared in this chapter. The results obtained from the Android developer survey, interviews with the case company's Android developers and evaluation with the object-oriented metrics are presented, respectively.

\noindent\textbf{5. Discussion}

This chapter will present the discussion and interpretation of answers to the research questions and evaluation results obtained in the previous chapters. Outcomes of the evaluations will be shared. Also, the limitations of the study will be mentioned.

\noindent\textbf{6. Conclusion}

The study's epitome and results, along with the final thoughts and comments, will be presented in this section. Also, future research opportunities will be discussed.
