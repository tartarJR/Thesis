In the last decade, the impact of smartphones on our lives has significantly increased, and smartphones and mobile applications became people's primary way of interacting with technology. This situation has made the applications that work on these smartphones a vital part of daily and business life, from ordinary people to large companies. Today, mobile applications have become one of the most critical parts of digitalisation. Notably, as a successful open-source mobile operating system, Android has been a core element of this change, and the demand for Android applications has increased. Android application development has become one of the most necessary parts of the business area with a significant market share. Today, there are more than 2.5 billion active Android devices in the world \cite{1}. Therefore, it is difficult to ignore the importance of Android application development.
However, the increasing importance of the mobile era also brought more challenges to mobile application development and Android application development. Designing, developing, testing and understanding Android applications is becoming more complex \cite{34}. These challenges can be examined under four major topics. These are Android’s nature and its platform-specific components, sophisticated business needs, the frequent update rate of Android applications, and lastly, changing development teams. 

Mooncascade is a  product development company based in Estonia. The company provides different software development services to its clients, including Android development. With the increasing demand for Android applications, the company has been facing the previously mentioned challenges when providing Android application development services to different customers from different fields during the development process of Android applications. To overcome these challenges, "maintainability" emerges as one of the most important non-functional requirements. In the context of software engineering, "maintainability" is the ease with which a system or component can be modified for use in applications or environments other than those for which it was specifically designed \cite{20}. Developing maintainable Android applications is the way to survive in the competitive market time and cost-effectively \cite{50}. However, achieving this goal is challenging and costly \cite{34}.

This study aims to evaluate the impact of the methods and technologies used by Mooncascade when developing Android applications in terms of maintainability. Although evaluating software maintainability is not hard, the ways to be used are vast and choosing a suitable method might be challenging. Different methods and software metrics can be used to estimate the software maintainability value \cite{34,23,26,33,36,45}. For that purpose, this study follows the triangulation strategy \cite{51}. Both qualitative analyses (via interviews and surveys) and quantitative analysis (via object-oriented software quality metrics) are conducted. This method is preferred to collect data from different resources (academic resources and Android community) for obtaining more coherent results. The main motivation of this study is to find proper methods and technologies for developing maintainable applications to eliminate the four major problems in Android application development mentioned above. Thus improving developer/team efficiency and providing time and cost-efficient solutions to the clients. The study only focuses on native Android application development.

\subsection{Problem Statement}
\label{section:1.1}
Mooncascade provides software development services, including Android application development. Demand for mobile applications in the industry has been increasing in the last decade. The company has been having an increasing number of requests for Android applications and facing different challenges during the development and maintenance phases of these Android applications. These challenges are detailed as follows:

\noindent\textbf{Android platform complexity:} Android SDK is an over-engineered framework that imposes inherent complexity on the developers  \cite{56}. It is different from other software due to the application life cycle, methods for accessing resources between objects, multi-threading concept, and the layout creation process \cite{52}. An Android app consists of activities, fragments, services, broadcast receivers, and content providers provided by the Android SDK and controlled by the Android OS. The necessity of working together in harmony for all these components make Android applications complex software systems.

\noindent\textbf{Business-specific complexity:} Android applications get more and more sophisticated to fulfil increasing user needs and business requirements. Mobile applications become more functional as user and business needs increase. Consequently, the complexity of Android apps from the software development point of view increases. When this business-specific complexity comes together with the platform-specific complexity mentioned above, it will be vital to implement software engineering processes to ensure the development of high-quality Android applications \cite{2}. 

\noindent\textbf{High update rate:} Android applications have a high update rate \cite{3} because of bug fixes and the frequent addition of new features based on the changing business requirement and user needs. Thus, Android developers need to keep their applications maintainable. Because it gets harder to maintain Android applications as the codebase grows or changes \cite{34}. For that reason, Android applications should be developed in a way that modifications for new features and bug fixing can be done smoothly.

\noindent\textbf{Changing Development Teams:}  Developers of an Android application change during the life cycle of the application. Whenever a new developer joins the team, the time required to onboard a new developer to the codebase is directly related to how the Android applications are developed. Therefore, Android applications must be implemented in an orderly fashion that enables developers to quickly read and understand the app's purpose.

The team aims to develop maintainable Android applications to solve these problems and have internalized some methods and technologies for this purpose, such as coding conventions/principles (e.g. SOLID, Clean Code, Clean Architecture) and using some third-party libraries (RxJava, Dagger 2, etc.). Although these methods and technologies are widely known by the Android community and are believed to be useful for maintainability, their impact on maintainability is empirically unknown for the case company, Mooncascade. Thus, to find a suitable method for maintainability evaluation and then evaluate the impact of these tools, techniques, and technologies used by Mooncascade’s Android team in terms of maintainability, this study will answer the following research questions. 

\begin{itemize}
\item \noindent\textbf{RQ1:} What is the fitting way to evaluate the maintainability of the Android applications developed by Mooncascade?
\item \noindent\textbf{RQ2:} What is the impact of the methods and technologies used by Mooncascade to develop Android applications on the maintainability of Android applications?
\end{itemize}

To answer the first research question, research is conducted to find a suitable method for evaluating the maintainability of the Android applications. During this research, previously conducted studies in software maintainability are examined, and efforts are made to find the most suitable method for this case study. The study aimed to answer the second research question using the maintainability evaluation methods obtained from answering the first research question. For this purpose, the obtained methods are applied to two different codebases, one developed by the methods used by the case company and the other not developed by following a certain fashion.

\subsection{Contribution}
The methods and third-party libraries used in the development of Android applications, which are mentioned in this study, are well-known by the Android community. However, there is not enough empirical data about the impact of these practices and third-party libraries on an important software development concept such as maintainability. The author contributes to understanding how Android application codebases can be measured in terms of maintainability and providing empirical data on the impact of the well known Android development methods and technologies on maintainability. 

\newpage
\subsection{Thesis Outline}
This study's main contribution is providing comprehensive information regarding the development of large scale and enterprise Android applications through the experience of a proficient software development company, Mooncascade. Since the study topic is highly coupled to industry and industry trends, the author contributes to identifying and understanding the practices used by one of the top software development companies in the region and their impact and bringing those practices into the academy through this study. With the help of the information that this study offers, developers and researchers will know the fundamental principles and technologies required to develop state-of-the-art Android apps that have increased maintainability and quality. Thus solving the difficulties explained in the introduction section can be facilitated.

Examining white and gray literature has shown that there is no similar case study regarding Android application development. Still, some studies focus on maintainability and the quality of Android applications and solutions for challenges mentioned earlier. However, the examination has also shown that such studies lack detail and up-to-dateness. This study's valuable set of information might help fill the gap of outdatedness and lack of detail between the industry and academia in developing Android applications. Unlike most similar studies, in this study, the investigated principles and technologies will be gathered from the industry's real-life Android application development best practices. The study will include detailed information about the implementation of these principles and technologies. Apart from the main contribution, the evaluation results of presented methodologies and technologies will provide empirical data, both from the maintainability point of view and the Android developer's perspective. Lastly, the Android developer survey conducted as a part of this study can provide insights from the latest technology trends in the Android community.