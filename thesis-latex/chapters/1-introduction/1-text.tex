In the last decade, the impact of smartphones on our lives has significantly increased, and smartphones and mobile applications became people's primary way of interacting with technology. This situation has made the applications that work on these smartphones a vital part of daily and business life, from ordinary people to large companies. Today, mobile applications have become one of the most critical parts of digitalisation. Notably, as a successful open-source mobile operating system, Android has been a core element of this change, and the demand for Android applications has increased. Android application development has become one of the most necessary parts of the business area with a significant market share. Today, there are more than 2.5 billion active Android devices in the world \cite{1} . Therefore, it is difficult to ignore the importance of Android application development, considering the Android operating system's market share and many people's interactions with mobile applications running on Android devices. Consequently, Android application development has become an essential topic in the IT industry and academy. However, the increasing importance of the mobile era also brought more challenges to mobile application development and, of course, to Android application development.

Today, the difficulties that arise during Android application development can be examined under four major topics. These are Android's nature and its platform-specific components, demanding business requirements and sophisticated business needs, the frequent update rate of Android applications, and lastly, growing codebases and fast-changing development teams. When developing Android applications, it is essential to facilitate these difficulties. To overcome these difficulties, "maintainability" emerges as one of the most important non-functional requirements when developing Android applications. While developing Android applications, it is wise to use the technologies and techniques to increase the Android application's maintainability. Developing high-quality applications that can survive in the competitive Android market and achieving these goals in a time and cost-effective manner is only possible in this way.

This study's primary purpose is to identify and explain the methodologies and technologies used by Mooncascade, a leading software product development company with global reach, to develop real-world enterprise Android applications and evaluate these methodologies and technologies in terms of maintainability. To that aim, both qualitative analyses (in the forms of interviews, questionnaires for measuring the impact from the software complexity point of view) and quantitative analysis (measuring maintainability via object oriented software quality metrics) will be conducted.

Considering the scarcity of academic studies in this field and the inadequacy of these studies' content and actuality problems, this study aims to provide a solid and up-to-date resource to the Android field's academic studies. The inadequacy and not being up-to-date with academic studies in the Android field are among the main motivations for creating this study. Another important source of motivation for this study is the negative effects of the problems arising from the lack of maintainability in Android application development on the developer experiences and the elimination of these negative effects and thus, to improve  Android developer experience.

The target audience of this study includes Android developers and researchers who are already experienced with Android application development basics and willing to learn advanced techniques and tools for Android application development. The study facilitates developers and researchers to follow the most up-to-date software engineering and Android development practices to develop state-of-the-art Android applications with high maintainability. The study only focuses on native Android application development.

\subsection{Problem Statement}
\label{section:1.1}
Mooncascade provides software development services, including Android application development. Demand for mobile applications in the industry has been increasing in the last decade. The company has been having an increasing number of requests for Android applications and facing different challenges during the development and maintenance phases of these Android applications. These challenges are detailed as follows:

\noindent\textbf{Android platform complexity:} Android SDK is an over-engineered framework that imposes inherent complexity on the developers  \cite{56}. It is different from other software due to the application life cycle, methods for accessing resources between objects, multi-threading concept, and the layout creation process \cite{52}. An Android app consists of activities, fragments, services, broadcast receivers, and content providers provided by the Android SDK and controlled by the Android OS. The necessity of working together in harmony for all these components make Android applications complex software systems.

\noindent\textbf{Business-specific complexity:} Android applications get more and more sophisticated to fulfil increasing user needs and business requirements. Mobile applications become more functional as user and business needs increase. Consequently, the complexity of Android apps from the software development point of view increases. When this business-specific complexity comes together with the platform-specific complexity mentioned above, it will be vital to implement software engineering processes to ensure the development of high-quality Android applications \cite{2}. 

\noindent\textbf{High update rate:} Android applications have a high update rate \cite{3} because of bug fixes and the frequent addition of new features based on the changing business requirement and user needs. Thus, Android developers need to keep their applications maintainable. Because it gets harder to maintain Android applications as the codebase grows or changes \cite{34}. For that reason, Android applications should be developed in a way that modifications for new features and bug fixing can be done smoothly.

\noindent\textbf{Changing Development Teams:}  Developers of an Android application change during the life cycle of the application. Whenever a new developer joins the team, the time required to onboard a new developer to the codebase is directly related to how the Android applications are developed. Therefore, Android applications must be implemented in an orderly fashion that enables developers to quickly read and understand the app's purpose.

The team aims to develop maintainable Android applications to solve these problems and have internalized some methods and technologies for this purpose, such as coding conventions/principles (e.g. SOLID, Clean Code, Clean Architecture) and using some third-party libraries (RxJava, Dagger 2, etc.). Although these methods and technologies are widely known by the Android community and are believed to be useful for maintainability, their impact on maintainability is empirically unknown for the case company, Mooncascade. Thus, to find a suitable method for maintainability evaluation and then evaluate the impact of these tools, techniques, and technologies used by Mooncascade’s Android team in terms of maintainability, this study will answer the following research questions. 

\begin{itemize}
\item \noindent\textbf{RQ1:} What is the fitting way to evaluate the maintainability of the Android applications developed by Mooncascade?
\item \noindent\textbf{RQ2:} What is the impact of the methods and technologies used by Mooncascade to develop Android applications on the maintainability of Android applications?
\end{itemize}

To answer the first research question, research is conducted to find a suitable method for evaluating the maintainability of the Android applications. During this research, previously conducted studies in software maintainability are examined, and efforts are made to find the most suitable method for this case study. The study aimed to answer the second research question using the maintainability evaluation methods obtained from answering the first research question. For this purpose, the obtained methods are applied to two different codebases, one developed by the methods used by the case company and the other not developed by following a certain fashion.

\subsection{Scope and Goal}
This study addresses the practices used by Mooncascade’s Android team to resolve the challenges mentioned in the previous section. To this end, the study aims to present comprehensive and up-to-date resources used by Mooncascade's Android team, including the tools, libraries, and techniques and how those are used to achieve the goal. The study will identify, understand, and share the practices and technologies used by Mooncascade's Android team. Moreover, it will present the determined practices in the forms of code samples and instructions. In this way, the study aims to facilitate resolving the challenges faced when developing state-of-the-art Android applications, which are also mentioned in the introduction section, by providing advanced techniques for developers and researchers.  

Moreover,  as a part of this study, interviews will be conducted amongst the Android developers of Mooncascade's Android team to evaluate the impact of the researched practices and technologies. Researched methods will also be compared to the data collected through an Android developer survey conducted amongst the Android community to support the validity and up-to-dateness. Lastly, the identified and studied practices will be evaluated from the maintainability point of view by using software quality metrics.

\subsection{Contributions}
This study's main contribution is providing comprehensive information regarding the development of large scale and enterprise Android applications through the experience of a proficient software development company, Mooncascade. Since the study topic is highly coupled to industry and industry trends, the author contributes to identifying and understanding the practices used by one of the top software development companies in the region and their impact and bringing those practices into the academy through this study. With the help of the information that this study offers, developers and researchers will know the fundamental principles and technologies required to develop state-of-the-art Android apps that have increased maintainability and quality. Thus solving the difficulties explained in the introduction section can be facilitated.

Examining white and gray literature has shown that there is no similar case study regarding Android application development. Still, some studies focus on maintainability and the quality of Android applications and solutions for challenges mentioned earlier. However, the examination has also shown that such studies lack detail and up-to-dateness. This study's valuable set of information might help fill the gap of outdatedness and lack of detail between the industry and academia in developing Android applications. Unlike most similar studies, in this study, the investigated principles and technologies will be gathered from the industry's real-life Android application development best practices. The study will include detailed information about the implementation of these principles and technologies. Apart from the main contribution, the evaluation results of presented methodologies and technologies will provide empirical data, both from the maintainability point of view and the Android developer's perspective. Lastly, the Android developer survey conducted as a part of this study can provide insights from the latest technology trends in the Android community.

\subsection{Thesis Outline}
The rest of this study is structured as follows:

\noindent\textbf{2. Background:}
This chapter starts with brief information about the Android platform and Android SDK. Later, the chapter continues with key concepts for the maintainability of software systems. Lastly, the methods and technologies used by Mooncascade's Android team to tackle the problems mentioned in the first chapter are presented.

\noindent\textbf{3. Research Methodology}

In this section, the methods used to achieve the primary goal of this study are presented. The contents of the survey and interviews are discussed in detail. Also, metrics used while evaluating methods and technologies used by Mooncascade's Android team in terms of maintainability are explained. Thus, the first research question will be answered in this chapter.

\noindent\textbf{4. Evaluation}

The results for the impact of the practices that are used by Mooncascade's Android team on the maintainability of Android applications are shared in this chapter. The results obtained from the Android developer survey, interviews with the case company's Android developers and evaluation with the object-oriented metrics are presented, respectively.

\noindent\textbf{5. Discussion}

This chapter will present the discussion and interpretation of answers to the research questions and evaluation results obtained in the previous chapters. Outcomes of the evaluations will be shared. Also, the limitations of the study will be mentioned.

\noindent\textbf{6. Conclusion}

The study's epitome and results, along with the final thoughts and comments, will be presented in this section. Also, future research opportunities will be discussed.
