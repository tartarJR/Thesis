In the last decade, the impact of smartphones on our lives has significantly increased, and smartphones and mobile applications became people's primary way of interacting with technology. This situation has made the applications that work on these smartphones a vital part of daily and business life, from ordinary people to large companies. Today, mobile applications have become one of the most critical parts of digitalisation. Notably, as a successful open-source mobile operating system, Android has been a core element of this change, and the demand for Android applications has increased. Android application development has become one of the most necessary parts of the business area with a significant market share. Today, there are more than 2.5 billion active Android devices in the world \cite{1}. Therefore, it is difficult to ignore the importance of Android application development.
However, the increasing importance of the mobile era also brought more challenges to mobile application development and Android application development.

Mooncascade is a  product development company based in Estonia. The company provides different software development services to its clients, including Android development. With the increasing demand for mobile applications, the company has been facing various challenges when providing Android application development services to different customers from different fields during the development process of Android applications. These difficulties can be examined under four major topics. Android's platform-specific complexity, sophisticated business needs, the frequent update rate of Android applications, and, lastly, growing codebases and fast-changing development teams. To overcome these challenges, "maintainability" emerges as one of the most important non-functional requirements for this purpose. Developing maintainable Android applications is the way to survive in the competitive market quickly and cost-effectively. 

This study aims to evaluate the effectiveness of the Android development methods and technologies used by Mooncascade in terms of maintainability. To that aim, both qualitative analyses (via interviews and surveys) and quantitative analysis (via object-oriented software quality metrics) are conducted. The main motivation for this study is to eliminate the problems arising from the lack of maintainability in Android application development and, thus, improve developer/team efficiency and provide time and cost-efficient solutions to the clients. The target audience of this study includes Android developers and researchers who are already experienced with Android application development basics. The study only focuses on native Android application development.
 
The author contributed to understanding how Android application codebases can be measured in terms of maintainability and providing empirical data on the impact of the well known Android development methods and technologies on maintainability. 

\subsection{Problem Statement}
\label{section:1.1}
Mooncascade is a leading product development company with global reach, based in Estonia. Mooncascade aims to provide quality at every stage of the software product development process, helping clients build innovative solutions that inspire, disrupt, and challenge markets worldwide. The company provides various software development services, including web development, data science, quality assurance, and mobile application development. 

The Android development team is one of the most active units of Mooncascade. The team has been working with clients from different industry domains and developing countless Android applications for clients. From the client's point of view, Mooncascade's Android team's biggest goal is developing Android apps with high quality. High quality means, following the latest industry trends and technologies, improving and providing a pleasant mobile application experience for customers and end-users. From the software engineering perspective, essential goals are enhancing developer and team efficiency and increasing code quality and maintainability. These are the goals that will constitute this study's main subject since they are directly related to the four significant challenges encountered when developing Android applications, which are mentioned in the introduction section. To have a better understanding of those challenges, it is wise to explain them in a bit more detail. 

\noindent\textbf{Android platform-specific complexity:} Android applications are distinguished from traditional web and desktop software with their sophisticated features and specific structure. They need to fulfill some platform-specific requirements. Apps should work per Android OS, and they have to use components offered by the Android Software Development Kit, such as Activity, Fragment, Service. A typical Android app consists of components such as views, activities, fragments, services, broadcast receivers, and content providers. These components are unique to the Android platform and provided by the Android SDK. The Android system mostly controls the behaviors of these components. Android developers must obey the contract supplied by the system when using these components, and this situation sometimes limits developers to use some specific programming techniques when developing Android applications. These components are directly related to the Android OS, and they should be separated from the other possible layers to decrease the level of complexity and increase maintainability. Nevertheless, that is not always the case, and sometimes this separation is not as easy as it sounds. Besides, there are other components used in an Android app that are not a part of the Android SDK for managing network operations, database transactions, and business rules. Whether it is a part of the Android SDK or not, each of these components represents a different "concern" of an Android application. The necessity of working together in harmony for all these components, the challenges that arise from the Android operating system and Android SDK's nature, and the limited resources of mobile devices make Android applications complicated software systems that are hard to develop. 

\noindent\textbf{Business-specific complexity:} Android applications get more and more sophisticated to fulfill increasing user needs and business requirements. Mobile applications become more functional as user and business needs increase. Consequently, the complexity of Android apps from the software development point of view increases, and apps become more business-critical \cite{2}. When this business-specific complexity comes together with the platform-specific complexity mentioned above, it makes developers’ jobs even harder as its influence on development is strong. In this regard, it makes sense to separate the business-specific requirements and logic from the rest of the application to improve the maintainability of the application and ease complexity related to business-specific needs.

\noindent\textbf{High update rate:} Android applications have a high update rate because of bug fixes and the frequent addition of new features based on the changing business requirement and user needs. That makes the software development life cycles of the Android applications quite active \cite{3}. For that reason, Android applications should be developed so that the addition of new features and fixing bugs can be done smoothly. Thus, problems can be solved in a time and cost-efficient manner. In that way, decreasing maintenance costs, shortening release times, and improving developer efficiency while developing Android applications could be possible.

\noindent\textbf{Growing Codebases and Fast-changing Development Teams:} Android applications' gets harder to maintain as the codebase and the development team grows or changes. The codebase of an Android application has to be in an orderly fashion that enables developers to read and understand the app's purpose quickly. Also, any time a new developer joins the team, the time required to onboard a new developer to the codebase is directly related to the way that the Android applications were developed.

Over the last decade, a couple of different ideas have been in place to resolve these issues in the context of Android application development, and similar views with the same purpose are continually evolving. However, the primary purpose of all these ideas is the same. All these ideas and methodologies, whether in the form of software architecture, design patterns, or coding conventions, aim to improve the "maintainability" of the Android applications.

In the context of Mooncascade's Android team, from the maintainability point of view, the challenges explained above become even more critical since the company provides services in the forms of software development and consultancy, and the reason for this is how Mooncascade works. In other means, the team is divided into sub-teams, and these sub-teams work on different projects for different clients from different domains. Over time members of the sub-teams can change, a project can be extended, or maintenance might be needed. In this situation, the extension or maintenance of a project might need to be done with different developers. That is the point where maintainability emerges as a critical non-functional requirement. High maintainability means better code readability and understandability, less onboarding time for a new developer, easily extendable and changeable code. If projects are developed with high maintainability, such a process can be managed more time and cost-efficiently.

The team had internalized some set of tools, techniques, and technologies over time to that aim. Although these tools, techniques, and technologies used by Mooncascade's Android team are used by the Android community worldwide, the impact and benefits are empirically unknown from the software engineering perspective. Knowing the benefits and effects of these tools, techniques, and technologies used by the team in software maintainability is essential to understand how useful and practical they are. Thus, to identify what these tools, techniques, and technologies used by Mooncascade's Android team are and what benefits and impacts they bring, this study will answer the following research questions. 

\begin{itemize}
\item \noindent\textbf{RQ1:} What are the metrics and/or methods for measuring maintainability in the context of Android application development?
\item \noindent\textbf{RQ2:} What are the methods, techniques, tools, and technologies used by Mooncascade's Android team to develop quality Android applications?
\item \noindent\textbf{RQ3:} How efficient and impactful are the methods, techniques, tools, and technologies used by Mooncascade's Android when developing Android applications, in terms of increasing software maintainability?
\end{itemize}