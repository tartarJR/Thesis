In the last decade, the impact of smartphones on our lives has significantly increased, and smartphones and mobile applications became people's primary way of interacting with technology. This situation has made the applications that work on these smartphones a vital part of daily and business life, from ordinary people to large companies. Today, mobile applications have become one of the most critical parts of digitalisation. Notably, as a successful open-source mobile operating system, Android has been a core element of this change, and the demand for Android applications has increased. Android application development has become one of the most necessary parts of the business area with a significant market share. Today, there are more than 2.5 billion active Android devices in the world \cite{1}. Therefore, it is difficult to ignore the importance of Android application development.
However, the increasing importance of the mobile era also brought more challenges to mobile application development and Android application development. Designing, developing, testing and understanding Android applications is becoming more complex \cite{34}. These challenges can be examined under four major topics. These are Android’s nature and its platform-specific components, sophisticated business needs, the frequent update rate of Android applications, and lastly, changing development teams. 

Mooncascade is a  product development company based in Estonia. The company provides different software development services to its clients, including Android development. With the increasing demand for Android applications, the company has been facing the previously mentioned challenges when providing Android application development services to different customers from different fields during the development process of Android applications. To overcome these challenges, "maintainability" emerges as one of the most important non-functional requirements. In the context of software engineering, "maintainability" is the ease with which a system or component can be modified for use in applications or environments other than those for which it was specifically designed \cite{20}. Developing maintainable Android applications is the way to survive in the competitive market time and cost-effectively \cite{50}. However, achieving this goal is challenging and costly \cite{34}.

This study aims to evaluate the impact of the methods and technologies used by Mooncascade when developing Android applications in terms of maintainability. Although evaluating software maintainability is not hard, the ways to be used are vast and choosing a suitable method might be challenging. Different methods and software metrics can be used to estimate the software maintainability value \cite{34,23,26,33,36,45}. For that purpose, this study follows the triangulation strategy \cite{51}. Both qualitative analyses (via interviews and surveys) and quantitative analysis (via object-oriented software quality metrics) are conducted. This method is preferred to collect data from different resources (academic resources and Android community) for obtaining more coherent results. The main motivation of this study is to find proper methods and technologies for developing maintainable applications to eliminate the four major problems in Android application development mentioned above. Thus improving developer/team efficiency and providing time and cost-efficient solutions to the clients. The study only focuses on native Android application development.

\subsection{Problem Statement}
\label{section:1.1}
Mooncascade is a leading product development company with global reach, based in Estonia. Mooncascade aims to provide quality at every stage of the software product development process, helping clients build innovative solutions that inspire, disrupt, and challenge markets worldwide. The company provides various software development services, including web development, data science, quality assurance, and mobile application development. 

The Android development team is one of the most active units of Mooncascade. The team has been working with clients from different industry domains and developing countless Android applications for clients. From the client's point of view, Mooncascade's Android team's biggest goal is developing Android apps with high quality. High quality means, following the latest industry trends and technologies, improving and providing a pleasant mobile application experience for customers and end-users. From the software engineering perspective, essential goals are enhancing developer and team efficiency and increasing code quality and maintainability. These are the goals that will constitute this study's main subject since they are directly related to the four significant challenges encountered when developing Android applications, which are mentioned in the introduction section. To have a better understanding of those challenges, it is wise to explain them in a bit more detail. 

\noindent\textbf{Android platform-specific complexity:} Android applications are distinguished from traditional web and desktop software with their sophisticated features and specific structure. They need to fulfill some platform-specific requirements. Apps should work per Android OS, and they have to use components offered by the Android Software Development Kit, such as Activity, Fragment, Service. A typical Android app consists of components such as views, activities, fragments, services, broadcast receivers, and content providers. These components are unique to the Android platform and provided by the Android SDK. The Android system mostly controls the behaviors of these components. Android developers must obey the contract supplied by the system when using these components, and this situation sometimes limits developers to use some specific programming techniques when developing Android applications. These components are directly related to the Android OS, and they should be separated from the other possible layers to decrease the level of complexity and increase maintainability. Nevertheless, that is not always the case, and sometimes this separation is not as easy as it sounds. Besides, there are other components used in an Android app that are not a part of the Android SDK for managing network operations, database transactions, and business rules. Whether it is a part of the Android SDK or not, each of these components represents a different "concern" of an Android application. The necessity of working together in harmony for all these components, the challenges that arise from the Android operating system and Android SDK's nature, and the limited resources of mobile devices make Android applications complicated software systems that are hard to develop. 

\noindent\textbf{Business-specific complexity:} Android applications get more and more sophisticated to fulfill increasing user needs and business requirements. Mobile applications become more functional as user and business needs increase. Consequently, the complexity of Android apps from the software development point of view increases, and apps become more business-critical \cite{2}. When this business-specific complexity comes together with the platform-specific complexity mentioned above, it makes developers’ jobs even harder as its influence on development is strong. In this regard, it makes sense to separate the business-specific requirements and logic from the rest of the application to improve the maintainability of the application and ease complexity related to business-specific needs.

\noindent\textbf{High update rate:} Android applications have a high update rate because of bug fixes and the frequent addition of new features based on the changing business requirement and user needs. That makes the software development life cycles of the Android applications quite active \cite{3}. For that reason, Android applications should be developed so that the addition of new features and fixing bugs can be done smoothly. Thus, problems can be solved in a time and cost-efficient manner. In that way, decreasing maintenance costs, shortening release times, and improving developer efficiency while developing Android applications could be possible.

\noindent\textbf{Growing Codebases and Fast-changing Development Teams:} Android applications' gets harder to maintain as the codebase and the development team grows or changes. The codebase of an Android application has to be in an orderly fashion that enables developers to read and understand the app's purpose quickly. Also, any time a new developer joins the team, the time required to onboard a new developer to the codebase is directly related to the way that the Android applications were developed.

Over the last decade, a couple of different ideas have been in place to resolve these issues in the context of Android application development, and similar views with the same purpose are continually evolving. However, the primary purpose of all these ideas is the same. All these ideas and methodologies, whether in the form of software architecture, design patterns, or coding conventions, aim to improve the "maintainability" of the Android applications.

In the context of Mooncascade's Android team, from the maintainability point of view, the challenges explained above become even more critical since the company provides services in the forms of software development and consultancy, and the reason for this is how Mooncascade works. In other means, the team is divided into sub-teams, and these sub-teams work on different projects for different clients from different domains. Over time members of the sub-teams can change, a project can be extended, or maintenance might be needed. In this situation, the extension or maintenance of a project might need to be done with different developers. That is the point where maintainability emerges as a critical non-functional requirement. High maintainability means better code readability and understandability, less onboarding time for a new developer, easily extendable and changeable code. If projects are developed with high maintainability, such a process can be managed more time and cost-efficiently.

The team had internalized some set of tools, techniques, and technologies over time to that aim. Although these tools, techniques, and technologies used by Mooncascade's Android team are used by the Android community worldwide, the impact and benefits are empirically unknown from the software engineering perspective. Knowing the benefits and effects of these tools, techniques, and technologies used by the team in software maintainability is essential to understand how useful and practical they are. Thus, to identify what these tools, techniques, and technologies used by Mooncascade's Android team are and what benefits and impacts they bring, this study will answer the following research questions. 

\begin{itemize}
\item \noindent\textbf{RQ1:} What are the metrics and/or methods for measuring maintainability in the context of Android application development?
\item \noindent\textbf{RQ2:} What are the methods, techniques, tools, and technologies used by Mooncascade's Android team to develop quality Android applications?
\item \noindent\textbf{RQ3:} How efficient and impactful are the methods, techniques, tools, and technologies used by Mooncascade's Android when developing Android applications, in terms of increasing software maintainability?
\end{itemize}

\subsection{Contribution}
%different people interpret maintainability in different ways so evaluation methods differ
A literature review was carried out within the scope of this study to examine previous studies conducted in the academy on the maintainability of Android applications and obtain more comprehensive academic information on maintainability measurement. Considering the tight relationship between the topic and industry, a Systematic Literature Review (SLR) was not adequate for finding relevant resources of data for the study. Hence, a Multivocal Literature Review (MLR) was conducted. As a type of SLR, MLR is collecting grey literature as well alongside formal literature \cite{40}. MLR considers resources like blogs, white papers, articles, academic literature and allows gathering information from academics, developers, practitioners, and independent researchers \cite{41}. The search query was separated into three parts to obtaining more successful findings, individually focusing on a single topic. These parts are Android, maintainability and methods/technologies, respectively. Fig. \ref{fig:lit_review_research_query} presents the visualization of the research query.
\begin{figure}[ht!] 
    \centering
    \includegraphics[scale=0.5]{figures/research_query.png}
    \caption{Literature review visualization}
    \label{fig:lit_review_research_query}
\end{figure}
\FloatBarrier

A set of criteria has been determined to increase the accuracy of the results obtained from the research query and to filter out possible irrelevant studies among the results. While determining the criteria, issues such as the language of the reviewed studies, their up-to-dateness and their relevance to native Android development were taken into consideration. The inclusion and exclusion criteria are shown in Fig. \ref{fig:lit_review_research_query_criteria}.

\begin{figure}[ht!]
    \centering
    \includegraphics[scale=0.5]{figures/research_query_criteria.png}
    \caption{Inclusion and exclusion criteria}
    \label{fig:lit_review_research_query_criteria}
\end{figure}
\FloatBarrier

More than fifty studies were found and reviewed as a result of this query. Grey literature results are not included in these numbers (e.g. blog posts, official documentation, books). However, the reviewed grey literature was also cited in this study. Moreover, twenty-six studies amongst the literature review results were review in detail since they are closely related to this study's field of work. These studies can be grouped according to their topics as follows.
\begin{itemize}
    \item 9 papers related to Android app maintainability \cite{34,50,52,2,18,19,43,44,53}.
    \item 4 papers related to Android app architecture \cite{56,14,47,48}.
    \item 10 papers related to software maintainability \cite{23,26,33,36,45,4,22,35,46,49}.
    \item 3 papers related to software architecture \cite{25,27,28,}.
\end{itemize}

In addition to the numbers mentioned above, several other studies were reviewed to better understand the studies conducted regarding software maintainability, metrics that can be used to measure software maintainability, general software engineering principles such as separation of concerns, and SOLID.

Based on these numbers, it can be said that the importance of maintainability for software systems and Android applications is recognized by the researchers. For example, Ivanov et al. (2020) emphasize the importance of maintainability for Android applications, and they investigate the evolution of various maintainability issues along the lifetime of Android apps. Their results uncover the frequency and evolution of maintainability issues of Android apps and point to the frequent maintainability issues for Android applications \cite{53}. The literature review has shown that there are many different ways to evaluate maintainability. Different studies have approached the maintainability of Android applications and software systems by considering different criteria. This situation caused the methods used to measure maintainability to differ in these studies. There are a few significant studies conducted on this topic. Hugo Källstrom (2016) conducted a study on the maintainability of Android applications. He implemented three different design patterns (MVC, MVP, MVVM) and evaluated the maintainability of these applications by comparing their performance, modifiability, and testability levels. Results of this study do not provide much insight into the impact of the used design patterns on maintainability \cite{18}. Also, Prabowo et al. (2018) researched the impact of the MVP and anti-pattern approaches on the maintainability of Android applications. They compared maintainability between two applications built with design pattern (MVP) and without design pattern using maintainability metrics such as LoC, Halstead Effort, and Cyclomatic Complexity. Their results showed that usage of MVP increases the maintainability level of the Android applications \cite{19}. Saifan and Al-Rabad (2017) present a different perspective on measuring the maintainability of Android applications by focusing on analyzing a set of Object-Oriented metrics and Android Metrics. Their research reveals that maintainability can be measured in different ways, and the metrics to be used depends on the source code.  They also mentioned that Android applications have special features that distinguished them, and to measure the maintainability of Android applications new method is needed \cite{34}. The study of Andrä et al. (2020) draws attention to the most recent research on this subject. They use several different maintainability metrics and match these metrics with different "Clean Code" conventions to evaluate the maintainability of the Android applications developed with Kotlin in class and method level. They tried different static code analysis tools for this purpose. Their results showed that there is currently no suitable tool or plugin for calculating maintainability metrics for Kotlin applications, which are defined by the clean code conventions \cite{43}. Panca et al. (2016) evaluate the impact of design pattern selection on the maintainability of Android applications. For that purpose, they implemented different applications using different design patterns such as singleton, memento, state, iterator, factory, builder, and flyweight. They used maintainability metrics such as LoC, Halstead Effort, and Cyclomatic Complexity for maintainability evaluation and compared the maintainability values of different applications. Their results showed that the maintainability metrics value is increased compared to before using anti-pattern \cite{52}. In addition, the research of Verdecchia et al. (2019) on software architecture choices in Android applications provides insights into the relationship between software architecture, and maintainability \cite{14}. On the other hand, most studies approach the maintainability of Android applications from a software architecture/design pattern perspective only. Other techniques and technologies can also be used to increase the maintainability of Android applications besides architectural patterns. The fact that the studies do not focus on these techniques and technologies can be considered a limitation of the academic literature regarding this topic. Apart from this, it is seen that similar metrics and methods are used in many studies in this field. Considering the differences of Android applications compared to other software, it can be mentioned that different methods should be used. In addition, it has not been possible to come across studies involving relatively new Android technologies. In some studies, the difficulties of evaluating Android applications developed with a relatively new programming language such as Kotlin in terms of maintainability are also addressed.

\newpage
\subsection{Thesis Outline}
The rest of this document is organized as follows. In Chapter 2, brief information about the Android platform and Android SDK is given. The chapter continues with key concepts for the maintainability of software systems. Lastly, the methods and technologies used by Mooncascade's Android team to tackle the problems mentioned in the first chapter are presented. In Chapter 3, why and how the literature review was carried out within the scope of this study and the results of this literature review is shared. Also, a brief analysis of the results is presented. In Chapter 4, the methods used to achieve the primary goal of this study are presented. The contents of the survey and interviews are discussed in detail. Also, metrics used while evaluating methods and technologies used by Mooncascade's Android team in terms of maintainability are explained. Thus, the first research question will be answered in this chapter. In Chapter 5, the results for the impact of the practices that are used by Mooncascade's Android team on the maintainability of Android applications are shared. The results obtained from the Android developer survey, interviews with the case company's Android developers and evaluation with the object-oriented metrics are presented as well. Chapter 6 presents the discussion and interpretation of answers to the research questions and evaluation results obtained in the previous chapters. Outcomes of the evaluations will be shared. Also, the limitations of the study will be mentioned. The last chapter concludes this thesis, and future research opportunities are discussed.

