In the last decade, the impact of smartphones on our lives has significantly increased, and smartphones and mobile applications became people's primary way of interacting with technology. This situation has made the applications that work on these smartphones a vital part of daily and business life, from ordinary people to large companies. Today, mobile applications have become one of the most critical parts of digitalisation. Notably, as a successful open-source mobile operating system, Android has been a core element of this change, and the demand for Android applications has increased. Android application development has become one of the most necessary parts of the business area with a significant market share. Today, there are more than 2.5 billion active Android devices in the world \cite{1} . Therefore, it is difficult to ignore the importance of Android application development.
However, the increasing importance of the mobile era also brought more challenges to mobile application development and, of course, to Android application development.

Mooncascade is a  product development company based in Estonia. The company provides different software development services to its clients, including Android development. With the increasing demand for the mobile applications, the company has been facing various challenges when providing Android application development services to different customers from different fields during the development process of Android applications. These difficulties can be examined under four major topics. Android's platform-specific complexity, sophisticated business needs, the frequent update rate of Android applications, and, lastly, growing codebases and fast-changing development teams. To overcome these challenges, "maintainability" emerges as one of the most important non-functional requirements for this purpose. Developing maintainable Android applications is the way to survive in the competitive market quickly and cost-effectively. 

This study aims to evaluate the effectiveness of the Android development methods and technologies used by Mooncascade in terms of maintainability. To that aim, both qualitative analyses (via interviews and surveys) and quantitative analysis (via object-oriented software quality metrics) are conducted. The main motivation for this study is to eliminate the problems arising from the lack of maintainability in Android application development and, thus, improve developer/team efficiency and provide time and cost efficient solutions to the clients. The target audience of this study includes Android developers and researchers who are already experienced with Android application development basics. The study only focuses on native Android application development.
 
The author contributed in understanding how Android application codebases can be measured in terms of maintainability and providing emprical data on impact of the well know Android development methods and technologies on maintainability. 

\subsection{Problem Statement}
\label{section:1.1}
Mooncascade provides software development services, including Android application development. Demand for mobile applications in the industry has been increasing in the last decade. The company has been having an increasing number of requests for Android applications and facing different challenges during the development and maintenance phases of these Android applications. These challenges are detailed as follows:

\noindent\textbf{Android platform complexity:} Android SDK is an over-engineered framework that imposes inherent complexity on the developers  \cite{56}. It is different from other software due to the application life cycle, methods for accessing resources between objects, multi-threading concept, and the layout creation process \cite{52}. An Android app consists of activities, fragments, services, broadcast receivers, and content providers provided by the Android SDK and controlled by the Android OS. The necessity of working together in harmony for all these components make Android applications complex software systems.

\noindent\textbf{Business-specific complexity:} Android applications get more and more sophisticated to fulfil increasing user needs and business requirements. Mobile applications become more functional as user and business needs increase. Consequently, the complexity of Android apps from the software development point of view increases. When this business-specific complexity comes together with the platform-specific complexity mentioned above, it will be vital to implement software engineering processes to ensure the development of high-quality Android applications \cite{2}. 

\noindent\textbf{High update rate:} Android applications have a high update rate \cite{3} because of bug fixes and the frequent addition of new features based on the changing business requirement and user needs. Thus, Android developers need to keep their applications maintainable. Because it gets harder to maintain Android applications as the codebase grows or changes \cite{34}. For that reason, Android applications should be developed in a way that modifications for new features and bug fixing can be done smoothly.

\noindent\textbf{Changing Development Teams:}  Developers of an Android application change during the life cycle of the application. Whenever a new developer joins the team, the time required to onboard a new developer to the codebase is directly related to how the Android applications are developed. Therefore, Android applications must be implemented in an orderly fashion that enables developers to quickly read and understand the app's purpose.

The team aims to develop maintainable Android applications to solve these problems and have internalized some methods and technologies for this purpose, such as coding conventions/principles (e.g. SOLID, Clean Code, Clean Architecture) and using some third-party libraries (RxJava, Dagger 2, etc.). Although these methods and technologies are widely known by the Android community and are believed to be useful for maintainability, their impact on maintainability is empirically unknown for the case company, Mooncascade. Thus, to find a suitable method for maintainability evaluation and then evaluate the impact of these tools, techniques, and technologies used by Mooncascade’s Android team in terms of maintainability, this study will answer the following research questions. 

\begin{itemize}
\item \noindent\textbf{RQ1:} What is the fitting way to evaluate the maintainability of the Android applications developed by Mooncascade?
\item \noindent\textbf{RQ2:} What is the impact of the methods and technologies used by Mooncascade to develop Android applications on the maintainability of Android applications?
\end{itemize}

To answer the first research question, research is conducted to find a suitable method for evaluating the maintainability of the Android applications. During this research, previously conducted studies in software maintainability are examined, and efforts are made to find the most suitable method for this case study. The study aimed to answer the second research question using the maintainability evaluation methods obtained from answering the first research question. For this purpose, the obtained methods are applied to two different codebases, one developed by the methods used by the case company and the other not developed by following a certain fashion.