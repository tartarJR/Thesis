When other academic studies dealing with the measurement of maintainability in Android or other software systems are considered, it can be seen that many studies use scientific and quantitative methods. The work of Verdecchia et al. in the maintainability and architecture of Android applications can be shown as a successful example of this situation \cite{14}. Although it cannot be claimed that such quantitative measurements are inaccurate, it would not be wrong to say that these measures are inadequate at times. It is essential to make qualitative evaluations, and quantitative evaluations in areas where technologies are rapidly developing and trends change quickly, especially in Android application development. In this way, it may be possible to measure developers' experiences that differ in the face of rapid change and development and the effects of these experiences on the maintainability issue we are working on more efficiently. For these reasons, it was deemed appropriate to add a qualitative evaluation technique to this study’s scope. An Android developer survey and some interviews were conducted within this study's scope as a part of qualitative evaluations. The contents and purposes of these surveys and conferences are discussed in detail in their respective sub-sections below.

\subsubsection{Android Developer Survey}
The first step of qualitative evaluations in this study is the Android developer survey. The consistency and stability of principles and third-party libraries used in Android applications indirectly affect the maintainability of applications. Therefore, although it does not directly contribute to measuring maintainability, this survey was conducted to identify current Android trends and provide support for this study's evaluation. While determining this survey's questions, priority was given to principles and technologies that directly and indirectly affect maintainability. Although the questions are generally prepared to cover the methods used by the Mooncascade Android's team, it can be said that the questions reflect the Android technology stack in general. The content of the survey can be accessed publicly\footnote{\url{https://forms.gle/MoiTGMV874yzJwuv8}}. The questions are organised with the help of the Forms application provided by Google. The Android developer survey has been delivered to Android developers from different companies in different countries through accounts or groups of Android developer communities on social media platforms such as LinkedIn, Discord, and Twitter. The author of the study has also shared the survey with many of his colleagues/ex-colleagues working in different companies/countries.

\subsubsection{Interviews with Team Members}
The second step of the qualitative evaluations carried out within this study’s scope is the interviews conducted with Mooncascade's Android team members. Unlike the Android developer survey that was explained in the previous section, these interviews are designed and conducted to qualitatively evaluate the techniques, technologies, and methods used by Mooncascade's Android team in terms of maintainability. Eight questions were determined to qualitatively measure the effect of the methods used by Mooncascade's Android team on the maintainability of Android applications. Below are listed the questions asked to each member of Mooncascade's Android team during these interviews:
\begin{itemize}
    \item How many years of experience do you have as an Android developer? Please specify the years in Mooncascade and other companies.
    \item How many different Android projects have you completed in Mooncascade, and how many different domains did those projects belong to?
    \item What is your understanding of maintainability in the context of software engineering?
    \item As an employee of a software development company that provides services to the different domains, what makes maintainability more essential for you?
    \item What is the importance of maintainability when developing Android applications?
    \item What is the most critical aspect for maintainability when developing Android applications (e.g. architecture, libraries, programming language, etc.)?
    \item How do you think the current technology stack of the team impacts Android applications’ maintainability? Please specify for each item below:
    \begin{itemize}
        \item Programming Languages(Kotlin/Java)
        \item Software Engineering principles (SOLID/Clean Code)
        \item Architecture (MVVM/Clean)
        \item Libraries (RxJava, Dagger 2, Apollo/Retrofit)
        \item Android Architecture Components (ViewModel, LiveData, Room, etc.)
    \end{itemize}
    \item What could be improved in our current tech stack and the principles we apply to improve the Android applications’ maintainability?
\end{itemize}

Three main criteria were taken into consideration while preparing these questions. First of all, questions were chosen to measure the participants' background, experience, and understanding of the concept of maintainability in software engineering. Subsequently, questions were drafted about technology and principles that directly or naturally affect maintainability. Finally, questions were prepared concerning the Android developer questionnaire, which was mentioned in the previous section, to make it possible to compare the results of both the survey and the interviews in the evaluation section. These questions, which are included in the interviews, were directed to the team members in the meetings conducted privately with each team member. The collected answers and information will be interpreted in detail in section \ref{section:5}. It is aimed that the knowledge and experience gathered through these interviews will support the accuracy and validity of quantitative assessments.
