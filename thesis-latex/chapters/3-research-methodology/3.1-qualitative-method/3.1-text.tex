When other academic studies dealing with the measurement of maintainability in Android or other software systems are considered, it can be seen that many studies use scientific and quantitative methods. The work of Verdecchia et al. in the maintainability and architecture of Android applications can be shown as a successful example of this situation \cite{14}. Although it cannot be claimed that such quantitative measurements are wrong, it would not be wrong to say that these measures are inadequate at times. It is essential to make qualitative evaluations, and quantitative evaluations in areas where technologies are rapidly developing and trends change quickly, especially in Android application development. In this way, it may be possible to measure developers' experiences that differ in the face of rapid change and development and the effects of these experiences on the maintainability issue we are working on more efficiently. For these reasons, it was deemed appropriate to add a qualitative evaluation technique to this study’s scope. An Android developer survey and some interviews were conducted within this study's scope as a part of qualitative evaluations. The contents and purposes of these surveys and conferences are discussed in detail in their respective sub-sections below.

\subsubsection{Android Developer Survey}
The first step of qualitative evaluations in this study is the Android developer survey. The consistency and stability of principles and third-party libraries used in Android applications indirectly affect the maintainability of applications. Therefore, although it does not directly contribute to measuring maintainability, this survey was conducted to identify current Android trends and provide support for this study's evaluation. While determining this survey's questions, priority was given to principles and technologies that directly and indirectly affect maintainability. Although the questions are generally prepared to cover the methods used by the Mooncascade Android's team, it can be said that the questions reflect the Android technology stack in general. The content of the survey can be accessed publicly\footnote{\url{https://forms.gle/MoiTGMV874yzJwuv8}}. The questions are organised with the help of the Forms application provided by Google. The Android developer survey has been delivered to Android developers from different companies in different countries through accounts or groups of Android developer communities on social media platforms such as LinkedIn, Discord, and Twitter. The author of the study has also shared the survey with many of his colleagues/ex-colleagues working in different companies/countries.

\subsubsection{Interviews with Team Members}
The second step of the qualitative evaluations carried out within this study’s scope is the interviews conducted with Mooncascade's Android team members. Unlike the Android developer survey that was explained in the previous section, these interviews are designed and conducted to qualitatively evaluate the techniques and technologies used by Mooncascade's Android team in terms of maintainability. Also, the interviews aim to determine the importance of maintainability from the case company's point of view.  Thus, proving or disproving the study's claim that maintainability is a key concept to overcome the issues mentioned in the problem statement section would be possible. Eight questions were determined for this purpose. The interview questions can be accessed publicly\footnote{\url{https://forms.gle/paMSj5e8Up5ZW6raA}}. Three main criteria were taken into consideration while preparing these questions. First of all, questions were chosen to get to know about the participants' background and experience. Later, some questions were designed to learn participants' understanding of maintainability in software engineering. Lastly, questions were drafted to learn about participants' thoughts about the impact of technologies and principles used by Mooncascade on maintainability. The interview was conducted privately with each team member.  It is aimed that the data gathered through these interviews will support the accuracy and validity of evaluations.
