The first step of qualitative evaluations in this study is the Android developer questionnaire. Although it is not a method that provides direct benefit in measuring and evaluating maintainability, this method was used to solve the issue of not being up-to-date for the Android field's academic resources, which was criticised in the previous sections. As stated in the literature review subtitle of the previous section, up-to-dateness among existing academic studies stands out as an essential deficiency. For this reason, before evaluating the methods, technologies, and techniques used by Mooncascade’s Android team in terms of maintainability, it is considered appropriate to compare the up-to-dateness of these methods, techniques, and technologies concerning current industry trends. To achieve this goal, the Android developer survey mentioned above was used. In addition to evaluating the up-to-dateness of Mooncascade's Android team’s methods, this Android developer survey also has the purpose of informing readers about the current Android technology stack. The information collected through this questionnaire can provide an up-to-date resource for researchers who want to work in Android application development. Although this is not one of the main objectives of this study, it is thought that it would be appropriate to have such an addition, given the rapid changes in this study's primary subject and the inadequacy of the current academic literature in this field. Besides, while determining this survey's questions, priority was given to principles and technologies that directly and indirectly affect maintainability, which is the main focus of this study. The questions asked in this survey are listed below. Although the questions are generally prepared to cover the methods used by the Mooncascade Android's team, it can be said that the questions can also include the Android technology stack in a more general context. 
\begin{itemize}
    \item How many years of professional experience do you have as an Android developer?
    \item Which programming language do you use for Android application development?
    \item What presentational design pattern do you apply to your Android apps?
    \item Do you apply Uncle Bob's Clean Architecture to your Android applications?
    \item Do you follow SOLID principles while developing Android applications?
    \item Do you follow Uncle Bob's Clean Code principles while developing Android applications?
    \item What networking library do you use?
    \item How do you handle asynchronous events in your Android applications?
    \item What dependency injection library do you use in your Android applications?
    \item Do you use Android Architecture Components in your Android applications? (LiveData, ViewModel, Room, etc.)
\end{itemize}

The questions are organised with the help of the Forms application provided by Google Docs. Multiple selections and multiple-choice options have been added so that Android developers can answer them efficiently and in a standard way. The full version of the questions and answers within the Android developer survey’s scope, prepared through Google Forms, can be accessed via this \href{https://forms.gle/hasQ8FNsMYtz1d1K9}{web address} . Social media has been used to deliver the Android developer survey to as many Android application developers as possible, working in different companies and domains. The Android developer survey has been delivered to Android developers from different companies in different countries through accounts or groups of Android developer communities on social media platforms such as LinkedIn, Discord, Twitter. The author of the study has also shared the survey with many of his colleagues/ex-colleagues working in different companies/countries. Also, the author's ex-colleagues shared the survey with their current colleagues too. To get maximum efficiency from the Android developer survey and obtain valuable and consistent answers, it is envisaged to get 150-200 answers from Android developers. The detailed interpretation of the information and answers collected through the Android developer survey will be made in section \ref{section:5}.