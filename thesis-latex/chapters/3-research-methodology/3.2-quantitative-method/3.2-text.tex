As a part of this study, many other academic works on the measurement of maintainability in Android and software engineering were examined to find the appropriate maintainability evaluation methods. The examination has shown that different methods can be used to evaluate the maintainability of software systems. Particularly for evaluating the maintainability of Android applications, different studies have been conducted by using different methods \cite{34} \cite{43}. Various metrics can be used to evaluate object-oriented software systems in terms of maintainability. Some of these metrics are detailed in Barak et al. (2012) \cite{33}. As a result of this examination, it was seen that the concepts of complexity, cohesion and coupling were emphasised in many studies, and it was concluded that the measurements made based on these concepts would be more efficient when measuring maintainability. Studies have shown that results retrieved from evaluating these concepts proved to define the level of maintainability \cite{33}. Therefore, research was done on measuring the concepts of complexity, cohesion and coupling effectively, and five metrics were selected for this purpose. While selecting these metrics, priority has been given to metrics that can handle a software system as a whole in the areas of complexity, cohesion and coupling to make more efficient evaluations. These metrics are Weighted Method Count \textbf{(WMC)}, Depth of Inheritance Tree \textbf{(DIT)}, Number of Children \textbf{(NOC)}, Coupling Between Object Classes \textbf{(CBO}), and Lack of Cohesion of Methods \textbf{(LCOM)}. Detailed information regarding these metrics can be found in various studies\cite{33,34,35,36}.

After determining the metrics to be used, the options were to apply the metrics manually or with a tool. Since manual implementation is prone to error, it has been decided to apply metrics with a reliable tool. After research, the CodeMR static code analysis tool drew attention. CodeMR is a powerful software quality tool that is integrated with IDEs and supports multiple programming languages. The tool provides a set of metrics to measure coupling, complexity, cohesion and size. These metrics are often affected by various code characteristics, making them promising for evaluating software maintainability. Besides, the tool provides a visualisation centric approach and generates detailed reports supported by different visualisation options. In this way, it facilitates understanding the results and allows understanding the bigger picture of the projects from the complexity, coupling, and cohesion point of view. Detailed information on these metrics and assessment methods can be found on the tool's website\footnote{\url{https://www.codemr.co.uk}}. The tool can also be installed as a plugin in Android Studio and is very easy to use. Also, two studies conducted using this tool were examined to gather more information regarding the tool \cite{38,39}. Lastly, the CodeMR static code analysis tool has been chosen for this study due to reasons mentioned above. After this decision, communication was established with the CodeMR team, and a free license was obtained to be used in academic studies. 

To evaluate the impact of the methods and technologies mentioned in section \ref{section:2.3} on maintainability, the metrics were applied on two different code bases of the same project. Thus, the impact of the team's current methods on maintainability could be measured by comparing the results from both versions of the same project. Although the project's full content cannot be published due to privacy principles and confidentiality reasons, more information regarding these codebases is shared in the next section.