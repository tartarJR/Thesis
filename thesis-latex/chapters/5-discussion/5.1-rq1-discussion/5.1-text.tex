First of all, the information obtained while answering the first research question guided the whole of this study. Studies conducted to answer this research question have shown that qualitative measurements are necessary as well as quantitative measurements. It has been seen that qualitative measurements can make important contributions to the results of the evaluation with the information obtained directly from the developers and support the quantitative measurement results. Therefore, qualitative assessment methods were determined and decided to be applied in the scope of this study. The fact that experienced Android developers make evaluations about the methods and technologies they use every day from the maintainability point of view and the results obtained from these evaluations can be added to this study increases the study's accuracy. From this point of view, it would not be wrong to say that the addition of qualitative methods as well as quantitative methods to studies focusing on the measurement of software development concepts such as maintainability would increase the qualification of the study.

The research conducted to answer the first research question has also shown that the maintainability of object-oriented software systems can be evaluated quantitatively by using many different metrics. In this study, an evaluation method based on the concepts of complexity, coupling and cohesion was chosen in order to measure the maintainability of Android applications and 5 metrics that can measure these concepts were determined. However, it cannot be said that these methods and metrics used in this study to measure the maintainability of software systems are the best solutions that can be used for this purpose. It was explained in the previous sections why these methods and metrics are chosen. Although these methods and metrics are sufficient for this study, it would not be wrong to say that there may be more effective quantitative measurement methods. Several different studies cover different maintainability evaluation methods for Android applications \cite{43,34}. However. the best quantitative measurement method for maintainability evaluation is controversial, and it would be appropriate to conduct comparative studies using different metrics and methods to determine the most effective solution. Nevertheless, the used methods are considered sufficient for this study, and the existence of more effective methods is beyond the scope of this research.