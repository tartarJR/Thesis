Within the scope of this study, qualitative and quantitative methods to evaluate the methods and technologies used by Mooncascade's Android team were conducted. There were some limitations during the process these evaluations. 
First of all, it should be stated that the features used in quantitative evaluations are relatively less complex features of the Android application used for the evaluation. Therefore the efficiency of evaluation and comparison was affected by this situation. Using more complex feature could provide better insight into the impact of the methods and technologies used by the case company. Unfortunately, this was not possible within the scope of this study due to a lack of features and time to develop more complex features. However, it is true that using relatively less complicated features while making the assessment slightly reduces the results' effectiveness. Still, it does not mean that these results are false or unrealistic. Also, considering that there may be deficiencies in the methods used in the evaluations, the necessity to carry out more detailed studies on how to evaluate maintainability better and how to develop Android applications in terms of maintainability is obvious. For example, there are different methods to evaluate the software system's maintainability, and different metrics can be used. Therefore choosing the most efficient metrics to measure the maintainability of software systems and Android applications is controversial. Further research should be conducted to find the most efficient quantitative evaluation method. In addition, this study provides an overview rather than measuring the impact of each technology and method used by the case company to maintainability individually. The study does not provide detailed information on how much effect which matter has on maintainability. From this point of view, this can be seen as a limitation for this study. Lastly, although very important information was collected within the scope of qualitative evaluations, the number of participants is relatively low. More accurate and more reflective results can be obtained with interviews and surveys with a higher number of participants.