SOLID stands for five principles \cite{26}. Following are the brief descriptions of each principle:
\begin{itemize}
    \item \textbf{The Single Responsibility Principle:} A class should have only one reason to chance.
    \item \textbf{The Open/Close Principle:} A module should be open for extension but closed for modification
    \item \textbf{The Liskov Substitution Principle:} Subclasses should be substitutable for their base classes.
    \item \textbf{The Interface Segregation Principles:} Many client specific interfaces are better than one general purpose interface 
    \item \textbf{The Dependency Inversion Principle:} Depend upon Abstractions. Do not depend upon concretions
\end{itemize}

The SOLID Design Principles are object-oriented design guidelines to satisfy software quality attributes such as understandability, modifiability, maintainability and testability. The steps to be taken to increase the maintainability of the software systems can be taken at the design stage. The SOLID Principles are very efficient when it comes to solving such design issues and increasing maintainability of software systems.  Not following SOLID principles may lead to serious maintainability problems in the software development lifecycle, such as tight coupling, code duplication, and bug fixing \cite{55}.