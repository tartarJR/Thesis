The steps to be taken to increase the maintainability of the software systems can be taken at the design stage. The SOLID Principles are very efficient when it comes to solving such design issues and increasing maintainability of software systems. SOLID stands for five principles, namely Single responsibilities principle, Open close principle, Liskov substitution principle, Interface segregation principle, and Dependency inversion principle. Application of SOLID principles when developing software systems facilitates improving critical factors such as maintainability, extendability, readability, and reducing code complexity and tight coupling, increasing cohesion \cite{26}. Not following SOLID principles may lead to serious maintainability problems in the software development lifecycle, such as tight coupling, code duplication, and bug fixing. Application of SOLID design principles when developing software systems helps to achieve essential quality factors such as understandability, flexibility, maintainability, and testability \cite{26}.