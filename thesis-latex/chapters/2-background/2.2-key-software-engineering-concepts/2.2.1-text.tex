\textbf{\textit{"Good programmers write code that humans can understand." - Martin Fowler}} \cite{21}

A few decades ago programmers were using low-level programming languages that are working close to computer CPUs. These programming languages were designed to be understood by computers, not humans because computers lacked proper hardware, resources, and speed. Priorities back then were different. The computer programs had to be fast and less memory consuming. However, this situation has changed. Today computers are much stronger and software systems are more complex. However this situation also brought some difficulties to software development. Especially when developing large-enterprise software products, not considering how to overcome these difficulties may cause significant failures. In this context,  this new reality brought new standards to software development. New programming paradigms were born and the priorities have altered \cite{12}.

When the priorities of modern software development are analyzed today, the maintainability of software systems emerges as one of the most critical ones \cite{53}. Maintainability is how well a software system is understandable, repairable, and extendable. In other words, it is a characteristic of software that provides insights into how easily a software system can be maintained \cite{54}. Software systems are born, they live, they change, and eventually, they die. During their lifetime, new features are added, some features are removed, bugs are fixed, and often their development team changes. Usually, there is always a time gap between these changes. Developers should be able to understand the systems easily even months after. Besides, changes to the code base should be able to be done with ease without breaking the other parts of the software system. Ignoring all these might cause companies a significant amount of time and money. Developing maintainable software systems is the way to tackle such issues \cite{50}. In his famous book "Clean Code", Robert C. Martin explains how a top-rated company in the late 80s was wiped out from the business due to the lack of maintainability and poorly managed code organization \cite{11}. When the release cycles of their product extended due to the unorganized code base of their product, they were not able to fix bugs, prevent crashes, and add new features. Eventually, they had to withdraw their product from the market and went out of business. Lousy code and, consequently, lack of maintainability was the reason for this company to go out of the business. Considering the changes in software development since the 80s, this example might sound outdated. However, this real-life incident clearly shows how vital maintainability is to software systems and what fatal consequences it can cause if ignored. 

The importance of maintainability for software systems can also easily be seen when looking at its effect on the software development lifecycle and software development costs. A study has shown that the relative expense for maintaining software and dealing with its development speaks to over 90\% of its absolute expense \cite{4}. The maintenance period for a software system starts as soon as the system is developed. Thus, maintainability becomes a vital aspect for applying new customer needs, adding/removing new features, adapting to the environmental changes \cite{23}. Reports indicate that maintenance cost is 75\% of the total project cost, and the cost for maintaining source code is ten times bigger than developing the source code \cite{22}. The importance of maintainability for software systems is obvious. Also, considering the fast software development lifecycle of mobile applications, the importance of maintainability becomes even more apparent for mobile applications. Facebook's Android application is a good example of this situation\footnote{\url{https://www.apk4fun.com/history/2430/}}. In principle, mobile apps with high maintainability are easier to publish, update and provide high-quality features with less effort. That's why maintenance is considered one of the most important activities for mobile applications \cite{53}.
