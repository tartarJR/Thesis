Software systems have many complex concerns such as persistence, real-time constraints, concurrency, visualization, location control \cite{27}. Software engineering, first of all, aims to increase software quality, lower the expenses of software production, and assist maintenance and development \cite{28}. That is where "Separation of Concerns" (SoC) emerges as a solution. In the context of software engineering, SoC is a software design principle for separating a software system into discrete modules that each module addresses a single concern. A good application of SoC to a software system provides benefits such as increasing maintainability, reducing complexity. The borders for different concerns might differ from a software system to another. Concerns depend on the requirements of a software system and the forms of decomposition and composition. 

Android applications have different concerns such as sustaining limited resource availability on mobile devices, user interface responsiveness,  interactions between application components,  network connectivity, local data storage, business domain-specific issues, frequent Android platform-level changes and so on. While dealing with these concerns, developers spend a considerable amount of time, and they are diverted from their main goal of building quality Android applications. Eliminating this complexity can be achieved through focusing on the separation of concerns and abstracting away different concerns from each other. Such a goal can be achieved by applying proper software architecture, e.g. "Clean Architecture" \cite{56}.