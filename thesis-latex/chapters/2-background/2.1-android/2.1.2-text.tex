Android applications consist of components that are essential for the building blocks of an Android app. These components are also an entry point through which the system or a user can enter your app. The components might depend on other compments\footnote{\url{https://developer.android.com/guide/components/fundamentals}}.

Java and Kotlin programming languages can be used to develop native Android applications. There are also other ways of developing Android applications, such as cross-platform solutions (e.g. Xamarin, React Native, Flutter), but this study only focuses on native Android application development. Native Android development means creating Android applications that run on Android-powered devices by using the Android Software Development Kit. Native Android applications consist of  XML resources, images, data files, classes and so on. These fundamental components can be listed as follows: 
\begin{itemize}
    \item \textbf{Activities:} In the Android environment, the term “activity” refers to the interaction entry point of Android applications for the end-user, a screen with a user interface. Activities are represented by the “Activity” class of the Android SDK. Each activity of an Android application is implemented as a subclass of the Activity class.
    \item \textbf{Services:} In the context if Android, "service" refers to an entry point for Android applications to keep the applications running in the background for any reason. A service does not have a user interface. Android services can be used to perform background operations such as network operations, content provider interaction, I/O processes, and playing music. In Android, every service must be implemented as a subclass of the Service class that the Android SDK provides.
    \item \textbf{Broadcast receivers:} This term refers to the Android system component that enables the system to distribute events that happen outside of the normal application flow to the Android applications. Broadcast receivers are also entry points to the Android applications. Broadcast receivers do not involve displaying a user interface but they have the ability to create notifications in the status bar in order to alert users. Android SDK provides the "BroadcastReceiver" class and each broadcast receiver must be implemented as a subclass of this provided class.
    \item \textbf{Content providers:} The term "content provider" refers to the component that is designed to manage a mutual application data set that can be stored via a file system or a local database. Other apps can query or modify the data if the content provider allows it through the content provider. To the system, a content provider is an entry point into an app for publishing named data items identified by a URI scheme.
\end{itemize}

Although not a main component, Fragments also has an important place in Android SDK. A fragment can be considered as a modular part of an activity. Fragments can be combined in an activity to build multi-window user interfaces. Fragments have their own lifecycle and input events. Unlike activities, fragments are reusable and reuse of a fragment in multiple activities is possible. Fragments cannot exist on their own. Every fragment is hosted by an activity. Fragments are represented by the “Fragment” class of the Android SDK. In order to create a fragment, a class must inherit the Fragment class or existing subclasses of the Fragment class.

Knowing the details of all these components (e.g. implementation details, life-cycle) plays an important role in solving the complications caused by Android components while developing Android applications. Details regarding key Android components and their responsibilities are beyond the scope of this study. For this reason, the summary information in this section has been shared to address these components to raise awareness about their impact on the development of Android applications. Also, examining the official Android documentation in this area can facilitate understanding their nature and their impact on the study topic.