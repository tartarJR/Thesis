Since the launch of the first mobile device that works with Android,  the Android operating system has improved, and the way the Android applications are developed changed a lot, but some fundamental components for developing Android applications have more or less stayed the same. Each of these fundamental components was developed for a specific purpose by the creators of the Android Software Development Kit (Android SDK). Knowing these components and their responsibilities are necessary to understand the problem that this study is trying to solve.

Java and Kotlin programming languages can be used to develop native Android applications. There are also other ways of developing Android applications, but this study only focuses on native Android application development. Native Android development means creating Android applications that run on Android-powered devices by using the Android Software Development Kit. Native Android applications consist of  XML resources, images, data files, classes and so on. To understand the problem that this study tries to resolve, it is essential to understand the fundamental Android components. It is the first step for solving the maintainability issues of Android applications. These fundamental components are Activities, Services, Broadcast receivers and Content providers. In addition to these main components, the Fragment component also has an important place among Android components. Knowing the details of all these components (e.g. implementation details, life-cycle) plays an important role in solving the complications caused by Android components while developing Android applications. Details regarding key Android components and their responsibilities are beyond the scope of this study. For this reason, the summary information in this section has been shared to address these components, albeit briefly, and to raise awareness about their effects on the maintainability of Android applications. Also, examining the official Android documentation in this area can facilitate understanding the study topic. \footnote{\url{https://developer.android.com/guide/components/fundamentals}}.