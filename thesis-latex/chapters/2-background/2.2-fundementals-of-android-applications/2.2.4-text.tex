In the Android world, the term "content provider" refers to the component that is designed to manage a mutual application data set that can be stored via a file system or a local database (e.g. SQLite) or any other kind of persistent storage. Content providers define, manage, and supply inter-application data sets. Android applications can provide content providers for other Android applications and through the content providers, any other Android application that has the necessary permissions can query the content provider to read and write data within its permissions. From the Android system perspective, a content provider can be considered as an entry point into an Android application in order to issue named data sets identified by a URI scheme. A solid example to a content provider can be given as the content provider that the Android system provides for the purpose of managing the contact information of the user between multiple apps \cite{9}.