In the Android world, the term "service" refers to a component that runs in the background in order to perform time-consuming and long-running processes or to perform remote processes. In other words, a service is an entry point for Android applications to keep the applications running in the background for any reason. A service does not have a user interface. Other Android components such as activities, fragments can start a service. Once a service is started, it continues to its long-run even if the user starts another application. Android services can be used to perform background operations such as network operations, content provider interaction, I/O processes, and playing music. In Android, a service is represented by the "Service" class and every service must be implemented as a subclass of the Service class that the Android SDK provides \footnote{\url{https://developer.android.com/guide/components/services}}.