This study focuses on solving the maintainability problems of Android applications. However, it is essential to know the fundamental components that create an Android application to really understand the problem.

Since the launch of the first mobile device that works with Android,  the Android operating system has improved, and the way the Android applications are developed changed a lot, but some fundamental components for developing Android applications have more or less stayed the same. Each of these fundamental components was developed for a specific purpose by the creators of the Android Software Development Kit (Android SDK). Knowing these components and their responsibilities are necessary to understand the problem that this study is trying to solve. %Because as it was already stated in the introduction section, one of the reasons for the complications that arise when developing Android applications is the nature of the fundamental Android components.

%Ever since the Android operating system has started running on mobile devices, Android applications are being developed. As of the second quarter of 2021, there are more than two and a half million applications in the Google Play Store \footnote{\url{https://www.appbrain.com/stats/number-of-android-apps}}.Since the launch of the first mobile device that works with Android,  the Android operating system has improved, and the way the Android applications are developed changed a lot, but some fundamental components for developing Android applications have more or less stayed the same. Each of these fundamental components was developed for a specific purpose by the creators of the Android Software Development Kit (Android SDK). Knowing these components and their responsibilities are necessary to understand the problem that this study is trying to solve. Because as it was already stated in the introduction section, one of the reasons for the complications that arise when developing Android applications is the nature of the fundamental Android components. So, in this section, some brief information will be given about these fundamental Android components. For more information and technical details regarding these components, it is recommended to read official Android documentation.

Java, Kotlin, and C++ programming languages can be used for developing native Android applications. There are also other ways of developing Android applications. But as it was already mentioned in the introduction section, this study only focuses on native Android application development. Native Android development means the creation of Android applications that run on Android-powered devices by using the Android Software Development Kit. When developing Android applications in a native way, in addition to the programming side, Android applications are supported by different types of resources such as XML layout files, XML resources, images, data files, etc. The detailed examination of these resources is not within the scope of this study. However, knowing that Android applications do not only consist of code might be useful to see the bigger picture of an Android application. Though, for the purpose of understanding the problem that this study tries to resolve, it is essential to have a basic understanding of the fundamental Android components. Consequently, knowing the nature of an Android application and its components is the first step for solving the maintainability issues of the Android applications and then there come the best practices and latest technologies of Android application development and how to apply them into the Android application development processes. These fundamental components can be listed as:
\begin{itemize}
    \item Activities
    \item Services
    \item Broadcast receivers
    \item Content providers
\end{itemize}

In addition to these main components, the Fragment component also has an important place among Android components. Knowing the details of all these components (e.g. implementation details, life-cycle) plays an important role in solving the complications caused by Android components while developing Android applications. Details regarding key Android components and their responsibilities are beyond the scope of this study. However, as stated before, knowing the nature, responsibilities and working methods of these key components will facilitate understanding of the issues addressed in this study. For this reason, the summary information in this section has been shared to address these components, albeit briefly, and to raise awareness about their effects on the maintainability of Android applications. Therefore, examining the official Android documentation in this area can help to understand the study better \footnote{\url{https://developer.android.com/guide/components/fundamentals}}.

%The remaining of this section will introduce fundamental Android components and the brief information regarding these components to help to understand the problem that is stated in the study. Since the impact of activities is bigger when it comes to the maintainability of Android applications, the information to be provided regarding this component is slightly more detailed when compared to the information regarding the rest of the components.

%Since the impact of activities and fragments is bigger when it comes to Android applications' maintainability, the information was shared regarding these components in the upcoming sections. For the rest of the components, Android's official documentation can be examined \footnote{\url{https://developer.android.com/guide/components/fundamentals}}. 