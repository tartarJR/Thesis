The term "broadcast receiver" refers to the Android system component that enables the system to distribute events that happen outside of the normal application flow to the Android applications. Just like the other main Android application components that were mentioned previously, broadcast receivers are also entry points to the Android applications. As a result of that, the Android system can deliver broadcasts to Android applications regardless if the application is running or not. Broadcast in Android tends to originate from the Android system itself. Low battery notification, captured screen notification, and the screen on/of indicators can be given as examples to the Android system broadcasts. However, Android applications can also initiate broadcasts. Broadcast receivers do not involve displaying a user interface but they have the ability to create notifications in the status bar in order to alert users. Android Software Development Kit provides the "BroadcastReceiver" class and each broadcast receiver must be implemented as a subclass of this provided class in Android applications \footnote{\url{https://developer.android.com/reference/android/content/BroadcastReceiver}}.