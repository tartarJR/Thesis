Today, the complexity of software applications is quite high, and it still is increasing. Today's software systems have many complex concerns such as persistence, real-time constraints, concurrency, visualization, location control \cite{27}. Software engineering, first of all, aims to increase software quality, lower the expenses of software production, and assist maintenance and development. In achieving these aims, software engineers are continuously on the lookout for developing technologies and approaches that increase maintainability, decrease software complexity, enhance understandability, support reuse, and boost development. That is where "Separation of Concerns" (SoC) emerges as a solution. 

In the context of software engineering, SoC is a software design principle for separating a software system into discrete modules that each module addresses a single concern. A good application of SoC to a software system provides benefits such as increasing maintainability, reducing complexity. The borders for different concerns might differ from a software system to another. Concerns depend on the requirements of a software system and the forms of decomposition and composition \cite{28}. 

In the context of Android, the importance of SoC is even more apparent. Android applications have different concerns with clearly drawn borders. Such concerns can be named as visualization and presentation of data, business logic, networking, persistence, location services. Depending on the requirements of an Android application, there might be other concerns as well. In light of the information above, it is vital to identify these concerns before Android applications are developed, and during the development phase, to use techniques that can apply SoC well.