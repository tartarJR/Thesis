\textbf{\textit{"Good programmers write code that humans can understand." - Martin Fowler}} \cite{21}

In ancient times, when computers were big, heavy, and slow, programmers were limited to use low-level programming languages that are working close to computer CPUs. These were imperative programming languages, and the programs written in these programming languages were following the procedural programming paradigm. Although that approach worked fine, the biggest problem was that these programming languages were designed to be understood by computers, not humans. The main reason for this situation was that, back in that time, computers lacked proper hardware, resources, and speed. Consequently, the priorities back then were different. The computer programs had to be fast and less memory consuming. However, this situation has changed in the current day. Even a low-quality mobile device is much stronger and smarter than the computers that people were using a couple of decades ago, and software systems became complex. Although this change brought a positive impact on the end-user side as it also brought more functionality and ease, the impact it brought the software development side is complexity. Especially when developing large-enterprise software products, ignoring that fact and not considering how to overcome this complexity may cause significant failures. In this context, new programming languages and paradigms were born, the priorities have altered, and this new reality brought different challenges and new quality standards to software development \cite{12}.

When the priorities of modern software development are analyzed today, the maintainability of software systems emerges as one of the most critical priorities, perhaps even the most important. According to the IEEE Standard Glossary of Software Engineering Terminology, the term "maintainability" is the ease with which a system or component can be modified for use in applications or environments other than those for which it was specifically designed \cite{20}. In the context of software engineering, maintainability means how well a software system is understandable, repairable, and extendable. Maintenance is one of the most important parts of the software development life cycle because the time spent on maintaining software systems requires more time and resources than the rest of the process. The relative expense for maintaining software and dealing with its development speaks to over 90\% of its absolute expense \cite{4}. The level of maintainability that a software system depends on several different factors. Overall, a software system can be considered maintainable if it is simple to grasp how it works and what it does, and making changes such as adding new features and fixing bugs is easy. Besides, maintainability is directly related to well-known software engineering concepts such as coupling, complexity and cohesion. Knowing the relationship between these concepts and maintainability is also essential in measuring maintainability, which is one of the topics of this study. Previous studies have proved this relationship. The inverse ratio between complexity and maintainability was mentioned in Saifan and Rabadi (2017) on measuring maintainability in Android applications \cite{34}. Also, the inverse relationship between coupling and maintainability and the correlation between cohesion and maintainability are discussed in detail in Barak et al., (2012) on maintainability metrics in open-source software \cite{33}.

Considering the life cycle of software systems might help to understand the importance of maintainability. Software systems are born, they live, they change, and eventually, they die. However, their lifetime is generally long, and during their lifetime, new features are added, some features are removed, bugs are fixed, and often their development team changes. Usually, there is always a time gap between these changes. In other words, developers might need to make a change to the software system, which they worked in weeks or months before. In such cases, developers should be able to understand the systems easily even months after. Besides, changes to the code base should be able to be done with ease without breaking the other parts of the software system. Also, when a new developer joins the development team, onboarding should be smooth, and the new developer should be able to understand the purpose of the software system easily. Ignoring all these might cause companies a significant amount of time and money. That is what makes maintainability that important. Developing maintainable software systems is the way to tackle such issues.

The importance of maintainability for software systems can also easily be seen when looking at its role in the software development lifecycle and its effect on software development costs. The maintenance period for a software system starts as soon as the system is developed. Thus, maintainability becomes a vital aspect for applying new customer needs, adding/removing new features, adapting to the environmental changes \cite{23}. The time that has to be spent on the maintenance of complex software products is comparatively more extended than the rest of the software development lifecycle processes. Reports indicate that the amount of effort spent on software maintenance is between 65\% and 75\% of the total amount of effort \cite{13}. Also, another report points out that maintenance cost is 75\% of the total project cost, and the cost for maintaining source code is ten times bigger than developing the source code \cite{22}. In his famous book "Clean Code", Robert C. Martin explains how a top-rated company in the late 80s was wiped out from the business due to the lack of maintainability and poorly managed code organization. When the release cycles of their prominent product extended, due to the unorganized code base of their product, they were not able to fix bugs, prevent crashes, and add new features. Eventually, they had to withdraw their promising product from the market and went out of the business. Lousy code and consequently, maintainability was the reason for this company to go out of the business \cite{11}. This real-life example clearly shows how vital maintainability is to software systems and what fatal consequences it can cause if ignored.

The importance of the maintainability for software systems is evident and this situation is no different for Android Applications. In fact, the importance of maintainability for Android applications is even higher since Android applications have a very active software development life cycle. Given that the growing user demands and business needs making the Android applications more and more complex and Android applications having frequent update rates, it is not hard to see how important the maintainability is for Android application development. This situation is one of the main sources of motivation for this study. 

 In addition to that, in the context of Android, when the main challenges mentioned in this study are evaluated together, the importance of maintainability as a non-functional requirement becomes even more evident for Android application development because the high level of maintainability is the way to overcome the challenges and complexities mentioned in this study while developing Android applications. In consequence, the question is how to achieve the goal of developing Android applications with high maintainability. From the software development point of view, the Android platform does not have strict rules on how the applications are developed. Developing maintainable applications is not an obligation. However, developing Android applications with high maintainability is a need to solve the difficulties mentioned in this study in a timely and cost-efficient manner, facilitate the development processes for the Android developers, and increase the quality of the Android applications. The methods to be followed and the technologies to be used in Android applications for meeting these requirements constantly evolve, and the topic is still controversial among the Android community. Different solutions have been proposed and tried since the birth of the platform. However, the unchanged reality is that the most important criteria for building reliable software systems in a timely and cost-efficient manner is maintainability and of course, this reality is not different for Android applications. 