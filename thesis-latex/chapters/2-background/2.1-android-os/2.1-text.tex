Android is an open-source operating system for mobile devices. The Android project/operating system was initially created by the Open Handset Alliance which includes organizations from various industries such as Google, Vodafone, T-Mobile, LG, Huawei, Asus, Acer, and eBay to give some examples \cite{6}. To be more specific, Android is an open-source software stack made for a varied range of mobile devices with different structure parameters. The main goal of the Android project is to provide an open software platform accessible for a variety of stakeholders such as developers, engineers, carriers, and device manufacturers to turn their innovative and imaginative ideas into successful real-world products that improve the mobile experience for the end-users. Today, numerous organizations from Open Handset Alliance and also other organizations are supporting and investing in Android and the project is led by Google. Android is designed in a distributed way to avoid the issue of the central point of failure. In another means, different industry players confine or control the advancements of another. As a result, a production-quality consumer product comes along with open source code that is ready for customization \cite{5}.

The platform architecture of Android consists of 6 major layers. Each layer has its own responsibility and handles a different area of the Android operating system. The following figure demonstrates the layers in a way that they are ordered from the highest level of abstraction from the top to the bottom.
\begin{figure}
    \centering
    \includegraphics[scale=0.5]{figures/android_os.png}
    \caption{Android platform architecture \protect\cite{5}}
    \label{fig:android_platform_architecture}
\end{figure}

The section below gives a brief description of each major layer of the Android platform architecture mentioned in Fig. \ref{fig:android_platform_architecture}: 