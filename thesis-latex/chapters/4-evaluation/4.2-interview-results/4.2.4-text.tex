The last two interview questions were directed to the participants to qualitatively evaluate the methods and technologies used by the company while developing Android applications  (see section \ref{section:2.3}) from the maintainability point  of view. 

When the responses of the participants are examined, positive comments about Kotlin programming language draw attention. Kotlin was found to be more concise and more readable than Java by the participants, and the participants stated that this situation has a positive effect on maintainability. Also, one participant stated that programming language did not have any positive or negative effect on maintainability. 

When looking at the answers about the SOLID and Clean Code principles, again, mostly positive reviews are seen. Participants think that the impact of these principles are positive on maintainability in terms of regulating the coding conventions of their codebases, increasing readability, facilitating separation of concerns and increasing consistency. 

On the subjects of architecture and design patterns, the participants stated that they found the MVVM design pattern positive for maintainability for it decreasing the coupling level between the classes responsible for view and presentation. Besides, the support of the Android Architecture Components framework offered by Google's Android team for MVVM and the fact that this framework was seen as a stable framework by the participants were considered positively by the participants in terms of maintainability. In the case of Clean Architecture, most participants stated that this architecture successfully managed to separate concerns, thus reducing the complexity and coupling levels of Android codebases, therefore increase the maintainability of the Android applications. However, these participants also stated that this architecture increases the complexity of small and medium-sized projects and decreases the readability of the codebase thus negatively impact the maintainability. 

It was observed that the participants made different comments about the effect of the third-party libraries used by the team on the maintainability of the applications. For example, the RxJava library was criticized in different ways. Participants stated that the steep learning curve of the RxJava library decreases the understandability of Android applications. Thus maintainability was negatively affected. In addition, the risk of RxJava becoming obsolete in the near future was also stated by some participants. It was mentioned that this situation had a negative effect on maintainability. On the other hand, some participants stated that this library has positive effects on maintainability due to its strong community, documentation and advanced features. Participants stated that Dagger 2 library has a positive impact on maintainability since it is a reliable and well-documented library supported by the Google Android team as a standard way of dependency injection. On the other hand, it was emphasized by the two participants that the complex structure of this library and its steep learning curve may cause problems in readability and understandability and that maintainability may be negatively affected. Regarding the Retrofit and Apollo libraries, all of the participants stated that these libraries have positive effects on maintainability. Reasons for that mentioned by the participants as strong community support and documentation, stability and reliability. In addition, the participants stated that the Retrofit library had become the industry standard, and it has a positive effect in terms of maintainability as it is easier to use than other alternatives. Architecture Components framework offered by Google's Android team has also found handy by the participants when it comes to the maintainability of Android applications. And as stated before, the reason for that is that the framework was seen as a stable framework by the participants and considered positively by the participants in terms of maintainability. However, one participant has stated that due to some internal issues that this framework has, the framework might badly affect maintainability. 

Ultimately, participants also stated that improvements can be made in the currently used methods and technologies. It is seen that the participants mentioned that the documentation of the Android applications could be improved. Also, they also mentioned that libraries can be reconsidered based on the current trends and up to date technologies.