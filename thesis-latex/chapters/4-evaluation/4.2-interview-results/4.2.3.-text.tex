The fourth, fifth and sixth questions of the interviews are designed to determine the importance of maintainability for the case company.

The fourth question of the interview asked the participants the importance of maintainability for the case company. Five of the participants stated that maintainability is an essential requirement for the company due to how the company works. Since the company works for different customers from different fields and the developer circulation in these projects is high, these participants stated that the maintainability of the projects is critical in terms of cost, time and effort. These participants emphasized that it is vital for the company to develop Android applications with high understandability, high modifiability, and high maintainability due to facilitating the development, hand over of the projects to the customer and maintenance operations required in the long term. Also, two participants stated that maintainability is already critical for software systems and that there is no extra situation that makes it more important for the company.

The fifth question of the interview asked the participants the importance of maintainability for the Android applications. Four of the participants stated that maintainability is important in solving the complexity issues in Android applications. Also, the participants stated that the fact that the Android platform and third-party Android libraries are evolving rapidly makes the maintainability of the Android application crucial. Participants stated  that if unstable libraries are preferred for the sake of the industry trends the fast evolution of the platform may cause problems in terms of maintainability. Also, two participants stated that maintainability is already important for software systems and that there is no extra situation that makes it more important for the Android applications.

The sixth question of the interview asked the participants for the most important matter for Android applications when it comes to maintainability. In general, it was seen that the participants provided 2 different answers to this question rather than stating a single matter. When the responses of the participants are examined, it is seen that the application architecture is emphasized four times, the third-party libraries emphasized 4 times, and the documentation once. The reason why third-party libraries affect maintainability was explained in the previous paragraph. It was seen that the developers put the same emphasis when answering this question as well. In addition, it was emphasized by the participants that the choice of architecture increases the maintainability of the applications as it makes the codebases of the applications better structured, standardized and consistent. Lastly, one participant stated that the documenting of the Android codebases improves understandability.