When other academic studies dealing with the measurement of maintainability of software systems were reviewed, it was seen that many studies use quantitative methods. When the quantitative evaluations made in these studies are examined, it is seen that Android applications use several object-oriented software metrics while evaluating maintainability. Details of these studies were shared earlier in section \ref{section:3.2}. Although it cannot be claimed that such quantitative measurements are inaccurate, it would not be wrong to say that these measures are inadequate at times. It is essential to make qualitative evaluations and quantitative evaluations to increase the efficiency of the evaluations. In this way, it may be possible to measure developers' experiences and their effects on maintainability. For these reasons, an Android developer survey and interviews were conducted within this study's scope. This section covers these qualitative methods.

\subsubsection{Android Developer Survey}
\label{section:4.1.1}
The first step of qualitative evaluations in this study is the Android developer survey. The fast evolution of third-party Android libraries and developers trends is vital for Android applications' maintainability because the consistency and stability of principles and third-party libraries affect maintainability. Therefore, this survey was conducted to identify current Android trends and provide support for this study's evaluation. While determining this survey's questions, priority was given to principles and technologies that affect maintainability. Although the questions are generally prepared to cover the methods used by the Mooncascade Android's team, it can be said that the questions reflect the Android technology stack in general. Also, the first question was added to the survey to learn about the competence of participants. Learning the competence of the participants is important for increasing the probability of getting more accurate results from the survey because the preferred technology and methods for Android application development differs based on the participants’ experience. The survey questions are presented below. 
\begin{itemize}
    \item How many years of professional experience do you have as an Android developer?
    \item Which programming language do you use for Android application development?
    \item What presentational design pattern do you apply to your Android apps?
    \item Do you apply Uncle Bob's Clean Architecture to your Android applications?
    \item Do you follow SOLID principles while developing Android applications?
    \item Do you follow Uncle Bob's Clean Code principles while developing Android applications?
    \item What networking library do you use?
    \item How do you handle asynchronous events in your Android applications?
    \item What dependency injection library do you use in your Android applications?
    \item Do you use Android Architecture Components in your Android applications? (LiveData, ViewModel, Room, etc.)
\end{itemize}
The questions are organised with the help of the Forms application provided by Google. The Android developer survey has been delivered to Android developers from different companies in different countries through accounts or groups of Android developer communities on social media platforms such as LinkedIn, Discord, and Twitter. The author of the study has also shared the survey with many of his colleagues/ex-colleagues working in different companies/countries.

\subsubsection{Interviews with Team Members}
\label{section:4.1.2}
In this kind of survey, it is essential to ensure the participants’ diversity to get sufficiently accurate and generally reflective results. The Android developer survey found participants in 5 different countries (Germany, Turkey, Portugal, Estonia, and Russia) through the author's current and former colleagues. The survey was answered by Android developers working in 7 different well-known companies in these countries. Also, as previously stated in the 3rd part, the survey reached answers from random Android developers through various social media platforms. In this way, the participants’ diversity was increased, and getting better results was ensured. 

Also, the first question was added to the survey to learn about the competence of participants. As can be guessed, the probability of getting more accurate results from a survey formed to determine the preferred technology and methods for Android application development is directly proportional to the participants’ experience. Results can be seen in Fig. \ref{fig:participant_bg} below.
\begin{figure}[ht!]
    \centering
    \includegraphics[scale=0.25]{figures/survey_q1_dev_experience.png}
    \caption{Participant background results}
    \label{fig:participant_bg}
\end{figure}
\FloatBarrier

As stated before, the participant diversity was provided to improve the accuracy of the survey. When the chart above is examined, 63\% of the participants have more than 3 years of experience while 40\% of them has 5+ years of experience. People with 0-2 years of experience are around 37\% of the all participants.
