In this chapter, the methods to evaluate the impact of the techniques and technologies used by the case company on maintainabilty are explained. This study follows the triangulation strategy \cite{51}. This strategy is applied by conducting two different methods within the scope of this study. These methods are quantitative and qualitative evaluations. Quantitative evaluation is made using object-oriented software maintainability metrics, and qualitative evaluation was made via interviews and surveys with Android developers. In this way collecting data from different resources is ensured, and more coherent results can be obtained. Another reason for using this method is that the study subject is closely related to the industry. Due to this close relationship, it is predicted that qualitative evaluations of the Android community and Android developers could increase the accuracy of this study. Thus, maximum efficiency by going beyond the traditional quantitative measurement techniques is aimed. Also, why and how these methods are chosen are explained in this chapter. Hence, the first research question is answered.

\subsection{Qualitative Evaluation}
The Android developer survey results, which is the first step of qualitative evaluations, are shared in this section. Android developer trends have been identified regarding technologies and principles that are likely to directly or indirectly impact the maintainability.

\subsubsection{Conduction Period of the Survey}
The first step of qualitative evaluations in this study is the Android developer survey. The fast evolution of third-party Android libraries and developers trends is vital for Android applications' maintainability because the consistency and stability of principles and third-party libraries affect maintainability. Therefore, this survey was conducted to identify current Android trends and provide support for this study's evaluation. While determining this survey's questions, priority was given to principles and technologies that affect maintainability. Although the questions are generally prepared to cover the methods used by the Mooncascade Android's team, it can be said that the questions reflect the Android technology stack in general. Also, the first question was added to the survey to learn about the competence of participants. Learning the competence of the participants is important for increasing the probability of getting more accurate results from the survey because the preferred technology and methods for Android application development differs based on the participants’ experience. The survey questions are presented below. 
\begin{itemize}
    \item How many years of professional experience do you have as an Android developer?
    \item Which programming language do you use for Android application development?
    \item What presentational design pattern do you apply to your Android apps?
    \item Do you apply Uncle Bob's Clean Architecture to your Android applications?
    \item Do you follow SOLID principles while developing Android applications?
    \item Do you follow Uncle Bob's Clean Code principles while developing Android applications?
    \item What networking library do you use?
    \item How do you handle asynchronous events in your Android applications?
    \item What dependency injection library do you use in your Android applications?
    \item Do you use Android Architecture Components in your Android applications? (LiveData, ViewModel, Room, etc.)
\end{itemize}
The questions are organised with the help of the Forms application provided by Google. The Android developer survey has been delivered to Android developers from different companies in different countries through accounts or groups of Android developer communities on social media platforms such as LinkedIn, Discord, and Twitter. The author of the study has also shared the survey with many of his colleagues/ex-colleagues working in different companies/countries.

\subsubsection{Participant Background}
In this kind of survey, it is essential to ensure the participants’ diversity to get sufficiently accurate and generally reflective results. The Android developer survey found participants in 5 different countries (Germany, Turkey, Portugal, Estonia, and Russia) through the author's current and former colleagues. The survey was answered by Android developers working in 7 different well-known companies in these countries. Also, as previously stated in the 3rd part, the survey reached answers from random Android developers through various social media platforms. In this way, the participants’ diversity was increased, and getting better results was ensured. 

Also, the first question was added to the survey to learn about the competence of participants. As can be guessed, the probability of getting more accurate results from a survey formed to determine the preferred technology and methods for Android application development is directly proportional to the participants’ experience. Results can be seen in Fig. \ref{fig:participant_bg} below.
\begin{figure}[ht!]
    \centering
    \includegraphics[scale=0.25]{figures/survey_q1_dev_experience.png}
    \caption{Participant background results}
    \label{fig:participant_bg}
\end{figure}
\FloatBarrier

As stated before, the participant diversity was provided to improve the accuracy of the survey. When the chart above is examined, 63\% of the participants have more than 3 years of experience while 40\% of them has 5+ years of experience. People with 0-2 years of experience are around 37\% of the all participants.

\subsubsection{Issues}
The most severe limitation of the survey was undoubtedly in finding respondents. Although the survey was shared on many different social media platforms and developer community pages, it was able to find very few participants compared to the number of people on these platforms and communities. The initial target for the number of participants was 150-200, and a serious effort was made to reach this number.

Besides, some corrections and filtering were made for the 164 responses collected from these Android developers to avoid off-topic responses and collect non-standard answers under a single topic, thus presenting more consistent results both visually and numerically. For example, for some of the questions, there is the option "other" among the answer options offered, and users can choose this option and enter their answers. In this case, when the data is plotted, results such as the following may occur.
\begin{figure}[ht!]
    \centering
    \includegraphics[scale=0.20]{figures/survey_non_standard.png}
    \caption{Example chart plotted with non-standardized answers}
    \label{fig:non_standard_chart_example}
\end{figure}

As shown in the chart, "Toothpick" or “Hilt” libraries do not have a ready-made response option. Android developers using this library have given their answers in different forms. Thus a chart such as this has emerged. Therefore, it was deemed necessary to arrange the inputs in different forms, which correspond to the same answer. 

As an example of the need to filter some answers, a situation in the chart below can be shown. As can be seen in the chart above, which was obtained from the unfiltered survey results, it is seen that developers who develop Android in other forms also participated in the survey. Although it was clearly stated to the participants that the survey included only "Native" Android developers before they filled in the questionnaire, a few such cases were unfortunately not avoided due to the human factor. These kinds of responses have been filtered and edited as they will not contribute to this survey’s purpose and reduce the survey’s accuracy.
\begin{figure}[ht!]
    \centering
    \includegraphics[scale=0.20]{figures/survey_corrupted_chart.png}
    \caption{Example chart plotted with corrupted data}
    \label{fig:corrupted_chart_example}
\end{figure}

The above and other similar situations have been filtered and edited as they will not contribute to the survey's purpose and they reduce the survey’s accuracy. Firstly, the survey results were extracted as a Google Sheets file to do this filtering and editing. Then inappropriate data in this file was corrected or filtered, and finally, the charts and numbers were obtained from the accurate data of the file. While applying corrections and filtering, no changes or manipulations were made on the relevant data. Around 20 of the 164 responses were edited to arrange the inputs in different forms, which correspond to the same answer. Besides, five corrupted responses were removed. After the filtering process, charts were created from the remaining 159 responses to obtain more accurate data. The interpretation of the responses is based on this filtered and corrected final version. The charts obtained through filtered and corrected responses and the data obtained from these charts’ interpretation are presented below, respectively. Since it is possible to choose more than one answer for some questions, it should be taken into account that the total number of answers for each question may exceed the total number of participants. Besides, since the first question was added to the survey a bit later than the answers started being accepted, the answers to these questions are less than the rest. While examining the answers, it will be helpful to consider these two situations to prevent confusion.


\subsubsection{Programming Language}
\input{chapters/4-evaluation/4.1-android-dev-survey-results/4.1.4-text}

\subsubsection{Architecture}
\input{chapters/4-evaluation/4.1-android-dev-survey-results/4.1.5-text}

\subsubsection{Principles}
The fifth question of the survey asked the participants whether they follow SOLID principles while developing Android applications. Results have shown that 66.3\% of the participants declared that they follow SOLID principles and 20.6\% of them declared that they might follow these principles. 6.3\% of the participants stated that they do not apply the SOLID principles, and 6.9\% stated that they are not aware of these principles. The figure below contains the graphical breakdown of this data.
\begin{figure}[ht!]
    \centering
    \includegraphics[scale=0.27]{figures/survey_q5_solid.png}
    \caption{SOLID principles usage results}
    \label{fig:solid}
\end{figure}
\FloatBarrier

The 6th question of the survey asks the participants whether they apply the "Clean Code" principles while developing Android applications. The results in the form of a pie chart can be seen below in Fig. \ref{fig:clean_code}.
\begin{figure}[ht!]
    \centering
    \includegraphics[scale=0.27]{figures/survey_q6_clean_code.png}
    \caption{Clean Code principles usage results.}
    \label{fig:clean_code}
\end{figure}
\FloatBarrier

According to the results, 75\% of the participants stated that they either used or could use these principles. While 13.8\% of the participants stated that they do not use these principles, it was observed that 11.3\% of the participants were not even aware of these principles.


\subsubsection{Libraries}
\input{chapters/4-evaluation/4.1-android-dev-survey-results/4.1.7-text}

\subsection{Maintainability Evaluation with Object-Oriented Metrics}
\label{section:4.2}
The literature review results revealed that (see section \ref{section:3.2}) there are different metrics and approaches for measuring the maintainability of software systems. On the other hand, considering the differences between Android applications and traditional software systems, it is obvious that a different maintainability assessment method is required for Android applications. When the problems encountered while developing Android applications, which were mentioned in the first chapter, are examined, it is seen that software characteristics such as software complexity, understandability/readability, modifiability come to the fore. Considering the relationship between maintainability and these characteristics, it was decided that the maintainability measurement to be carried out within the scope of this study should cover these characteristics. To evaluate these software characteristics, it is considered that complexity, cohesion and coupling concepts can be used because software complexity, modifiability, understandability/readability are affected by these concepts. These characteristics and concepts can be matched respectively with each other. As a result of this consideration, it was concluded that the measurements made based on these concepts could be efficient when measuring the maintainability of Android applications. Also, studies have shown that results retrieved from evaluating these concepts proved to define the level of maintainability \cite{33}. Therefore, the maintainability model of this study for quantitative evaluation was formed based on the concepts of complexity, cohesion and coupling and research was done on measuring these concepts effectively. As a result, five metrics were selected for this purpose. While selecting these metrics, priority has been given to metrics that can handle a software system as a whole in the areas of complexity, cohesion and coupling to make more efficient evaluations. These metrics and their intended use are listed below.
\begin{itemize}
    \item \textbf{Weighted Method Count (WMC):} This metric is used to measure object-oriented software systems’ complexity. WMC represents a class's cyclomatic complexity, also known as McCabe complexity \cite{35}. It, therefore, portrays the complexity of a class as a whole, and this measure can be used to indicate the maintainability level of the class. The number of methods and complexity can be used to divine maintaining effort. If the number of methods is high, that class is described as domain-specific and is less reusable. Also, such classes tend to be prone to change and defects.
    \item \textbf{Depth of Inheritance Tree (DIT):} This is another metric to measure software complexity. Inheritance increases software reusability; however, one side can create complexity by possibly violating encapsulation since the subclass needs to access the superclass. Furthermore, changes made during maintenance might increase the inheritance tree's depths by adding more children. Therefore, by assessing the inheritance tree available in the product, it is easy to predict how much effort needed to make it stable \cite{33}. It is harder to predict its behaviour if the tree depth is high, and this causes maintenance issues.
    \item \textbf{Number of Children (NOC):} NOC measures the number of descendants of a class, and it is used to measure the coupling level for the corresponding class. NOC also indicates the reusability level of a software system. It is assumed that the number of child classes and the maintainer's responsibility to maintain the children's behaviour are directly proportional. If the NOC level is high, it is harder to maintain and modify the class \cite{36}.
    \item \textbf{Coupling Between Object Classes (CBO):} This metric calculates the number of connections to other classes from a particular class, and it is used to measure coupling. A class is considered coupled if it depends on another class to get its work done \cite{34}. CBO metric is related to the reusability of the class. High coupling makes the code more difficult to maintain because changes in other classes can also affect that class. Therefore these classes are less reusable and less maintainable.
    \item \textbf{Lack of Cohesion of Methods (LCOM):} This metric is used to determine how class methods are related to each other, and it is applied to evaluate cohesion. Cohesion promotes the maintainability of the software systems. High cohesion for a class meant the class is understandable, maintainable and easy to modify \cite{33}.
\end{itemize}

CodeMR static code analysis tool has been chosen for this study for the purpose of quantitative evaluation\footnote{\url{https://www.codemr.co.uk}}. After this decision, communication was established with the CodeMR team, and a free license was obtained to be used in academic studies. It is a powerful software quality tool that is integrated with IDEs and supports multiple programming languages. The tool provides a set of metrics to measure coupling, complexity, cohesion and size. Besides, the tool provides a visualisation centric approach and generates detailed reports supported by different visualisation options. In this way, it facilitates understanding the results and allows recognition of the bigger picture of the projects from the complexity, coupling, and cohesion point of view. Detailed information on these metrics and assessment methods can be found on the tool's website. The tool can also be installed as a plugin in Android Studio and is very easy to use. Lastly, two studies conducted using this tool were examined to gather more information regarding the tool \cite{38,39}.The CodeMR static code analysis tool has been chosen for this study due to reasons mentioned above. To evaluate the impact of the methods and technologies mentioned in section \ref{section:2.3} on maintainability, the metrics were applied on two different code bases of the same project via CodeMR. Thus, the impact could be measured by comparing the results from both versions. Although the project's full content cannot be published due to the confidentiality reasons, more information regarding these codebases is shared in the next section.
