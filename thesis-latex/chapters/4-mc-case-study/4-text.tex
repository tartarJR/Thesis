\subsection{Case Company}

\subsection{Principles}
\subsubsection{SOLID Principles}
\label{section:4.2.1}
\subsubsection{Clean code}
\label{section:4.2.2}

\subsection{Programming Language}
\label{section:4.3}

\subsection{Architecture}
\subsubsection{MVVM}
\label{section:4.4.1}
\subsubsection{Clean Architecture}
\label{section:4.4.2}

\subsection{Libraries}


\subsubsection{Dependency Injection}
\label{section:4.5.1}
\subsubsection{Networking}
\label{section:4.5.2}
\subsubsection{Asynchronous Events}
\label{section:4.5.3}
\subsubsection{Android Architecture Components}
\label{section:4.5.4}
\subsubsection{Dependency Management}
\label{section:4.5.5}

\subsection{Summary}

and these developers can be considered as mid or senior level Android developers and 40\% of the participants have five years or more experience.
When the 164 participants of the survey are examined, it can be easily said that the survey level is sufficient in terms of participant diversity. This fact proves that the respondents are Android developers who have sufficient experience and are proficient at Android development. The responses to the rest of the survey questions can be interpreted based on that fact, which is believed to provide more accurate results.

Considering that Kotlin is a programming language suggested by Google and Android and is more "programmer-friendly" than Java, it is not difficult to understand that the above table is not surprising. As of 2020, Google declared that more than 60\% of Android applications were developed with Kotlin. It can be said that the survey results largely overlap with this statement\footnote{\url{https://developer.android.com/kotlin}}. On the other hand, he fact that some users still use Java can be explained by the existence of Android applications developed with Java before Kotlin was declared as an official programming language for Android. As stated in section \ref{section:4.3}, Mooncascade's Android team develops Android applications using Kotlin programming language, unless otherwise requested by its customers. When the survey results presented in detail above and the company's choice are compared, it is seen that this choice coincides with the Android community’s current trends.

Android Architecture Components framework provides some out of box solutions for MVVM. We see that Android developers highly adopt it as of the first quarter of 2021. 
In other words,  it can be said that as the developer experience increases, the tendency of the developers to choose more than one design pattern also increases. In this case, it can be said that experienced Android developers make the presentational design pattern selection by considering which design pattern will fit the project size and content, rather than what is more popular. As another proof of this situation, it can be shown that developers with 0-3 years of experience have answered this question by selecting the MVVM option. In other words, it is possible to talk about the tendency of Android developers at the beginning of their career to choose popular or "hype" technologies. proving that the knowledge of architecture and design pattern in software development correlates with experience. Lastly, concerning this question, it will be helpful to mention the participants’ tendency to choose design patterns such as MVC, MVP and MVI. Comparing the survey results with Figure 7, which is presented in section \ref{section:2.7} and cited from a study conducted a few years ago in Android architectures, we are faced with similar results despite minor differences. When we look at the comparison results, it is seen that MVVM and MVP were popular among the Android community a few years ago, but MVVM is more preferred today. As mentioned before, it can be said that since the MVVM design pattern started to be provided as an out of box solution by the Google Android team three years ago, this situation increased usage of the MVVM design pattern. In the survey, we also see that 18 of the participants declared that they used the MVC design pattern. Although the MVC design pattern is considered an outdated design pattern in the Android community, the existence of projects developed using this pattern, and considering the suitability of this pattern for small projects; it is understandable why the pattern is still in use. This fact is not surprising, given the MVI design pattern’s growing popularity during 2020 and 2021. It can be said that this population will increase even more in the upcoming period. As stated in section \ref{section:4.4.1}, Mooncascade's Android team prefers the MVVM presentational design pattern when developing Android applications. When the survey results (presented in detail above) and the company's choice are compared, it is seen that this choice coincides with the Android community’s current trends.

Clean Architecture's details, pros and cons were previously shared, but it is widely used among developers, as seen from the survey results. As can be seen in Fig. \ref{fig:arch_patterns}, which is cited from a study on Android architecture carried out a few years ago. Considering the advantages of Clean Architecture, especially when developing large and complex Android applications, and the growing and complexity of Android applications, developers' choice of Clean Architecture makes much sense. Finally, it is possible to say that the Clean Architecture choice of the Mooncascade Android team coincides with the Android developer trends.

The importance of the SOLID principles and their use requirements were discussed in detail in section \ref{section:SOLID}. Considering how important it is to comply with SOLID principles in software development processes, it can be said that this rate is below expected. It is not easy to understand why people who develop software professionally in the Android field or any other field do not want to follow SOLID principles or are not aware of these principles, especially if these people are experienced developers. As stated in chapter 4 before, Mooncascade's Android team actively applies SOLID principles in Android application development processes. This selection is compatible with general Android developer behaviour, as can be seen in the results above.


The purpose, advantages and disadvantages of these principles are given in section 4 in detail. 
Although there are many advantages of Clean Code principles, discussions are still going on in Android and other software development communities about Uncle Bob and his principles. From this point of view, it can be understood that although most of them actively use these principles, some developers do not. This situation can be interpreted as applying advanced concepts such as Clean Code or SOLID while developing the software directly proportional to the experience. Mooncascade's Android team mainly applies Clean Code principles in Android application development processes. This selection is compatible with general Android developer behaviour when compared to the results above. Further information about how Mooncascade's Android team applies Clean Code principles can be found in section \ref{section:4.4.2}.

Although the network library's use does not directly affect maintainability, this question was included in the questionnaire. It was also among the aims of this study to identify developer tendencies. Also, the use of some advanced networking libraries indirectly affects maintainability due to the out of box solutions they offer. For this reason, it was deemed appropriate to add this question to the survey. 
It would not be wrong to say that this library is mainly preferred due to its ease when integrating back-end systems running on REST architecture into Android applications. More detailed information and comments about these libraries can be found in \ref{section:4.5.2}. Mooncascade's Android team prefers Retrofit or Apollo libraries depending on the back-end system’s type to be used in the project. This preference is in line with the survey results. APOLLO ALTERNATIFI YOK

This question was included in the survey, considering that many Android applications are based on asynchronous events and the impact of the tools used in managing these events on the application architecture and thus on maintainability.
The use of more than one solution can be explained by applications that need to be maintained or preferring a solution based on the project needs. This situation can be explained by maintaining some previously coded applications using the AsyncTask and are still in use. The use of this solution is no longer recommended \footnote{\url{https://developer.android.com/reference/android/os/AsyncTask}}. Usage of Kotlin coroutines is increasing among Android developers, as it is easier to learn and use than the RxJava library and because it requires no external dependency. Although RxJava has a steep learning curve and faces the growing popularity of the Kotlin coroutines, it is still preferred by many Android developers for the advanced features it offers. However, there has been a severe increase of applications that have recently migrated their RxJava solutions to Kotlin coroutines\cite{42}. Although the Mooncascade Android prefers RxJava for now, it has been continuing its efforts to switch to Kotlin Coroutines solution. Details on how RxJava is by used are shared in section \ref{section:4.5.3}

, which significantly impact software maintainability and software architecture when developing Android applications.
Dagger 2 and Hilt are DI frameworks recommended by the Android team. However, it is predicted that Hilt's use will surpass Dagger 2 soon, primarily due to the ease of learning it brings and the decrease in boilerplate code  \footnote{\url{https://developer.android.com/training/dependency-injection/hilt-android}}. It can be said that the Koin is preferred among Android developers because of its ease of learning and its ability to get integrated into Android applications with much less boilerplate code when compared to Dagger 2. Also, it is essential to mention that Koin was developed by using Kotlin programming language. This situation is not surprising given that all of the participants, who were not aware of the concept of DI, had less than a year of experience. Because DI is an advanced software development concept, and its practical implementation is a technique that requires solid experience. It is not mandatory to use any DI framework when developing Android applications. Therefore, it can be mentioned that 17.5\% of the participants stated that they do not use any framework and apply their custom solutions. Mooncascade's Android team applies DI principles in their projects and makes these applications through the Dagger 2 framework. Details on how this framework is used are shared in section \ref{section:4.5.1}. The Team is also considering migrating to Hilt soon.

The high rate of usage is understandable, considering the out of box solutions it offers in solving some of the difficulties encountered while developing Android applications (which were mentioned in the first section, e.g. activity/fragment life-cycle) and the other facilities it provides for Android developers. In addition to this situation, there are groups in the Android community that are distant from this framework because it causes some other difficulties while solving the previously mentioned problems. This claim is controversial, and its details are beyond the focus of this study. However, this may be the reason why some participants do not prefer using this framework.
 Mooncascade's Android team prefers to use the Android Architecture Components framework. Details on how this framework is used are shared in section \ref{section:4.5.4}.