This study covers the subject of evaluating the impact of the methods and technologies used by the software product development company Mooncascade while developing Android applications on the maintainability of these applications. For this purpose, quantitative and qualitative evaluation methods were determined to measure the impact of the methods and technologies used by the Android team of the case company on maintainability and the evaluations were carried out using these methods.

To make a qualitative evaluation, a public Android survey was carried out public with random Android developers and interviews were conducted with each member of the Android team of the company. Qualitative evaluation results indicate that the methods and technologies used by the company while developing Android applications have a positive impact on the maintainability of these applications while also revealing that there are some shortcomings and areas open to improvement. Qualitative evaluations also led to findings that prove the importance of maintainability for software systems, Android applications and case company.

To make the quantitative evaluation, a set of object-oriented software metrics were used. These metrics were applied to common features of the different codebases belonging to the same project through the CodeMR static code analysis tool. While determining these metrics, metrics that can best evaluate the entire software system in terms of maintainability were preferred. For this purpose, priority was given to metrics related to concepts such as complexity, coupling and cohesion, whose relationship with maintainability has been proven. After obtaining the results, interpretations and comparisons were made for each codebase. As a result of these evaluations, it has been observed that the methods and technologies used by the case company while developing Android applications have a positive effect on maintainability. The comparison between the two codebases, one developed by using the case company's methods and technologies, the other developed without a specific order and standard, showed that even for the relatively simple application features, maintainability had been increased.

As a result, the importance of maintainability for the case company's Android applications and software systems has been emphasized. In addition, the primary goal of this study, which is to measure the impact of the methods and technologies used by the case company while developing Android applications on maintainability, was reached. Although the results allow having information about the subject at the intended level, it should not be forgotten that the limitations stated in the previous section affect the study results. From this point of view, it can be said that this study is like preliminary research that led to new studies that could overcome the limits stated in the previous section. Thus more effective and efficient results could be obtained with the help of new studies where these limitations can be overcome.